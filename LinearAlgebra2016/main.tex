\documentclass[12pt]{../notes}
\usepackage{docmute}
\title{Элементы линейной алгебры}
\date{22.06.2016}
\author{\texttt{taxus}}
\graphicspath{{img/}}
\begin{document}
\maketitle

\abstract{\sl
  Сей труд не стоит рассматривать как исчерпывающий конспект лекций. 
  Скорее он представляет субъективно выбранный мною материал, 
  показавшийся или наиболее важным, или наиболее непонятым, или ещё не знаю каким.
  Надеюсь, он хоть кому-нибудь принесёт немного пользы.
}

\tableofcontents
\newpage
\setcounter{chapter}{5}
\chapter{Линейные пространства}
\documentclass[12pt]{../../notes}
\usepackage{silence}
\WarningFilter{latex}{Reference}
\graphicspath{{../../img/}}

\begin{document}

\setcounter{paragraph}{0}
\paragraph{Определения}
\begin{defn}\label{defn:linspace}
  Пусть $K$~--- поле. Рассмотрим множество  $V$ с двумя операциями
  \begin{align*}  
    + &: V \times V \to V \\
    \cdot &: K\times V \to V
  \end{align*}
  Тогда $V$~--- линейное пространство над $K$, если 
  $\forall\, \mathbf{x},\mathbf{y},\mathbf{z}\in V,\; \alpha_i \in K$
  \begin{enumerate}
    \item $(\mathbf{x}+\mathbf{y}) + \mathbf{z} = \mathbf{x} + (\mathbf{y}+\mathbf{z})$
    \item $\mathbf{x} + \mathbf{y} = \mathbf{y} + \mathbf{z} $
    \item $\exists\, \mathbf{0}\in V : \mathbf{x} + \mathbf{0} = \mathbf{x}$
    \item $\exists\, (-\mathbf{x})\in V : \mathbf{x} + (-\mathbf{x}) = \mathbf{0}$
    \item $(\alpha_1+\alpha_2) \mathbf{x} = \alpha_1 \mathbf{x} + \alpha_2 \mathbf{x}$
    \item $\alpha (\mathbf{x}_1 + \mathbf{x}_2) = \alpha \mathbf{x}_1 + \alpha \mathbf{x}_2$
    \item $1 \cdot \mathbf{x} = \mathbf{x}$
    \item $(\alpha_1 \alpha_2) \mathbf{x} = \alpha_1 (\alpha_2 \mathbf{x})$
  \end{enumerate}
\end{defn}

{ \defn\label{defn:subspace}
  Пусть $U,V$~--- линейные пространства над $K$, $U \subset V$. Тогда $U$~--- подпространство 
  $V$.
}

{ \defn\label{defn:lincomb}
Пусть $V$~--- линейное пространства над $K$, $\mathbf{x}_1, \dotsc, \mathbf{x}_n\in~V$, 
$\alpha_1, \dotsc, \alpha_n \in K$. Тогда $\alpha_1 \mathbf{x}_1 + \dotsb + \alpha_n \mathbf{x}_n$~--- 
линейная комбинация $\mathbf{x}_1, \dotsc, \mathbf{x}_n$.
}


\begin{lem}\label{lem:linspsign}
  Пусть $U,V$~--- линейные пространства над $K$, $U \subset V$. Тогда если $U$ замкнуто относительно
  $+, \cdot$  из  $V$, то  $U$~--- подпространство
  $V$.
\end{lem}
\begin{itlproof}
  Формулировка леммы аналогична тому, что всякая линейная комбинация элементов $U$ лежит в нем же.
  Нужная дистрибутивность, ассоциативность и т.д. унаследуется от соответствующих операций в 
  надпространстве, так как их свойства заданы на всём множестве $V$, а значит и на подмножестве $U$.
  Однако в некоторых свойствах требовалось существование в множестве чего-нибудь.
  Покажем, что все эти требования равносильны существованию линейной комбинации.
  \begin{enumerate}
      \setcounter{enumi}{2}
    \item $\exists\, \mathbf{0}\in U \Leftarrow \exists\, 0\cdot \mathbf{x},\; \mathbf{x}\in U$
    \item $\exists\, \mathbf{-x}\in U \Leftarrow \exists\, (-1)\cdot \mathbf{x},\; \mathbf{x}\in U$
  \end{enumerate}
\end{itlproof}

{ \defn\label{defn:linshell}
Пусть $V$~--- линейное пространства над $K$, $M \subset V$
\[
  \langle M \rangle = 
  \left\{ \alpha_1 \mathbf{x}_1 + \dotsb + \alpha_n \mathbf{x}_n \middle|
  \Big\{ \begin{array}{l}
    \alpha_1, \dotsc, \alpha_n \in K\\
    \mathbf{x}_1, \dotsc, \mathbf{x}_n \in M
  \end{array} \right\}
\] 
$\langle M \rangle$~--- линейная оболочка $M$.
}
\begin{lem}\label{lem:linshell}
  Верны утверждения:
  \begin{enumerate}
    \item $\langle M \rangle$~--- подпространство $V$
    \item $ \langle M \rangle = \bigcap\limits_i W_i$, $W_i \supset M$, $W_i$~--- подпространство $V$
  \end{enumerate}
\end{lem}
\begin{itlproof}
  Доказательства очень похожи на соответствующие в теории групп.
\end{itlproof}

\paragraph{Линейная независимость системы векторов}

{\defn\label{defn:linindp} Пусть $\mathbf{x}_1,\dotsc, \mathbf{x}_n \in V$, 
$\alpha_1, \dotsc, \alpha_n \in K$.
Тогда если 
\[
  \alpha_1\,\mathbf{x}_1 + \dotsb + \alpha_n\,\mathbf{x}_n = 0 \Leftrightarrow \forall\,i\;\alpha_i = 0
\]
то система векторов $\mathbf{x}_1, \dotsc, \mathbf{x}_n$ линейно независима.
}
\end{document}


\chapter{Матрицы}
\documentclass[12pt]{../../../notes}
\usepackage{silence}
\WarningFilter{latex}{Reference}
\graphicspath{{../../img/}}

\begin{document}

\paragraph{Матрицы, основные определения}
\begin{defn}[Матрицы над $K$]\label{defn:matrices}
  Пусть $K$"--- поле, $m,n\in \N$. Тогда
  \[
    M_{m,n}(K) = 
    \left\{
      A 
    \,\middle|\,
      A  = 
      \begin{pmatrix}
        a_{11} & a_{12} & \cdots & a_{1n} \\
        a_{11} & a_{12} & \cdots & a_{1n} \\
        \vdots & \vdots & \ddots & \vdots \\
        a_{m1} & a_{m2} & \cdots & a_{mb}
      \end{pmatrix},\;
      a_{ij} \in K
    \right\}
  \]
\end{defn}

\begin{defn}[Сложение матриц]\label{defn:matradd}
  $A_{mn}\cdot B_{mn}:$
  \[
    (a+b)_{ij} = a_{ij} + b_{ij}
  \]
\end{defn}

\begin{defn}[Умножение матриц]\label{defn:matrmul}
  $A_{mn}\cdot B_{nk}:$
  \[
    (ab)_{ij} = \sum_{l=1}^n a_{i\ell}\cdot b_{\ell j}
  \]
\end{defn}

\paragraph{Кольцо квадратных матриц}
Обозначается $M_n(K)$

\noindent Ноль:
\[
  Z_n = 
  \begin{pmatrix}
    0      & 0      & \cdots & 0 \\
    0      & 0      & \cdots & 0 \\
    \vdots & \vdots & \ddots & \vdots \\
    0      & 0      & \cdots & 0 \\
  \end{pmatrix}
\]
Единица:
\[
  E_n = 
  \begin{pmatrix}
    1      & 0      & \cdots & 0 \\
    0      & 1      & \cdots & 0 \\
    \vdots & \vdots & \ddots & \vdots \\
    0      & 0      & \cdots & 1 \\
  \end{pmatrix}
\]
\parrange{2}{Определитель}
\begin{defn}\label{defn:determinant}
  Пусть $A\in M_n(K)$
  \[
    \det A = \sum_{\sigma \in S_n} (-1)^{I(\sigma)} \cdot a_{1\sigma(1)} \dotsm a_{n\sigma(n)}
  \]
\end{defn}
\begin{defn}\label{defn:determrows}
  Если обозначать строки $A_1, \dotsc , A_n$, а столбцы $A^{(1)}, \dotsc , A^{(n)}$,
  то можно ввести ещё такую функцию:
  \[
    \det (A_1, \dotsc , A_n) = \det (A^{(1)}, \dotsc , A^{(n)}) := \det A
  \]
\end{defn}

\begin{defn}[Элементарные преобразования]\label{defn:elemtranf}
  \noindent\newline\par
  \begin{tabular}{c|l|l}
    I   & $A_i \leftrightarrows A_j$ & $A^{(i)} \leftrightarrows A^{(j)}$     \\
    II  & $A_i := A_i + \lambda A_j$ & $A^{(i)} := A^{(i)} + \lambda A^{(j)}$ \\
    III & $A_i := \lambda A_i$       & $A^{(i)} := \lambda A^{(i)}$           
  \end{tabular}
\end{defn}
\begin{defn}[Транспонированная матрица]\label{defn:transpose}
  \[
    A^T \colon (a^T)_{ij} = (a)_{ij}
  \]
\end{defn}

\subparagraph{Свойства}
\begin{enumerate}
  \item Определитель (в описанном в~\ref{defn:determrows} смысле) полилинеен и кососимметричен по
    строкам и столбцам.
  \item Если 2 строчки или столбца одинаковые, то определитель равен 0
  \item Элементарные преобразования влияют на определитель следующим образом:\par
    \begin{tabular}{c|r}
      I   & $(-1)\det A$ \\
      II  & $\det A$     \\
      III & $\lambda \det A$
    \end{tabular}
  \item $\det A^T = \det A$
\end{enumerate}


\paragraph{Теорема Лапласа}
\begin{defn}[Минор]\label{defn:minor}
  Пусть $A\in M_n(K)$, а $k\in \N$. Тогда определитель подматрицы, собранной из $k$ строк и $k$
  столбцов называется \emph{минором} порядка $k$.
  \[
    \Delta = 
    \begin{vmatrix}
      a_{i_1j_1} & \cdots & a_{i_1j_k} \\
      \vdots & \ddots & \vdots \\
      a_{i_kj_1} & \cdots & a_{i_kj_k} 
    \end{vmatrix}
  \]
  $\Delta'$~--- дополнительный минор"--- всё, что осталось.
  Его ещё иногда (когда минор"--- один элемент) обозначают как $M_{ij}$
\end{defn}
\begin{defn}[Алгебраическое дополнение]\label{defn:cofactor}
  \[
    A_\Delta = (-1)^{i_1 + \dotsb + i_k + j_1 + \dotsb + j_k} \Delta'
  \]
\end{defn}

\begin{thrm}[Теорема Лапласа]\label{thrm:laplacecofactor}
  Пусть $A\in M_n(K)$, $k\in \N$. Выберем из матрицы $k$ строчек. Тогда
  \[
    \det A = \sum_{\Delta} \Delta \cdot A_\Delta
  \]
  где $\Delta$~--- любой минор, содержащий нужные $k$ строчек.
\end{thrm}
\begin{ittproof}
  Выберем какой-то один минор, $i_k$~--- его строчки, $j_\ell$~--- его столбцы
  \[
    \Delta : 
    \begin{cases}
      i_1, \dotsc , i_k \\
      j_1, \dotsc , j_k
    \end{cases}
  \]
  Теперь отправим все элементы, попавшие в минор, в левый верхний угол.
  Сначала сдвинем все строчки:
  \[
    \begin{cases}
      i_1 \to 1 & (i_1 -1 \text{~сдвигов})\\
      \hdotsfor{2} \\
      i_k \to k & (i_k - k \text{~сдвигов})
    \end{cases}
  \]
  Потом все столбцы
  \[
    \begin{cases}
      j_1 \to 1 & (i_1 - 1 \text{~сдвигов})\\
      \hdotsfor{2} \\
      j_k \to k & (i_k - k \text{~сдвигов})
    \end{cases}
  \]
  В итоге, из свойств элементарных преобразований, получим: 
  \[
    \det B = (-1)^{i_1 - 1 + \dotsb + i_k - k + j_1 - 1 + \dotsb + j_k - k} \det A 
    = (-1)^{i_1  + \dotsb + i_k  + j_1  + \dotsb + j_k } \det A 
  \]
  так как все добавки парные $\Rightarrow$ делится на $2$.

  С другой стороны, $\Delta$ и $\Delta'$ никак не поменялись, так как переставляемые чиселки в
  них не не входят. Также нужно отметить, что $B_\Delta = \Delta'$, по тем же причинам, в
  общем-то.

  Посмотрим, что такое  $\Delta\cdot \Delta'$
  \footnote{Вообще, тут маленькая неточность: здесь фактически 
      \emph{действие} группы перестановок на множестве $\{k+1, \dotsc , n\}$}.
  \begin{align*}
    \Delta \cdot \Delta' 
    &= \Bigg( \sum_{\tau \in S_k} (-1)^{I(\tau)} b_{1\tau(1)} \dotsm
    b_{k\tau(k)} \Bigg) \cdot \Bigg( \sum_{\tau' \in S_{n-k}} (-1)^{I(\tau')} b_{k+1\tau'(k+1)} \dotsm
    b_{n\tau(n)} \Bigg)
  \end{align*}
  Пусть теперь $\sigma = \tau \circ \tau'$. Тогда $I(\sigma) = I(\tau)+I(\tau')$ по свойствам
  перестановок, а $\sigma$ фактически разбивается на 2 независимых цикла: $\tau$ и $\tau'$.
  В итоге
  \[
    \Delta \cdot \Delta' 
    = \sum_{\sigma\in S_n} (-1)^{I(\sigma)} b_{1\sigma(1)} \dotsm b_{n\sigma(n)}
  \]
  А это вообще-то правильный кусок определителя $B$. 
  Поймём, что это за кусок определителя $A$.
  Числа-то те же самые. А вот на вопрос со знаком мы уже по сути ответили, когда рассуждали про 
  определитель. Слагаемые точно не перемешиваются, так что каждое слагаемое просто умножается на
  $(-1){\cdots}$. 
  Так что 
  \begin{align*}
    \Delta \cdot \Delta' &= (-1)^{i_1  + \dotsb + i_k  + j_1  + \dotsb + j_k } 
    \sum_{\sigma'} a_{1\sigma'(1)} \dotsm a_{n\sigma'(n)} \\
    A_{\Delta} &= (-1)^{i_1  + \dotsb + i_k  + j_1  + \dotsb + j_k } \Delta'  \\
    \Delta \cdot A_{\Delta} &= \sum_{\sigma'} a_{1\sigma'(1)} \dotsm a_{n\sigma'(n)}
  \end{align*}
  где у $\sigma'$ уже другие независимые циклы: ${i_1, \dotsc , i_k\choose j_1, \dotsc , j_k }$ 
  и всё остальное.

\end{ittproof}
\begin{imp}
  \[
    \det A = a_{i1} A_{i1} + \dotsb + a_{in} A_{in} 
  \]
\end{imp}
\begin{imp}
  \[
    a_{j1} A_{i1} + \dotsb + a_{jn} A_{in} = 0
  \]
\end{imp}
\begin{itlproof}
  Приравняем $i$ строчку к $j$-ой, получим матрицу $B$. Тогда
  \[
    a_{j1} A_{i1} + \dotsb + a_{jn} A_{in} = b_{i1} B_{i1} + \dotsb + b_{jn} B_{in} = \det B = 0
  \]
  Так как определитель $B$ очевидно равен 0
\end{itlproof}

\paragraph{Ступенчатая матрица}

\begin{defn}\label{defn:stairsmtx}
  $A$ = \raisebox{-0.45\height}{\includegraphics[scale=0.6]{stairsmtx}}~--- ступенчатая 
  матрица.
\end{defn}
\begin{thrm}[Определитель ступенчатой матрицы]\label{thrm:stairsdet}
  \[
    \det A = \det A_1 \dotsm \det A_n
  \]
\end{thrm}
\begin{ittproof}
  По индукции через теорему Лапласа~\ref{thrm:laplacecofactor}
\end{ittproof}

\paragraph{Определитель произведения матриц}
\begin{thrm}\label{thrm:detmult}
  Пусть $A,B \in M_n(K)$. Тогда
  \[
    \det (AB)  = \det A \cdot \det B
  \]
\end{thrm}
\begin{ittproof}
  Докажем, что такие матрицы имеют одинаковый определитель:
  \[
    C = 
    \left(
    \begin{array}[h]{c|c}
      A    & 0 \\
      \hline 
      -E_n & B
    \end{array}
    \right) \quad \text{и} \quad 
    D = 
    \left(
    \begin{array}[h]{c|c}
      AB    & A \\
      \hline 
      0 & -E_n
    \end{array}
    \right)
  \]


\end{ittproof}


\end{document}


\chapter{Линейные операторы}
%------------------------------------------------------------
% Description : 
% Author      : tis-p30 <iliya.t@mail.ru>
% Created at  : Sat Jun 18 12:38:13 MSK 2016
%------------------------------------------------------------
\documentclass[12pt]{../../../notes}
\usepackage{silence}
\WarningFilter{latex}{Reference}
\graphicspath{{../../img/}}

\begin{document}
<++>
\end{document}


\begin{thebibliography}{9}
\addcontentsline{toc}{section}{Использованная литература}
  \bibitem{vinberg}
  \textbf{Винберг~Э.~Б.} \\ 
  Курс алгебры.~---
  2-е изд., стереотип.~---
  М.:~МЦНМО, 2013.~---
  592~с.:~ил.
\end{thebibliography}
\end{document}


















