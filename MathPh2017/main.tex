%%&electrodcheat
\documentclass{trchesh}
\usepackage{trmath}
\usepackage{trsym} 

\setlayout[page-orient=landscape, size=10]{hardcopy}
\makeatletter
\def\note{\ifexpmode{\@gobble}{\endnote}}
\makeatother
% \usepackage{tikz}
% \usepackage{pgfplots}
% \pgfplotsset{compat=1.11}
% \usetikzlibrary{external,calc,arrows}
% \tikzsetexternalprefix{img/}
% \tikzexternalize%
% \usepackage{docmute}
\hypersetup{
  pdftitle={Шпора по матфизике},
  pdfauthor={taxus},
  pdfsubject={Матфизика},
  pdfkeywords={Матфизика;экзамены;СПбГУ;by-taxus}
}
\columnseprule=0.1dd
\def\arraystretch{1.3}
\setlist[enumerate,1]{leftmargin=1.6em, noitemsep}
\setlist[itemize,1]{leftmargin=2em, label=$\triangleright$}

\def\waveop{\mathop{\boldsymbol\Box}}

\def\defsign{\hbox{$\,\Rrightarrow\,$}}
\newcommand{\deflabel}[1]{
  \makebox[\labelwidth][l]{%
    \parbox[t]{\labelwidth}{\hspace{0pt}\textsf{#1}}%
  }~\defsign
}
\newenvironment{defs}[1][\hspace{12ex}]%
  {\begin{list}{}{%
        \let \makelabel=\deflabel%
        \setlength{\labelwidth}{\widthof{#1}} %
        \setlength{\leftmargin}{\labelwidth+\labelsep}%
        \itemsep=0pt %
      }}%
  {\end{list}}
\newenvironment{facts}{\begin{list}{$\triangleright$}{}}{\end{list}}
% \csuse{endofdump}
\graphicspath{{img/}}

\renewcommand{\notesname}{Примечания}
\begin{document}
\parindent=0pt
\parskip=0.2\baselineskip

\section{Волновое уравнение}
\paragraph{Классификация уравнений второго порядка}

Выглядят так
\[
  \mathcal L[u] = \sum_{i,j} A^{ij} \partial^2_{ij} u + \sum_{i} B^i \partial_i u + C u = 0
\]
В зависимости от собственных чисел $A$ классифицируются
\begin{defs}[гиперболический]
  \item[эллиптический] $\forall\,i \holds \Lambda_i > 0 $
  \item[параболический] $\exists\,j \that \Lambda_j = 0, \quad \forall\,i\neq j \holds \Lambda_i > 0$
  \item[гиперболический] $\exists\,j \that \Lambda_j < 0, \quad \forall\,i\neq j \holds \Lambda_i > 0$
\end{defs}
И их канонические формы
\begin{align*}
  \waveop u &= 0 & &\defsign &&\text{волновое уравнение} \\
  \Delta u  &= 0 & &\defsign &&\text{уравнение Лапласа} \\
  (-\partial_t + a^2 \Delta) \, u &= 0 & &\defsign &&\text{уравнение теплопроводности}
\end{align*}

\paragraph{Характеристические поверхности}
Здесь непросто, потому что обычно характеристики определяют для уравнений первого 
порядка. А тут нужно как-то факторизовать. 

Впрочем, можно сказать, что характеристическая поверхность~"--- $(w_x, A w_x) =0$
Здесь $\xi = w (x,y)$, и для тех типов уравнений, что мы рассматриваем, верно
\begin{facts}
\item $\omega \equiv \mathrm{const}$
\item при замене переменных $\xi = \omega(x,y)$ член при $u_{\xi\xi}$ зануляется%
  \note{
    У нас было всё наоборот, но так ещё непонятней что происходит
  }
\end{facts}

\paragraph{Волновое уравнение}

Уравнение и начальные условия:
\begin{align*}
  u_{tt} - a^2 u_{xx} &= 0 \\
                & u(x,0) = \varphi_0(x) \\
                & u_t(x,0) = \varphi_1(x) 
\end{align*}

Решение Даламбера 
\[
  u(x,t) = \frac12 \left( \varphi_0 (x - at) + \varphi_0(x+at) + 
  \frac{1}{a} \int_{x-at}^{x+at} \varphi_1 (\xi)\, \del\xi \right)
\]

\paragraph{Принцип Дюамеля}
Основная мысль в том, чтобы научиться получать из решения однородного уравнения
решение неоднородного неочевидным способом.
\begin{enumerate}
  \item $\waveop P = 0$, $P(x,t,t) = 0$, $P_t(x,t,t) = f(x,t)$ (если существует)
  \item $w = \dint_0^t P(x,t,t') \, \del t'$
  \item $\waveop w = f(x,t)$
\end{enumerate}
Для волнового уравнения $P = \frac{1}{2a} \int\limits_{x-a(t-t')}^{x+a(t-t')} f(\xi, t')\,\del\xi$


\paragraph{Энергетическое неравенство}

\plholdev{картиночка}
\begin{enumerate}
  \item $\Omega_t$~--- срез конуса причинности 
    (характеристическая поверхность волнового уравнения).
  \item $E[u] = u_t^2 + a^2 u_x^2$
\end{enumerate}
Само энергетическое неравенство
\[
  \int_{\Omega_0} E[u] \, \del x \geqslant \int_{\Omega_t} E[u] \, \del x
\]

Отсюда можно заполучить единственность решения волнового уравнения.

\paragraph{Формула Кирхгофа (\texorpdfstring{$\R^3$}{3D})}

\[
  u(x,t) = t \fint_{\Ball{at}{x}} \varphi_1 (y) \,\del S + 
  \pder{}{t} \left( t \fint_{\Ball{at}{x}} \varphi_0(y) \, \del S \right)
\]

Здесь видно, что что возмущения (начальные) влияют на решение только будучи на границе
конуса прчинности. Получается, что волна не <<запоминает>> своё прошлое.
То есть, возмущение приходит с волновым фронтом и уходит с ним. 

Это не принцип Гюйгенса-Френеля (тот про функцию Грина), но тоже принцип Гюйгенса.

\paragraph{Формула Пуассона (\texorpdfstring{$\R^2$}{2D})}

\[
  \begin{split}
    u(x,t) &= 
    \frac{t^2}{2} \fint_{\Ball{at}{x}} \frac{\varphi_1(y)}{\sqrt{(at)^2 - |x-y|^2}} \,\del y + \\
    &+ \pder{}{t} \left( \frac{t^2}{2} \fint_{\Ball{at}{x}} \frac{\varphi_0(y)}{\sqrt{(at)^2 - |x-y|^2}} \, \del y \right)
  \end{split}
\]

Здесь возмущение <<чувствуется>> даже после прохождения фронта. В каком-то смысле
оно <<размывается>>. Видимо, это и есть упомянутая диффузия?

\section{Теплопроводность}

\paragraph{Вывод уравнения}
\begin{enumerate}
  \item Уравение неразрывности: $u_t = - \div \v F $
  \item Связь потока с текущим веществом $\v F \propto -\grad u$
\end{enumerate}
\[
  u_t + a^2 \Delta u = 0 
\]

Характерные множества
\begin{defs}[параболический цилиндр $(Q_T)$]
\item[параболический цилиндр ($Q_T$)] $\Omega \times (0,T)$, $\Omega \subset \R^n$
  % ------------------------------------------------------V very tricky, but looks like no other option
\item[параболическая граница ($\boundary' Q_T$)] \makebox[9em]{$(\Omega \times \{0\}) \cup \boundary \Omega \times [0;T]$}
\end{defs}
Для удобства $R_T := Q_T ( \Omega = \R^n)$.

Разные задачи:
\begin{align}
  & u \in C^2(\R^n) \,\bintersect\, C^1(0;T) \,\bintersect\, C(\boundary'R_T),& 
  u(x,0) &= \varphi\\
  & u \in C^2(\Omega) \,\bintersect\, C^1(0;T) \,\bintersect\, C(\boundary'Q_T),&
  u \big|_{\boundary'Q_T} &= \varphi
\end{align}

\paragraph{Закон сохранения}
При довольно мутных условиях (что-то вроде в меру быстрого убывания $u$ к бесконечности)
\[
  \mathcal U(t,R) :=\int_{\Ball{R}{0}} u (x,t) \, \del x = \mathrm{const}
\]

\paragraph{Ограниченный принцип максимума}
\label{par:heat::lmax}
\[
  \inf_{\boundary'Q_T} u (x,t) \leqslant u(x,t) \leqslant \sup_{\boundary'Q_T} u(x,t)
  \quad(\text{в $Q_T$})
\]

\paragraph{Принцип максимума в полупространстве}
\label{par:heat::tmax}
Пусть $u$ ограничена сверху\note{в Эвансе $\leqslant Ae^{a|x|^2}$, например}
\[
  \inf_{\R^n} u (x,t) \leqslant u(x,t) \leqslant \sup_{\R^n} u(x,t)
\]

\paragraph{Единственность}
кажется, это очевидно следует из \ref{par:heat::lmax}, \ref{par:heat::tmax}.


\paragraph{Автомодельные решения}
\begin{facts}
\item $\xi = \frac{x}{\sqrt{t}}$
\item $v(\xi) = c \int_0^\xi e^{-\xi^2/4 a^2}\, \del \xi$
\end{facts}

\paragraph{Функция источника (одномерье)}
\[
  u(x,t) = \frac{1}{2a\sqrt{\pi t}} \, e ^{-\frac{x^2}{4a^2t}}
\]
\paragraph{Функция источника (многомерье)}
\[
  G(x,t) = \frac{1}{(2a\sqrt{\pi t})^n} \, e ^{-\frac{|x|^2}{4a^2t}}
\]

\paragraph{Свойства функции источника}

\end{document}
%vim: syntax sync minlines=200
