\documentclass[10pt, timbord]{../notes}
\usepackage{docmute}

\title{Определения из дифуров}
\date{\today}
\author{Лектор: В.~В.~Басов \\
        Записал: \texttt{taxus}}

\DeclareMathOperator{\Lip}{Lip}
\newcommand{\liploc}[1]{\Lip_{#1}^{\mathrm{loc}}} 
\newcommand{\lipgl}[1]{\Lip_{#1}^{\mathrm{gl}}} 
\usepackage{xstring}

\newcommand*\ov[2]{
  \IfEqCase{#1}{%
    {~}{\widetilde{#2}}%
    {-}{\overline{#2}}%
    {^}{\widehat{#2}}%
  }[\PackageError{tmath}{Undefined option to ov: #1}{}]%
}

\newcommand*\und[2]{
  \IfEqCase{#1}{%
    {-}{\underline{#2}}%
  }[\PackageError{tmath}{Undefined option to und: #1}{}]%
}

\newcommand*\arr[1]{
  \IfEqCase{#1}{%
    {<=}{\Leftarrow}%
    {=>}{\Rightarrow}%
    {<=>}{\Leftrightarrow}%
    {<==>}{\Longleftrightarrow}%
  }[\PackageError{tmath}{Undefined option to arr: #1}{}]%
}

\begin{document}

\maketitle

\chapter{Дифуры, разрешённые относительно производной}

\paragraph{Основные определения}

\begin{defn}[Дифференциальное уравнение]\label{defn:diffeq}
  Пусть $f \in C(G)$, $G \subset \R^2$. Тогда диффур~--- вот такая штука:
  \[
    y' = f(x, y)
  \]
\end{defn}
\begin{defn}\label{defn:diffsol}
  Решение~--- функция $y=\varphi(x)$, определённая на $\langle a, b \rangle$ :
  \begin{enumerate}
    \item $\varphi(x)$~--- дифференцируема
    \item $(x, \varphi(x)) \in G$
    \item $\varphi'(x) = f(x, \varphi(x))$
  \end{enumerate}
\end{defn}

\begin{defn}[Задача Коши и вокруг неё]\label{defn:cauchyprobl} 
  Основные понятия тут:
  \begin{enumerate}
    \item $(x_0, y_0)$~--- начальные данные
  \item Решение задачи Коши~--- \emph{частное} решение дифура + выполнение начальных условий
    \item Решение задачи Коши существует, если $\exists\, (a,b) \ni x_0,
      y=\varphi(x)\colon y_0 = \varphi(x_0)$
    \item Решение задачи Коши единственно, если любые 2 решения совпадают в окрестности
      $x_0$ 
  \end{enumerate}
\end{defn}

\begin{defn}\label{defn:uniqpnt}
  $(x_0, y_0) \in G$~--- тоска единственности, если решение задачи Коши в ней единственно.
\end{defn}
\begin{defn}\label{defn:uniqset}
  $\widetilde{G} \in G$~--- область единственности, если каждая её точка~--- точка
  единственности.
\end{defn}

\begin{defn}\label{defn:inteq}
  Пусть $f\in C(G)$, $G\subset\R^2$, $(x_0, y_0) \in G$, $x_0, x\in \langle a, b \rangle$
  Тогда интегральным уравнением называется 
  \[
    y = y_0 + \int_{x_0}^{x} f(u, y)\, \mathrm d s
  \]
\end{defn}

\begin{defn}\label{defn:intsol}
  Решение~--- функция $y=\varphi(x)$, определённая на $\langle a, b \rangle$ :
  \begin{enumerate}
    \item $\varphi(x)$~--- непрерывна
    \item $(x, \varphi(x)) \in G$
    \item $ y = y_0 + \dint_{x_0}^{x} f(u, y)\, \mathrm d s $
  \end{enumerate}
\end{defn}

\begin{thrm}\label{thrm:diffinteqconnect}
  Пусть $f\in C(G)$. Тогда $y=\varphi(x)$~--- решение \ref{defn:diffeq} $\Leftrightarrow$ решение \ref{defn:inteq}.
\end{thrm}

\begin{defn}[Отрезок Пеано]\label{defn:peanosegm}
  \begin{align*}
    P_h(x_0, y_0) &= \{x\colon |x-x_0| \leqslant h\} \\
    h &= \min \{a, b/M\},\; M = \max_{(x, y)\in \overline R} | f(x, y) | \\
    \overline R &= \{ (x, y) \colon |x-x_0| \leqslant a, | y-y_0| \leqslant b\} \subset G
  \end{align*}
\end{defn}

\begin{defn}\label{defn:partsol}
  Решение~--- частное, если всякая его точка~--- точка единственности.
\end{defn}

\begin{defn}\label{defn:singsol}
  Решение~--- особое, если всякая его точка~--- точка неединственности.
\end{defn}

\begin{defn}[Общее решение]\label{defn:generalsol}
  Пусть $G$~--- область единственности. Тогда непрерывная по обеим аргументам функция $y = \varphi(x,C)$~--- общее решение 
  дифференциального уравнения в $A \subset G$, если $\forall (x_0, y_0) \in A $:
  \begin{enumerate}
    \item  уравнение $y_0 = \varphi(x_0, C)$ однозначно разрешимо относительно $C$. \\
      /Наверное, локальная монотонность нужна/
    \item $y=\varphi(x, C_0)$ ($C_0$~--- корень $y_0=\varphi(x_0, C)$)~--- решение задачи Коши с начальными данными $x_0, y_0$
  \end{enumerate}
  Тут кстати в самом конспекте что-то странное. Почему-то требуется, чтобы $\varphi(x,C_0)$ была решением в окрестности $C_0$.
\end{defn}

\begin{defn}\label{defn:solcont}
  Решение дифура, определённое на $\langle a, b \rangle$ продолжимо за точку $b$, если 
  \[
    \exists\, \widetilde{\varphi}(x) \colon \widetilde{\varphi}\vert_{\langle a, b \rangle} = \varphi
    \land \widetilde{\varphi}'(x) \overset{\langle a, \widetilde b \rangle}{\equiv} f(x, \widetilde{\varphi}(x))
  \]
\end{defn}


\paragraph{Существование решения задачи Коши}

\begin{defn}[Ломаная Эйлера]\label{defn:eulerlin}
  \[
    \begin{cases}
      \forall\, x\in (x_j, x_{j+1}], j\in 0\mathbin{..} {N-1}  \\
      \forall\, x\in [x_{j-1}, x_{j}), j\in {1-N} \mathbin{..} 0
    \end{cases}
    \psi(x) = \psi(x_j) + f(x_j, \psi(x_j))(x-x_j)
  \]
\end{defn}

\begin{thrm}[Теорема Пеано]\label{thrm:peanothrm}
  Пусть $f\in C(G)$. Тогда
  \[
    \forall\, (x_0, y_0), P_h(x) \;\: \exists\, y=\varphi(x) \colon
    \begin{cases}
      \varphi'(x) \overset{P_h}{\equiv} f(x, \varphi(x)) \\
      y_0 = \varphi(x_0)
    \end{cases} \text{(решение \ref{defn:cauchyprobl})}
  \]
\end{thrm}

\begin{lem}[Лемма Гронуолла]\label{lem:gron}
  Пусть $h\in C\bigl(\langle a;b\rangle\bigr)$. Тогда 
  \[
    \exists\, \lambda, \mu \geqslant 0 \colon \forall\, x_0, x \in \langle a;b\rangle \;\:\: 
    0 \leqslant h(x) \leqslant \lambda + \mu \left| \int_{x_0}^x h(s) \, \del s \right| 
    \arr {=>} h(x) \leqslant \lambda e^{\mu |x-x_0|} 
  \]
\end{lem}

\chapter{Диффуры в симметричной форме}
\label{cha:simm}


\begin{defn}\label{defn:simaccept}
  Пусть $G$~--- область единственности. Тогда $U\colon G\to \R$, $U \in C(G)$~--- допустимая
  функция, если $U(x,y) - U(x_0, y_0)$ задаёт единственную неявную функцию $y = \varphi(x)$ или $x =
  \psi(y)$.
\end{defn}

\begin{defn}\label{defn:simint}
  Допустимая функция называется интегралом уравнения, если неявная функция которую она задаёт~---
  решение ЗК с начальными данными $(x_0; y_0)$.
\end{defn}


\begin{thrm}\label{thrm:constsol}
  Если допустимая функция постоянна вдоль любого решения в области единственности , то она~---
  интеграл. Обратное тоже верно.
\end{thrm}

\begin{defn}\label{defn:smoothacc}
  Гладкая допустимая функция~--- $U\in C^1(G)\colon (U_x')^2 + (U_y')^2 > 0$ 
\end{defn}

\begin{thrm}\label{thrm:smoothint}
  $U$~--- гладкий интеграл в области единственности $G$ $\Leftrightarrow$
  \[
    N(x,y)U_x'(x,y) - M (x, y) U_y' (x, y) \overset{G}{\equiv} 0.
  \]
  и $U$~--- гадкая допустимая функция.
\end{thrm}

\begin{defn}\label{defn:genint}
  Если $U$~--- интеграл, то $U(x,y) = C$~--- общий интеграл.
\end{defn}

ещё теормемы:
\begin{thrm}\label{thrm:simintexist}
  Все эти интегралы~--- существуют.
\end{thrm}
\begin{ittproof}
  Ну, это же почти общее решение.
\end{ittproof}

\begin{thrm}\label{thrm:intconn}
  Пусть $U$~--- интеграл, $U_1$~--- допустимая функция. Тогда $U_1$~--- интеграл $\Leftrightarrow$
  \[
    \exists\, \Phi, A\colon U_1 \overset A \equiv \Phi(U)
  \]
\end{thrm}


\chapter{Нормальные и не очень системы диффуров}
\paragraph{Условия Липшица}


\begin{lem}[Лемма Адамара]\label{lem:sys::ad}
  Пусть $f(x,y) \colon \R^l \times \R^m \to \R^n$, $G$~--- область, выпуклая по $y$.
  Пусть также $f(x,y), \pder{f}{y^i} \in C(G)$. Тогда \[
    \exists\, h(x,y_1, y_2) \colon \R^l\times\R^m\times\R^m \to \R^m\times\R^n \;\:
    f(x,y_2) - f(x, y_1) = h(x, y_1, y_2)^T (y_2 - y_1) 
  \]
  где $\displaystyle h(x, y_1, y_2) = \int_0^1 \frac{\partial f(x, u(s))}{\partial y} \, \del s$, \
  а $u(s) = y_1 + s(y_2 - y_1)$
\end{lem}

\begin{lem}\label{lem:sys::lipcon}
  Пусть $f(x,y) \in \liploc{y}(G)$, $G$~--- область. Тогда $\forall \ov-{H} \subset G 
  \;\: f\in \lipgl{y}$. 
\end{lem}

\paragraph{Пикаровские приближения}
\label{par:pik}


\begin{defn}[Пикаровские приближения]\label{defn:pik}
  \begin{align*}
    y^{(0)} &\equiv y_0 \\
    y^{(k+1)} &= y_0  + \int_{x_0}^x f(s, y^{(k-1)}(s))\, \del s 
  \end{align*}
\end{defn}

\begin{thrm}[Пикара]\label{thrm:pik}
  Пусть $f(x,y) \in \liploc{y}(G)$ и 
  \[
    \forall\, (x_0, y_0) \in G, k \;\:\: \exists\, \ov{-} H \subset G, y^{(k)}(x) \colon [a;b] 
    \to \R^n \:\colon \: (x, y^{(k)}(x)) \in \ov{-} H  \land y^{(k)}(x) \to y(x)
  \]
  которое уже и является решением задачи Коши с НД $(x_0, y_0)$ на $[a;b]$. Здесь $y^{(k)}$~--- пикаровские
  приближения.
\end{thrm}
\begin{ittproof}
  В процессе последовательность заменяют рядом $\varphi^{(k)}$, а потом доказывают, что 
  \[
    \|\varphi^{(k)}(x)\| \leqslant \frac{M}{L} \frac{(L|x-x_0|)}{k!}
  \]
\end{ittproof}

\begin{thrm}\label{thrm:pikex}
  Пусть $f\in \liploc{y}(G)$. Тогда отрезок Пеано покатит в качестве отрезка $[a;b]$ в теореме выше,
  а компакт можно выбрать так : $\ov{-} H = \{(x,y) \mid x\in [a;b], \|y-y_0\| \leqslant b\}$ .
\end{thrm}

\paragraph{Почти линейные системы}

\begin{defn}\label{defn:plin}
  Пусть $f(x,y) \in C(G)$, $G = (a;b) \times \R^n$, $L,M$~--- непрерывны и неотрицательны. Тогда
  почти линейная система выглядит так:
  \[
    \forall\,(x,y) \in G \quad \|f(x,y)\| \leqslant L(x) + M(x) \|y\|
  \]
\end{defn}

\end{document}
