\documentclass{trlnotes}
\setlayout{hardcopy}
\usepackage{silence}
\WarningFilter{latex}{Reference}
\graphicspath{{../../img/}}

\begin{document}
    \paragraph{Интегральное уравнение II рода, метод замены ядра на вырожденное}

    \begin{de}
        \ti{Интегральным уравнением Фредгольма II рода} называется уравнение вида
        \[
            \varphi(x) = f(x) + \mu \int\limits_a^b K(x, \, t) \varphi(t) \, \del t.
        \]
        Функция $K$~--- его \ti{ядро}, а $\mu$~--- \ti{характеристическое число}.\footnote{Кажется, иногда в определении полагают $\mu = 1$, но всегда ведь можно внести его в ядро.}
    \end{de}

    Обозначим через $K$ (хм, да, вольность) оператор 
    \[
        \varphi(t) \mapsto \int\limits_a^b K(x, \, t) \varphi(t) \, \del t.
    \]
    Ясно, что он компактен. Уравнение теперь примет вид
    \[
        (I - \mu K)\varphi = f.
    \]
    Оператор $T = I - \mu K$, конечно, фредгольмов.

    \begin{st}
        Сопряжённый в $L^2\big([a, \, b]\big)$ оператор к $K$ выражается следующим образом:
        \[
            K^* \varphi(x) = \int\limits_a^b \ov{-}{K(t, \, x)} \varphi(t) \, \del t.
        \]
        \begin{proof}
            Прямым вычислением (ну, там внутри ещё теорема Фубини) проверяется, что 
            \[
                \langle K \varphi, \, \psi \rangle = \langle \varphi, \, K^{*} \psi \rangle.
            \]
        \end{proof}
    \end{st}

    \begin{rem}
        У ядра меняются местами аргументы и оно сопрягается~--- точно так же, как транспонирование вместе с комплексным сопряжением дают матрицу сопряжённого оператора в конечномерном случае!
    \end{rem}

    Сформулируем альтернативу Фредгольма \ref{thm:fred-alt} для такого уравнения:

    \begin{st}
        $\hphantom{.}$
        \begin{enumerate}
            \item Уравнение $T\varphi = f$ разрешимо однозначно тогда и только тогда, когда $\mu^{-1}$~--- не собственное число оператора $K$.
            \item В противном случае уравнение $T \varphi = f$ разрешимо тогда и только тогда, когда функция $f$ ортогональна всем собственным векторам оператора $K^*$, соответствующим числу $\ov{-}{\mu}^{-1}$.
            \item $\mu^{-1}$ и $\ov{-}{\mu}^{-1}$~--- собственные числа операторов $K$ и $K^*$ соответственно одинаковой конечной кратности.
        \end{enumerate}
    \end{st}
\end{document}
% vim:wrapmargin=3
