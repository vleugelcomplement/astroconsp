% \documentclass{trlnotes}
% \setlayout{hardcopy}
% \usepackage{silence}
% \WarningFilter{latex}{Reference}
% \graphicspath{{../../img/}}
\documentclass{trlnotes}
\usepackage{trmath}
\addcompatiblelayout{commonplace}
\setlayout{commonplace}
\usepackage{trthm}
\usepackage{trsym} 
\usepackage{trphys}
% \RequirePackage{stmaryrd}

% \RequirePackage{etoolbox}


%\RequirePackage{cmap}
% \RequirePackage{hyperref}
\theoremstyle{definition}
\newtheorem{thm}{Теорема}
\newtheorem{de}{Определение}
\newtheorem{lm}{Лемма}
\newtheorem{exm}{Пример}
\newtheorem{pr}{Свойство}
\newtheorem{exc}{Упражнение}
\newtheorem{cor}{Следствие}
\newtheorem{st}{Утверждение}
\theoremstyle{remark}
\newtheorem{rem}{Замечание}
\newcommand*{\icom}{\ensuremath\text{\textit{,}}}
\newcommand*{\icol}{\ensuremath\text{\textit{:}}}
\newcommand*{\iscol}{\ensuremath\text{\textit{;}}}
\newcommand*{\com}{\ensuremath\text{\text{,}}}
\newcommand*{\col}{\ensuremath\text{\text{:}}}
\newcommand*{\scol}{\ensuremath\text{\text{;}}}
% \newcommand*{\so}{\ensuremath\Rightarrow}
\newcommand*{\bso}{\ensuremath\Leftarrow}
\newcommand*{\eqv}{\ensuremath\Leftrightarrow}
\newcommand*{\all}{\forall}
\newcommand*{\ex}{\exists \,}
% \newcommand*{\ov}{\overline}
\newcommand*{\un}{\underline}
\newcommand*{\ova}{\overrightarrow}
\newcommand*{\auth}[1]{\hfill \textit{#1}}
\newcommand*{\pd}{\partial}
% \newcommand*{\R}{\mathbb{R}}
% \newcommand*{\N}{\mathbb{N}}
\newcommand*{\DD}{\mathbb{D}}
\newcommand*{\K}{\mathbb{K}}
% \newcommand*{\C}{\mathbb{C}}
% \newcommand*{\C}{\mathbb{C}}
% \newcommand*{\Q}{\mathbb{Q}}
\newcommand{\An}{\wedge}
\newcommand{\Or}{\vee}
% \newcommand*{\Z}{\mathbb{Z}}
\newcommand*{\J}{\mathbb{J}}
\newcommand*{\bas}[2]{\overset{\vspace{-3pt}\tiny{\mb{#1}}}{#2}}
\newcommand*{\mc}{\mathcal}
\newcommand*{\mb}{\mathbf}
\newcommand*{\mf}{\mathfrak}
\newcommand*{\ti}{\textit}
\newcommand*{\la}{\langle}
\newcommand*{\ra}{\rangle}
\newcommand*{\tb}{\textbf}
\newcommand*{\mr}{\mathrm}
\newcommand*{\wt}{\widetilde}


\renewcommand{\arraystretch}{1}

\counterwithin{thm}{section}
\counterwithin{de}{section}
\counterwithin{lm}{section}
\counterwithin{st}{section}
\counterwithin{exm}{section}
\counterwithin{pr}{section}
\counterwithin{cor}{section}
\counterwithin{rem}{section}
\renewcommand{\thethm}{\arabic{section}.\arabic{thm}}
\renewcommand{\thede}{\arabic{section}.\arabic{de}}
\renewcommand{\thelm}{\arabic{section}.\arabic{lm}}
\renewcommand{\thest}{\arabic{section}.\arabic{st}}
\renewcommand{\theexm}{\arabic{section}.\arabic{exm}}
\renewcommand{\therem}{\arabic{section}.\arabic{rem}}
\renewcommand{\thepr}{\arabic{section}.\arabic{pr}}
\renewcommand{\thecor}{\arabic{section}.\arabic{cor}}



\counterwithin{subsection}{section}


% \DeclareMathOperator{\rk}{rk}
\DeclareMathOperator{\Dom}{Dom}
% \DeclareMathOperator{\id}{id}
\DeclareMathOperator{\Cl}{Cl}
% \DeclareMathOperator{\Res}{Res}
\DeclareMathOperator{\im}{Im}
% \DeclareMathOperator{\rot}{rot}
\DeclareMathOperator{\Div}{div}
% \DeclareMathOperator{\grad}{grad}
\DeclareMathOperator{\Id}{Id}
% \DeclareMathOperator{\Aut}{Aut}
% \DeclareMathOperator{\Stab}{Stab}
\DeclareMathOperator{\const}{const}

%\titleformat{\section}
%  {\sffamily\mdseries\upshape\LARGE}
%  {Билет \thesection:}{0.5em}{}



\usepackage{silence}
\usepackage{tikz}
\WarningFilter{latex}{Reference}
\graphicspath{{../../img/}}
\makeatletter
\let\xymatrix\@gobble
\makeatother
\begin{document}
\paragraph{Разностный метод для общего уравнения теплопроводности, явная схема}

\begin{de}
	Общее \ti{уравнение теплопроводности} выглядит вот так:
	\[
		\dfrac{\pd u}{\pd t} = a_0 \dfrac{\pd^2 u}{\pd x^2} + a_1 \dfrac{\pd y}{\pd x} + a_2 u + f.
	\]
	Функции $a_i$ и $f$ зависят от $x$ и $t$.
\end{de}

Работать будем, как всегда, на отрезке $[a, \, b]$; временной отрезок будет $[0, \, T]$.

\begin{de}
	У уравнения теплопроводности бывает \ti{начальное условие}:
	\[
		u(x, \, 0) = \varphi(x),            
	\]
	а также три типа \ti{граничных условий}
	\begin{enumerate}
		\item $u(a, \, t) = \psi_0(t)$, $u(b, \, t) = \psi_1(t)$.
		\item $\dfrac{\pd u}{\pd x}(a, \, t) = \psi_0(t)$, $\dfrac{\pd u}{\pd x}(b, \, t) = \psi_1(t)$.
		\item $\dfrac{\pd u}{\pd x} - \alpha u \big|_{x = a} = \psi_0(t)$, $\dfrac{\pd u}{\pd x} - \beta u \big|_{x = b} = \psi_1(t)$.
	\end{enumerate}
\end{de}
Сетка характеризуется такими же, как обычно, величинами:
\[
	\begin{array}{lll}
		x_i = a + ih, & h = \dfrac{b - a}{h}, & i \in 0\ldots n; \\
		t_k = k\tau, & \tau = \dfrac{T}{M}, & k \in 0 \ldots M.
	\end{array}
\]

Положим $u_i^k = u(x_i, \, t_k)$ и
\[
	Lu = a_0 \dfrac{\pd^2 u}{\pd x^2} + a_1 \dfrac{\pd y}{\pd x} + a_2 u.
\]
Тогда
\[
	(\tilde{L}u)_i^k = a_0 \dfrac{u_{i+1}^k - 2u_i^k + u_{i - 1}^k}{h^2} + a_1 \dfrac{u_{i + 1}^k - u_{i - 1}^k}{2h} + a_2 u_i^k.
\]
Есть два варианта для производной по времени:
\begin{align}\label{eq:therm-A-B}
				&\text{A:} \quad \dfrac{\pd u}{\pd t}(x_i, \, t_k) \approx \dfrac{u^{k + 1}_i - u^k_i}{\tau}, \\
				&\text{B:} \quad \dfrac{\pd u}{\pd t}(x_i, \, t_k) \approx \dfrac{u^{k}_i - u^{k-1}_i}{\tau}.
\end{align}
Для варианта A получается 
\[
	\boxed{\dfrac{u_i^{k + 1} - u_i^k}{\tau} = \tilde{L}u_i^k + f(x_i, \, t_k)}\,.
\]
Это простейшая явная схема. 

\begin{figure}[h] \label{fig:therm-simple}
	\begin{center}
		\includegraphics[scale=0.9]{../img/pde/therm-simple.pdf}
	\end{center}
	\caption{Простейшая явная схема для уравнения теплопроводности.}
\end{figure}

В таком виде уравнения можно писать для $i \in 1\ldots n-1$, $k\in 0\ldots M-1$; нужны дополнительные с граничными условиями. 

\begin{itemize}
	\item Начальные условия: $u_i^0 = \varphi(x_i)$.
	\item Граничные условия:
		\begin{enumerate}
			\item $u_0^k = \alpha_1(t_k)$, $u_n^k = \alpha_2(t_k)$; при этом выполняются условия согласования \ti{нулевого порядка}
				\[
					\varphi(a) = \alpha_1(0), \quad \varphi(b) = \alpha_2(0).
				\]
			\item Для типов II, III используются такие же трюки, как в обычных диффурах. Надо аппроксимировать производные. Можно применять метод фиктивных точек или метод исключения главного члена погрешности.

				В угловых точках снова возникнет два разных условия:
				\[
					u_0^0 = \varphi(a) \text{ и } \dfrac{\pd u}{\pd x}(a, \, 0) = \beta_1(0) u_0^0 + \alpha_1(0).
				\]
				Будет ли выполняться равенство
				\[
					\varphi'(a) = \beta_1(0) u_0^0 + \alpha_1(0)?
				\]
				Оно называется \ti{условием согласования I порядка}. Без него уравнения не станут формально противоречивы.
		\end{enumerate}
\end{itemize}

Если разрешить уравнения относительно $u_i^{k + 1}$, получится 
\[
	u_{i}^{k + 1} = A_i^k u_{i-1}^k + B_i^k u_i^k + C_i^k u_{i+1}^k + D_i^k.
\]
Коэффициенты выражаются по формулам
\[
	\begin{array}{ll}
		A_i^k = \sigma a_0 - \sigma a_1 \dfrac{h}{2}, & C_i^k = \sigma a_0 + \sigma \dfrac{h}{2} a_1, \\
		B_i^k = 1 - 2 \sigma a_0 + \tau a_2,  &D_i^k = \tau f(x_i, \, t_k),
	\end{array}
\]
где $\sigma = \dfrac{\tau}{h^2}$.

Можно просто двигаться вперёд по \ti{слоям}~--- множествам точек с постоянным временем; значения находятся последовательно.

\paragraph{Неявная схема для уравнения теплопроводности}

Неявная схема получается, если в \ref{eq:therm-A-B} выбрать вариант B. Сверху вниз (т.е. назад по времени) просчитать не получится, поскольку начальные данные даются в начале, а не в конце.

\begin{figure}[h] \label{fig:therm-implicit}
	\begin{center}
		\includegraphics[scale=0.9]{../img/pde/therm-implicit.pdf}
	\end{center}
	\caption{Неявная схема для уравнения теплопроводности.}
\end{figure}

Формулы получатся такие:
\[
	A_i^k u_{i-1}^k - B_i^k u_i^k + C_i^k u_{i+1}^k = D_i^k,
\]
где
\[
	\begin{array}{ll}
		A_i^k = \sigma a_0 - \sigma a_1 \dfrac{h}{2}, & C_i^k = \sigma a_0 + \sigma \dfrac{h}{2} a_1, \\
		B_i^k = 1 + 2 \sigma a_0 - \tau a_2,  &D_i^k = -u_i^{k-1} - \tau f(x_i, \, t_k),
	\end{array}
\]
и $\sigma = \dfrac{\tau}{h^2}$.

По сути, движение всё ещё послойное. Но на каждом слое я не могу просто посчитать значение, используя три значения с предыдущего слоя: наоборот, получается уравнение, которое связывает три значения с текущего слоя с одним уже известным. В итоге получается система с трёхдиагональной матрицей, которая замыкается добавлением граничных условий:
\[
	u_0^k = \alpha_1(t_k), \; u_n^k = \alpha_2(t_k),
\]
если они заданы по первому типу, в противном случае применяются стандартные аппроксимации. 

Система решается методом разностной прогонки \ref{par:ode::fintdma}.

Метод прогонки срабатывает, поскольку
\[
	A_i^k + C_i^k = 2\sigma a_0 = B_i^k + \tau a_2 - 1,
\]
и можно сослаться на \ref{prop:ode::diffeqest::suff}. Наверное, на практике это правда так, потому что $\tau a_2 \ll 1$.



\paragraph{Явная схема для простейшего уравнения теплопроводности, решение разностных уравнений, неустойчивость}

Рассмотрим уравнение
\[
	\dfrac{\pd u}{\pd t} = \dfrac{\pd^2 u}{\pd t^2}, \quad t \in [0, \, \pi].
\]
с начальным условием $u(x, \, 0) = \varphi(x)$ и граничными условиями
\[
	u(0, \, t) = u(\pi, \, t) = 0
\]

Для него разностные уравнения исключительно просты:
\[
	\dfrac{u^{k + 1}_l - u^k_l}{\tau} = \dfrac{u^k_{l + 1} - 2u^k_l + u^k_{l-1}}{h^2}, \quad u_0^k = u_n^k = 0.
\]
При этом
\[
	h = \dfrac{\pi}{n}, \quad x_l = lh, \quad u_l^0 = \varphi(x_l).
\]

Решим наше разностное уравнение методом разделения переменных, будем искать решение в виде
\[
	u_l^k = \lambda^k e^{imx}, \quad x = x_l = lh.
\]
Подставим:
\[
	\dfrac{\lambda^{k + 1}e^{imx} - \lambda^{k}e^{imx}}{\tau} = \dfrac{\lambda^k e^{im(x+h)} - 2\lambda^k e^{imx} + \lambda^k e^{im(x-h)}}{h^2}.
\]
Несложными выкладками отсюда находится
\[
	\lambda = 1 + 2\sigma \big(\cos mh - 1\big)\,, \quad \sigma = \dfrac{\tau}{h^2}.
\]
Но такое решение не удовлетворяет граничным условиям; можно рассмотреть какую-нибудь комбинацию решений! Заметим, что $\lambda(m)$~--- чётная функция, поэтому 
\[
	\lambda^k(m) \big(e^{imx} - e^{-imx}\big) = 2i\lambda^k(m)\sin(mx)
\]
тоже решение. Оно удовлетворяет граничным условиям при целых $m$; в итоге получаем
\[
	\boxed{u_l^k = \lambda^k(m) \cdot \sin (mx), \quad m \in \Z}\,.
\]
У нас теперь есть $n-1$ ЛНЗ решение, из которых можно собирать новые:
\[
	\varphi(x) = \sum\limits_{m = 1}^{n - 1} C_m \lambda^k(m) \cdot \sin (mx).
\]

\begin{rem}
	Остальные значения $m$ нам не интересны, поскольку у нас набралась $n-1$ базисная функция: действительно, изначально наши разностные уравнения решались однозначно, а сейчас мы их решили, учитывая граничные условия, но отпустив начальные. А их как раз $n - 1$~--- от $u^0_1$ до $u^0_{n-1}$, они и создают все степени свободы.
\end{rem}

Рассмотрим $\tau = h^2 \so \sigma = 1$:
\[
	\lambda(m) = -1 + 2 \cos mh.
\]
При $m = n - 1$ и густой сетке (большом $n$)
\[
	\cos \dfrac{(n - 1)\pi}{n} = \cos \left(\pi - \dfrac{\pi}{n}\right) \approx -1 \so \lambda(n - 1) \approx -3!
\]

При увеличении $k$ решение
\[
	(-3)^k \sin(n - 1)x
\]
очень быстро растёт по модулю и всё время меняет знак. Кажется, что-то пошло не так!

\begin{rem}
	Реальное решение такой задачи~--- быстро убывающая колебашка. Конечно, пространственный шаг взят большим: у начальных данных есть переменность на том же масштабе. Однако то, что при уменьшении шага по времени $k$ получает возможность становиться больше, уже вообще ни в какие ворота не лезет.
\end{rem}

%unsure
%когда говорят про устойчивость, имеют в виду схему с фиксированным \tau или нет?
%unsure

Чтобы решения не вымирали подобным образом, можно наложить ограничение $|\lambda| \leqslant 1$:
\[
	1 + 2\sigma \big(\cos mh - 1\big) \leqslant 1 \so 2\sigma \cos mh - 1 \leqslant 0 \so 2 \sigma \cos mh \leqslant 1.
\]
Чтобы это выполнялось при любых $m$, нужно, чтобы
\[
	\boxed{\sigma \leqslant \dfrac{1}{2} \eqv \tau \leqslant \dfrac{h^2}{2}} \, .
\]

Если точно так же решить разделением переменных систему уравнений для простейшей неявной схемы, получим
\[
	\lambda = \dfrac{1}{1 + 2\sigma(1 - \cos mh)} \leqslant 1,
\]
и устойчивость всегда присутствует.

\paragraph{Общее определение устойчивости, теорема об устойчивости и сходимости}

В начале книги \cite{gavurin} есть общие рассуждения про вычислительные методы и разные пространства. В книге \cite{comp-krilov-2} есть про устойчивость, аппроксимацию, сходимость, их связь между собой, и про схемы для уравнения теплопроводности.

С какой ситуацией мы сталкиваемся, занимаясь сеточными методами? У нас есть оператор $A\col \; U \to F$, и мы решаем уравнение вида
\[
	Au = f.
\]
Выбирая сетку с шагом $h$ на отрезке, мы вместо функций на отрезке начинаем рассматривать функции на самой сетке~ они образуют другое, гораздо более маленькое пространство $U_h$. При этом по любому элементу $U$ можно легко найти элемент $U_h$, просто вычислив его значения на сетке. Аналогично строится пространство $F_h$\footnote{Зачастую $U_h = F_h$ и даже $U = F$, но может быть и не так, в принципе~--- вдруг, например, оператор действует в пространство функций на другом отрезке, или просто там другие ограничения на гладкость/непрерывность.}.

Наконец, есть оператор $A_h \col \; U_h \to F_h$~--- приближение $A$, которое получается при переходе к конечным разностям. Для иллюстрации полезна диаграмма
\[
	\xymatrix{
		U \ar@{->}[r]^{A} \ar@{->}[d]_{\varphi_h} & F \ar@{->}[d]^{\psi_h} \\ 
		U_h \ar@{->}[r]^{A_h} & F_h
	}
\]
\begin{de}
	Операторы $\varphi_h(u)(x_l) = u(x_l)$ и такой же $\psi_h$ называются \ti{операторами (простого) сноса}.
\end{de}

\begin{rem}
	Понятно, что диаграмма должна быть почти коммутативна, но не совсем: если мы сначала продифференцируем функцию, а потом возьмём результат на сетке, и если мы сначала возьмём её на сетке, а потом посчитаем разностный аналог производной, получатся близкие, но разные вещи. Разность
	\[
		A_h\varphi_h(u) - \psi_h(Au)
	\]
	называется \ti{естественной погрешностью метода}.
\end{rem}

Далее, записывается разностное уравнение
\[
	A_h \tilde{u} = \psi_h(f)
\]
и решается.

Во всех четырёх пространствах надо ввести нормы. В пространствах функциональной природы $U$, $F$ они уже и так есть, вероятно.

\begin{de}
	Говорят, что норма на $U_h$ \ti{согласована} с нормой на $U$, если верно, что 
	\[
		\|\varphi_h u\|_{U_h} \to \|u\|_U,
	\]
	когда $h \to 0$ хотя бы для $u \in K \subset U$, где $K$ плотно в $U$.

\end{de}

Будем считать, что у нас нормы согласованы.

\begin{de}
	Говорят, что $A_h$ \ti{аппроксимирует} $A$ на $u \in U$, если 
	\[
		\big\|A_h\varphi_h(u) - \psi_h(Au)\big\| \to 0 \text{ при } h \to 0.
	\]
\end{de}

\begin{de}
	Говорят, что сеточные функции $u_h$ сходятся к функции $u \in U$, если 
	\[
		\big\|u_h - \varphi_h(u)\big\| \to 0 \text{ при } h \to 0.
	\]
\end{de}

\begin{de}
	Говорят, что \ti{сеточное приближение} обладает \ti{свойством аппроксимации}, если $A_h$ аппроксимирует $A$, и сеточные функции $f_h$ сходятся к $f$.
\end{de}

\begin{de}  
	Говорят, что сеточное приближение \ti{устойчиво}, если
	\begin{enumerate}
		\item Уравнение $A_hu_h = f_h$ однозначно разрешимо для всех $f_h \in F_h$;
		\item Для этого решения $\|u_h\| \leqslant k \|f_h\|$, где $k$ не зависит от $h$.
	\end{enumerate}
\end{de}

\begin{thm}[Основная теорема теории разностных методов]
	Пусть дана некоторая краевая задача, и сеточная аппроксимация удовлетворяет следующим свойствам:
	\begin{enumerate}
		\item $u^*$~--- единственное решение уравнения $Lu^* = f$.
		\item Сеточное приближение обладает свойством аппроксимации.
		\item Сеточная задача устойчива.
	\end{enumerate}
	Тогда есть сходимость сеточных решений: $u^*_h \to u$.
	\begin{proof}
		Запишем ошибку сеточного решения:
		\[
			w_h = u_h^* - \varphi_h u^*.
		\]
		Заметим, что по свойству устойчивости
		\begin{align*}
			\|w_h\| &\leqslant k \|L_h w_h\| = k\|L_h u_h^* - L_h \varphi_h u^*\| = \\ &= k\|f_h - \psi_h f + \psi_h f - L_h \varphi_h u^*\| \leqslant k\|f_h - \psi_h f\| + k\|\psi_h L u^* - L_h \varphi_h u^*\|.
		\end{align*}
		Оба слагаемых в правой части стремятся к нулю по свойству аппроксимации.
	\end{proof}
\end{thm}

\paragraph{Разностные схемы для задач с начальными условиями, дискретное преобразование Фурье}

\begin{rem}
	Мне кажется, тут не очень понятно вышло, надо бы потом переписать. 
\end{rem}

В этом параграфе в целом посмотрим на уравнение 
\[
	\dfrac{\pd u}{\pd t} = Lu + f,
\]
где всё многомерное (т.е. $u$~--- вектор, а $L$~--- <<матрица>> из частных производных), и $L$~--- линейный дифференциальный оператор с постоянными коэффициентами. Область определения $u$~--- цилиндр $D\times [0, \, T] \subset \R^{p + 1}$. Начальные условия~--- $u(x, \, 0) = \varphi(x)$.

Ограничимся теперь ситуацией, когда $D$~--- куб $[0, \, 2\pi]^p$, а граничные условия периодические по каждой из переменных (т.е. $u(x_1, \ldots, \, 0, \ldots, \, x_p; \, t) = u(x_1, \ldots, \, 2\pi, \ldots, \, x_p; \, t)$). 

По каждой из пространственных переменных выберем одинаковые шаги
\[
	h = \dfrac{2\pi}{N},
\] 
а по временной~--- шаг
\[
	\tau = \dfrac{T}{M}.
\]
По пространственной причём рассматриваем только от $0$ до $M-1$, потому что справа снова будет то же граничное значение. Уравнения будут двухслойными, с $k$-го и $k+1$-го слоя.

Даже в неявном случае с помощью разностной прогонки можно выразить все следующие слои через предыдущие и получить уравнения
\[
	u_h(k + 1) = R_h u_h(k) + \rho_h(k),
\]
где $R_h$~--- \ti{оператор перехода в однородном случае}. $R_h$~--- просто матрица с постоянными коэффициентами, а $\rho_h(k)$ зависит от $f$.

\begin{rem}
	Всё-таки скажу про эти обозначения пространств... $V_h$~--- пространство сеточных функций \ti{на фиксированном слое} (т.е. оно $N$-мерное), а $F_h$~--- видимо, аналогичное пространство, которое мы отличаем только по формальным причинам, в котором лежат $f_h$~--- сеточные версии $f$. Нормы в обоих пространствах~--- просто $l^{\infty}$, т.е.
	\[
		\big\|\{u_i\}\big\| = \max |u_i|.
	\]
\end{rem}

\begin{thm} \label{thm:stab-1}
	Для устойчивости при $f = 0$ необходимо и достаточно, чтобы были ограничены $\|R_h^k\|$ (здесь $k$~--- степень!) при $k\tau \leqslant T$.
	\begin{proof}
		Оно в целом понятно: когда нет $f$-ок, нет и $\rho$-шек, а без них переход на следующий слой~--- тупо умножение на матрицу $R_h$. Ясно, что если нормы этих матриц в совокупности ограничены, то и
		\[
			\|u_h(k+1)\| \leqslant C \|\varphi\|,
		\]
		где $\varphi$ задаёт начальные условия.

		Обратно тоже понятно: если ограниченности норм матриц нет, можно просто пойти от противного и сконструировать мерзкую последовательность.
	\end{proof}
\end{thm}

\begin{thm}
	Если $f \neq 0$ и $\|R_h^k\|$ ограничены, то для устойчивости достаточно, чтобы
	\[
		\|\rho_h\|_{V_h} \leqslant c_2 \tau \|f_h\|_{F_h}
	\]
	\begin{proof}
		Тут несложная оценка, она есть в бумажном конспекте. 
	\end{proof}
\end{thm}

\begin{cor}
	Если $\|R_h\| \leqslant 1 + c_3\tau$, то есть устойчивость (при $f = 0$).
	\begin{proof}
		\[
			\|R_h^k\| \leqslant \|R_h\|^k \leqslant (1 + c_3 \tau)^k \leqslant e^{c_3 \tau k} \leqslant e^{c_3 \tau}.
		\]
	\end{proof}
\end{cor}

По поводу этих теорем можно ещё заглянуть в следующий параграф \ref{par:neumann}, там доказаны очень похожие вещи.

Перейдём теперь к дискретному преобразованию Фурье. Пусть размерность $p$ пока равна $1$. Введём на пространстве $V_h$ функций на фиксированном слое скалярное произведение:
\[
	(u_h, \, v_h) = h \sum\limits_{i = 0}^{N - 1} u_h \ov{-}{v_h}
\]

\begin{st}
	Набор функций $e_m(x) = e^{imx}$, где $m \in 0\ldots N-1$, образует ортогональный базис в $V_h$, причём
	\[
		(e_m, \, e_m) = 2\pi.
	\]
	\begin{proof}
		Чтобы увидеть, что они ортогональны, достаточно посчитать скалярное произведение. Отсюда следует, в принципе, что они ЛНЗ. Ну а дальше~--- их $N$, пространство $N$-мерное, потому и базис.
	\end{proof}
\end{st}

\begin{de}
	\ti{Обратное дискретное преобразование Фурье}~---
	\[
		\{a_1, \, \ldots, \, a_N\} \mapsto \sum\limits_{i = 1}^{N-1} a_i e_i(x).
	\]
	\ti{Прямое ДПФ}~---
	\[
		u_h \mapsto \dfrac{1}{2\pi}\big\{(u_h, \, e_1), \ldots, \, (u_h, \, e_n)\big\}.
	\]
	Ясно, что это взаимно обратные операторы.
\end{de}

\begin{st}[Формула замкнутости]
	Дискретное преобразование Фурье~--- почти унитарный оператор, т.е. 
	\[
		\left(\sum\limits_{i = 1}^{N - 1} a_i e_i, \; \sum\limits_{i = 1}^{N - 1} b_i e_i\right) = 2 \pi \sum_{i = 0}^{N - 1} a_i \ov{-}{b_i}.
	\]
	\begin{proof}
		Проверяется прямым вычислением.
	\end{proof}
\end{st}

\begin{st}
	$e^{imx}$~--- собственная функция оператора сдвига
	\[
		T_h u(x) = u(x + h)
	\]
	с собственным числом $e^{imh}$.
	\begin{proof}
		Действительно,
		\[
			T_h e^{imx} = e^{im(x + h)} = e^{imh} e^{imx}.
		\]
	\end{proof}
\end{st}

\begin{rem}
	У нас периодические граничные условия, поэтому оператор сдвига может действовать, <<переходя>> через границу:
	\[
		T_h \{u_0, \, \ldots, \, u_{N - 1}\} = \{u_{1}, \, u_2, \, \ldots, \, u_{N-1}, \, u_0\}.
	\]
	Можно представлять себе, что индекс $i$ на самом деле меняется от $-\infty$ до $\infty$, но $u_{i + N} = u_i$. Периодическую функцию можно восстановить, зная её значения внутри периода, вот и здесь так же.
\end{rem}

\begin{rem}
	В такой ситуации любой разумный разностный оператор можно собрать из операторов сдвига. Например, пусть
	\[
		(Du)_i = \dfrac{u_{i + 1} - 2u_i + u_{i - 1}}{h}.
	\]
	Это можно переписать просто как
	\[
		Du = \dfrac{T_hu - 2u + T_{-h}u}{h}.
	\]
	Общая формула, естественно, будет такая:
	\[
		Lu = \sum\limits_{\alpha} c(\alpha) T_h^{\alpha}(u),
	\]
	где $\alpha \in \Z$~--- показатель степени.

	Тем удобнее будет применять эти операторы к экспонентам~--- от них ведь сдвиг считать легко.
\end{rem}

Это всё можно написать и в многомерии, при $p>1$, но, кажется, У Оли этого нет. См. конспект Ангелины, билет и так длинный.

Применим теперь построенную теорию к сеткам. Сеточное уравнение будет выглядеть примерно так:
\[
	\sum\limits_{\alpha \in A_0} A(\alpha) u(x + \alpha h, \, k \tau) = \sum\limits_{\beta \in B_0} B(\beta) u\big(x + \beta h, \, (k+1) \tau\big).
\]
Ну, это просто два слоя, $k$-й и $k+1$-й. Теперь сделаем ДПФ, пусть
\[
	u(x, \, k\tau) = \sum a^k(m) e^{imx}.
\]
Подставим это в суммы:
\[
	\sum\limits_{\alpha \in A_0} e^{im\alpha h} A(\alpha) \sum\limits_m a^k(m) e^{imx} = \sum\limits_{\beta \in B_0} e^{im\beta h} B(\beta) \sum\limits_m a^{k+1}(m) e^{imx}.
\]
Слева и справа написаны два разложения по базису, коэффициенты в которых должны совпадать:
\[
	a^k(m) \sum\limits_{\alpha \in A_0} e^{im\alpha h} A(\alpha)  = a^{k+1}(m) \sum\limits_{\beta \in B_0} e^{im\beta h} B(\beta) 
\]
В итоге получаем
\[
	a^{k + 1}(m) = c(m) a^k(m), \quad c(m) = \dfrac{\sum\limits_{\alpha \in A_0} e^{im\alpha h} A(\alpha)}{\sum\limits_{\beta \in B_0} e^{im\beta h} B(\beta)}.
\]

\paragraph{Необходимое условие устойчивости по фон Нейману}\label{par:neumann}

\begin{rem} 
	%unsure
	Коэффициент $c$~--- по сути, видимо, диагональная матрица. Это матрица перехода между слоями в терминах коэффициентов Фурье. 

	Я не совсем понял, где в конспекте проходит грань между матрицей и числом (всё усложняется тем, что в высших размерностях $m$~--- мультииндекс, а $u^k$ и $c$, видимо, <<тензоры>>). Поэтому я буду исходить из того, что $c(m)$~--- просто число, а $c$~--- набор этих чисел, причём
	\[  
		\|c\| = \|c\|_{l^{\infty}} = \max\limits_m c(m).
	\]
	Ну и вообще, пусть все доказательства будут одномерными.
	%unsure
\end{rem}

\begin{thm}
	Для устойчивости при $f = 0$ необходимо и достаточно, чтобы
	\[
		\|c^k\| \leqslant c_3, \quad k\tau \leqslant T.
	\]
	\begin{proof}
		Интересно, видимо, доказывать достаточность. Попробуем просто найти оценку на норму $u_h(k)$.
		\[
			a(k, \, m) = c^k(m)a(0, \, m),
		\]
		поэтому 
		\[
			u_h(k) = \sum\limits_m a(k, \, m) e^{imx} = \sum\limits_m a(0, \, m) c^k(m) e^{imx}.
		\]
		Далее $K$~--- произвольная неотрицательная константа, в которую можно вносить другие.
		По формуле замкнутости
		\begin{align*}
			\big\|u_h(k)\big\|_{l^2}^2 &= 2\pi \sum\limits_m \big|a(0, \, m) c^k(m)\big|^2 \leqslant \\ & \leqslant K \max\limits_{m} \big|a(0, \, m)\big|^2 \big|c^k(m)\big|^2 \leqslant  K \max\limits_m |a(0, \, m)\big|^2.
		\end{align*}
		Последний переход возможен, поскольку $\|c^k\| \leqslant c_3$.
		При этом
		\begin{align*}
			\max\limits_m |a(0, \, m)\big|^2 &= \left(\max\limits_m |a(0, \, m)\big|\right)^2 = \|a^0\|_{l^{\infty}}^2 \leqslant \\ &\leqslant K\|a^0\|_{l^2}^2 = K \big\|u_h(0)\big\|_{l^2}^2 \leqslant K\big\|u_h(0)\big\|_{l^{\infty}}^2.
		\end{align*}
		В итоге получаем
		\[
			\big\|u_h(k)\big\|_{l^{\infty}}^2 \leqslant K\big\|u_h(k)\big\|_{l^2}^2 \leqslant K\big\|u_h(0)\big\|_{l^{\infty}}^2.
		\]
		Это и есть устойчвость, по сути.
		Чтобы доказать необходимость, предположим, что нет такой оценки $\|c^k\| \leqslant c_3$, не зависящей от $h$ и $\tau$. Рассмотрим $u_h(0) = e^{imx}$. Тогда
		\[
			a(0, \, l) = \delta_{ml} \so u_h(k) = c^k(m)e^{imx}.
		\]
		По предположению мы можем так подобрать $h, \, \tau, \, m, \, k$, что $\big|c^k(m)\big|$ станет сколь угодно большим; но тогда это произойдёт и с $\|u_h(k)\|$! Понятно, что никакой устойчивости нет и в помине.
	\end{proof}
\end{thm}

\begin{thm}[условие фон Неймана]
	Для устойчивости при $f = 0$ необходимо и достаточно, чтобы собственные числа $c$ удовлетворяли условию
	\[
		|\lambda| \leqslant 1 + c_4 \tau.
	\]
	\begin{proof}
		Ну, достаточность не слишком сложна:
		\[
			\|c^k\| = \max \limits_m \big|c(m)\big|^k \leqslant \big|1 + c_4 \tau\big|^k \leq e^{kc_4\tau} \leqslant e^{c_4 T}.
		\]
		Необходимость, впрочем, тоже. Пусть этого условия нет; тогда для любого $c_4$ можно подобрать такие $h$, $\tau$ и $m$, что $c(m) > 1 + c_4\tau$. Но тогда 
		\[
			\|c^k\| \geqslant \big|c(m)^k\big| \leqslant (1 + c_4 \tau)^k.
		\]
		Ясно, что увеличивая $c_4$, можно неограниченно увеличивать $\|c^k\|$.
	\end{proof}
\end{thm}

\begin{rem} 
	Кажется, в многомерии это условие только необходимое.
\end{rem}

\paragraph{Простейшие схемы для уравнения бегущей волны}

\begin{de}
	\ti{Уравнение бегущей волны} имеет вид
	\[
		\dfrac{\pd u}{\pd t} = a \dfrac{\pd u}{\pd x}.
	\]
\end{de}

\begin{rem}
	Любая функция вида
	\[
		u(x, \, t) = f(x + at)
	\]
	является решением.
\end{rem}

\begin{rem}
	Видимо, мы будем работать с периодическим граничными условиями I типа, как и с простейшим уравнением теплопроводности.
\end{rem}

Запишем разностные уравнения:
\[
	\dfrac{u^{k + 1}_l - u^k_l}{\tau} = a \dfrac{u^k_{l+1} - u^k_l}{h}.
\]
Это обычная, явная схема. Чтобы исследовать устойчивость, найдём матрицу перехода. Для этого подставим $u_l^k = e^{imx}$ и $u_l^{k + 1} = c(m)e^{imx}$:
\[
	\dfrac{c(m)e^{imx} - e^{imx}}{\tau} = a \dfrac{e^{imx}e^{imh} - e^{imx}}{h}.
\]
Отсюда быстро находим
\begin{equation}\label{eq:wave-trans}
	\boxed{c(m) = 1 + a\sigma(e^{imh} - 1), \quad \sigma = \dfrac{\tau}{h}}\,.
\end{equation}

Нужно понять, когда $\big|c(m)\big| \leqslant 1$.

%unsure
% \begin{rem}
% 	Этого будет достаточно, но мне не очень понятно, почему не проверяется более точное условие с $1 + c\tau$... Впрочем, в данном конкретном случае это, видимо, то же самое.
% \end{rem}
%unsure

\begin{st}
	$\hphantom{.}$
	\begin{enumerate}
		\item Если $a < 0$, то $|c| > 1$.
		\item Если $a > 0$ и $|a\sigma| \leqslant 1$, то $|c| < 1$.
	\end{enumerate}
	\begin{proof}
		Очень советую нарисовать все картинки, иначе непонятно будет. Если кратко, то
		\begin{enumerate}
			\item $e^{imh}$ пробегает единичную окружность.
			\item $e^{imh} - 1$ пробегает единичную окружность с центром в $-1$ (правой стороной она касается нуля).
			\item Умножение на $a\sigma$ либо просто растягивает (относительно нуля), либо растягивает и переворачивает.
			\item Когда $a < 0$, переворачивает, и получается окружность справа от нуля. После прибавления $1$~--- окружность справа от единицы.
			\item Если $a > 0$, получается окружность слева от единицы, которая через неё проходит, радиуса $a\sigma$. Логично, что этот радиус можно увеличить до $1$~--- тогда центр будет в нуле. А дальше нельзя.
		\end{enumerate}
	\end{proof}
\end{st}

\begin{figure}[h]
	\begin{center}
		\includegraphics[scale=0.8]{../img/pde/wave-dep.pdf}
	\end{center}
	\caption{К замечанию \ref{rem:wave-dep}.}
\end{figure}


\begin{rem}\label{rem:wave-dep}
	Заметим, что $u$ постоянна на прямой $x + at = const$. Будем пока считать $a > 0$. Рассмотрим значение $u_l^k$. Оно должно определятся соответствущим значением на прямой $t = 0$:
	\[
		lh + ak\tau = sh \so s = l + a\sigma k.
	\]
	С другой стороны, в нашей схеме значение $u_l^k$ определяется $u_l^{k-1}$ и $u_{l+1}^{k-1}$. Если продолжить этот процесс до $k = 0$, увидим, что $u_l^k$ зависит лишь от $u^0_l\ldots u^0_{l + k}$. Таким образом, чтобы вообще использовать нужное значение из начальных данных, надо
	\[
		s \leqslant l + k \so l + a \sigma k \leqslant l + k \so a \sigma \leqslant 1.
	\]
	Если $a < 0$, то мы вообще не будем использовать это значение, ибо красная прямая будет наклонена в другую сторону.
\end{rem}

Именно с этим связано то, что для $a < 0$ срабатывает схема
\[
	\dfrac{u^{k + 1}_l - u^k_l}{\tau} = a \dfrac{u^k_{д} - u^k_{l-1}}{h}.
\]
В ней информация распространяется в другую сторону, и оценки получаются те же с точностью до знака $a$.

\paragraph{Схема Куранта-Рисса}

Рассмотрим теперь систему
\[
	\dfrac{\pd u}{\pd t} = A \dfrac{\pd u}{\pd x},
\]
где $u$~--- вектор из $\R^p$. Будем считать, что $A$~--- симметричная матрица с постоянными коэффициентами (поэтому у неё все собственные числа вещественны).

Собственные числа матрицы $A$ могут быть разных знаков, это приводит к появлению решений-волн, которые бегут в разные стороны. Это причина, по которой ни одна из простейших схем, вероятно, не будет работать.

Рассмотрим две матрицы: $A_+$ и $A_-$. Первая из них имеет положительные собственные числа такие же, как у $A$, а вместо отрицательных у неё нули. $A_-$ вместо отрицательных собственных чисел $A$ имеет их модули, а вместо отрицательных~--- нули. 

\begin{rem}
	Можно эти две матрицы построить в собственном базисе $A$, а потом вернуть их оттуда назад.
\end{rem}

В итоге $A = A_+ - A_-$. Схема будет устроена так:
\[
	\dfrac{u_l^{k+1} - u_l^k}{\tau} = A_+ \dfrac{u^k_{l + 1} - u^k_l}{h}  - A_- \dfrac{u_l^k - u^k_{l-1}}{h}.
\]
Естественно ожидать от неё хорошего поведения. Такая схема называется \ti{схемой Куранта-Рисса}. Найдём матрицу перехода; для этого вычислим её на
\[
	u_l^k = fe^{imx}, \quad f \text{~--- произвольный постоянный вектор.}
\]
Такой же подстановкой, как в прошлом пункте, находим
\[
	c(m) = I + \sigma \left((e^{imh} - 1)A_+ - (1 - e^{-imh}) A_-\right), \quad \sigma = \dfrac{\tau}{h}.
\]
Теперь нам надо искать собственные числа <<матрицы>> $C$ (мы тут всё-таки вляпались в многомерие, но не сильно). Запишем условие на собственные числа:
\[
	g + \sigma \left((e^{imh} - 1)A_+ - (1 - e^{-imh})A_- \right)g = \lambda g.
\]
Отсюда
\[
	\left((e^{imh} - 1)A_+ - (1 - e^{-imh})A_- \right)g = \dfrac{\lambda - 1}{\sigma} g
\]
Слева стоит линейная комбинация матриц с известными собственными числами, причём там, где у первой ненулевое собственное число, у второй ноль, и наоборот. Поэтому получаем
\[
	\lambda = 1 + \sigma \lambda_{A_{\pm}} (e^{\pm imh} - 1).
\]
Очень похоже на \ref{eq:wave-trans}, но только теперь $a = \lambda_{A_{\pm}}$ и $m = \pm m$.

Ясно, что знак $m$ ни на что не повлияет, и теперь $a > 0$. Второе условие получится аналогичным.

\begin{rem}
	Мы не доказывали достаточность условия для многомерности, но в этом конкретном случае её можно доказать.
\end{rem}

\paragraph{Явная схема для уравнения колебаний струны}

Рассмотрим уравнение колебаний
\[
	\dfrac{\pd^2 u}{\pd t^2} = \dfrac{\pd^2 u}{\pd x^2}.
\]

Вообще, надо было бы перейти к системе уравнений первого порядка, но мы попробуем найти $c(m)$ не переходя~--- вдруг получится!..

Схема
\[
	\dfrac{u_l^{k + 1} - 2u_l^k + u_l^{k-1}}{\tau^2} = \dfrac{u^k_{l+1} - 2u^k_l + u^k_{l-1}}{h^2}.
\]
Положим
\[
	u_l^k = e^{imx}, \quad u_l^{k+1} = ce^{imx}, \quad u_l^{k-1} = c^{-1} e^{imx}.
\]
Выйдет уравнение на $c$:
\[
	\dfrac{c + c^{-1}}{2} = 1 + \sigma^2\big(\cos(mh) - 1\big) = p.
\]
Решая, получим
\[
	c = p \pm \sqrt{p^2 - 1}.
\]
Неудивительно, что нашлись два значения для каждого $m$: на самом деле это собственные числа матрицы перехода, $\lambda_1$ и $\lambda_2$, потому что система второго порядка!

Заметим, что по теореме Виета (ну или руками) 
\[
	\lambda_1 \lambda_2 = 1,
\]
поэтому, чтобы была устойчивость, нам нужно, чтобы $|\lambda_1| = |\lambda_2| = 1$. Это условие выполняется, когда корни комплексны:
\[
	\lambda_{1, \, 2} = p \pm i\sqrt{1 - p^2} \so |\lambda_{1, \, 2}| = 1.
\]
Для этого необходимо, чтобы $|p| < 1$. Ещё оно выполняется, когда корни равны. Чтобы они были равны, нужно $p = \pm 1$, поэтому общее условие:
\[
	|p| \leqslant 1.
\]

Посмотрев на формулу для $p$, поймём, что это гарантированно происходит при 
\[
	\boxed{|\sigma| \leqslant 1}\,.
\]

\begin{rem}
	Можно ещё сказать, что при $\sigma = 1$ на самом деле проявляется слабая неустойчивость, но она не влияет на сходимость.
\end{rem}

\paragraph{Явная и неявная схемы для двумерного уравнения теплопроводности}

\begin{de}
	\ti{Двумерное уравнение теплопроводности}
	\[
		\dfrac{\pd u}{\pd t} = \dfrac{\pd^2 u}{\pd x^2} + \dfrac{\pd^2 u}{\pd y^2}.
	\]
\end{de}

Теперь нам понадобится два индекса внизу:
\[
	u^k_{np} \approx u(nh, \, ph, \, k\tau).
\]

Рассмотрим явную схему
\begin{equation}\label{eq:therm-2-expl}
	\dfrac{u^{k+1}_{np} - u^k_{np}}{\tau} = \dfrac{u^k_{n+1 \; p} - 2u^k_{np} + u^k_{n-1 \; p}}{h^2} + \dfrac{u^k_{n \; p+1} - 2u^k_{np} + u^k_{n \; p - 1}}{h^2}.
\end{equation}

Попробуем найти матрицу перехода. Для этого рассмотрим 
\[
	u^k_{np} = e^{imx} e^{ily}, \quad x = nh, \quad y = ph.
\]
Делая стандартную подстановку и преобразования, получаем
\[
	c(m, \, l) = 1 + 2\sigma\big(cos(mh) - 1\big) + 2\sigma\big(\cos(lh) - 1\big), \quad \sigma = \dfrac{\tau}{h^2}.
\]
Чтобы $\big|c(m, \, l)\big| \leqslant 1$ всегда, нужно
\[
	\boxed{\sigma \leqslant \dfrac{1}{4}}\,.
\]
Получилось в два раза жёсткое условие, чем для одномерного уравнения!

Посмотрим теперь на простейшую неявную схему, в ней левая часть уравнения \ref{eq:therm-2-expl} просто заменится на
\[
	\dfrac{u^{k}_{np} - u^{k-1}_{np}}{\tau}.
\]
Тем же путём находим
\[ 
	c(m, \, l) = \dfrac{1}{1 - 2\sigma\big(cos(mh) - 1\big) - 2\sigma\big(\cos(lh) - 1\big)}.
\]
Видно, что необходимое условие устойчивости выполнено всегда, как и в одномерном случае.

Поговорим о том, как решать систему уравнений для неявной схемы, которая целиком выглядит вот так:
\[
	\dfrac{u^{k}_{np} - u^{k-1}_{np}}{\tau} = \dfrac{u^k_{n+1 \; p} - 2u^k_{np} + u^k_{n-1 \; p}}{h^2} + \dfrac{u^k_{n \; p+1} - 2u^k_{np} + u^k_{n \; p - 1}}{h^2}.
\]
Если её переписать, выйдет
\[
	(1 + 4\sigma) v_{np} - \sigma v_{n + 1\; p} - \sigma v_{n \; p+1} - \sigma v_{n - 1\; p} - \sigma v_{n \; p-1} = \alpha_{np},
\]
где
\[
	v_{np} = u_{np}^k \text{ и } \alpha_{np} = u^{k-1}_{np}.
\]
Граничные условия, как обычно в последнее время, I типа:
\[
	v_{00} = v_{0N} = v_{N0} = v_{NN} = 0.
\]
Поэтому будем рассматривать $n$ и $p$ от $1$ до $N-1$. 

Упорядочим элементы $v$ следующим образом:
\[
	v_{11}, \; v_{12}, \; \ldots, \; v_{1 \; N-1}, \; v_{21}, \; \ldots
\]
Тогда матрица системы (для примера $N=4$) будет выглядеть так:
\[
	\begin{array}{ccc|ccc|ccc}
		1 + 4\sigma & -\sigma & 0 & -\sigma & 0 & 0 & 0 & 0 & 0 \\
		-\sigma & 1 + 4\sigma & -\sigma & 0 & -\sigma & 0 & 0 & 0 & 0 \\
		0 & -\sigma & 1 + 4\sigma & 0 & 0 & -\sigma & 0 & 0 & 0 \\
		\hline
		-\sigma & 0 & 0 & 1 + 4\sigma & -\sigma & 0 & -\sigma & 0 & 0 \\
		0 & -\sigma & 0 & -\sigma & 1 + 4\sigma & -\sigma & 0 & -\sigma & 0 \\
		0 & 0 & -\sigma & 0 & -\sigma & 1 + 4\sigma & 0 & 0 & -\sigma \\
		\hline
		0 & 0 & 0 & -\sigma & 0 & 0 & 1 + 4\sigma & -\sigma & 0 \\
		0 & 0 & 0 & 0 & -\sigma & 0 &  -\sigma & 1 + 4\sigma & -\sigma \\
		0 & 0 & 0 & 0 & 0 & -\sigma & 0 & -\sigma & 1 + 4\sigma
	\end{array}
\]
В общем случае получается $(2N-1)$-диагональная матрица.

Пусть
\[
	v_n = (v_{n1}, \, \ldots, \, v_{n \; N-1}), \quad \alpha_n = (\alpha_{n1}, \, \ldots, \, \alpha_{n \; N-1}).
\]
Тогда можно записать систему, как
\[
	A_n v_{n-1} + B_n v_n + C_n v_{n+1} = \alpha_{n},
\]
где $A_n, \; B_n$ и $C_n$~--- блоки из соответствующей строки.
Дальше можно действовать так же, как обычной прогонкой. Это называется \ti{матричная прогонка}.

К сожалению, обычная прогонка содержит умножения чисел, которые делаются за $O(1)$, и работает $O(N)$, а матричная прогонка содержит умножения матриц, которые делаются за $O(N^3)$, и работает за $O(N^4)$. Долго! Поэтому нам такой неявный метод не подходит.

\paragraph{Схема продольно-поперечной прогонки}

Нужно сделать систему трёхдиагональной. Естественное желание~--- вынести часть переменных в правой части уравнения на соседний слой, чтобы сократить количество тех, что входит с текущего. Эта идея приводит к схеме

\[
	\dfrac{u^{k+1}_{np} - u^{k}_{np}}{\tau} = \dfrac{u^{k+1}_{n+1 \; p} - 2u^{k+1}_{np} + u^{k+1}_{n-1 \; p}}{h^2} + \dfrac{u^k_{n \; p+1} - 2u^k_{np} + u^k_{n \; p - 1}}{h^2}.
\]
Если переписать, получится
\[
	\sigma u^{k+1}_{n-1 \; p} - (1 + 2\sigma) u_{np}^{k+1} + \sigma u^{k+1}_{n+1 \; p} = \alpha_{np}.
\]
Матрица трёхдиагональная, всё замечательно.

Надо проверить на устойчивость. Обычной техникой получаем
\[
	c(l, \, m) = \dfrac{1 - 2\sigma(1 - \cos lh)}{1 + 2\sigma(1 - \cos mh)}.
\]
Чтобы эта штука всегдя была меньше $1$ по модулю, нужно
\[
	\boxed{\sigma \leqslant \dfrac{1}{2}}\,.
\]
Только условная устойчивость!

Можно рассмотреть аналогичную схему
\[
	\dfrac{u^{k+1}_{np} - u^{k}_{np}}{\tau} = \dfrac{u^{k}_{n+1 \; p} - 2u^{k}_{np} + u^{k}_{n-1 \; p}}{h^2} + \dfrac{u^{k+1}_{n \; p+1} - 2u^{k+1}_{np} + u^{k+1}_{n \; p - 1}}{h^2}.
\]
У неё свойства примерно такие же, только $l$ и $m$ меняются местами.

Идея: чередовать схемы I и II:
\[
	2k \xrightarrow{\text{I}} 2k+1 \quad 2k+1 \xrightarrow{\text{II}} 2k+2.
\]
Пусть у чётных слоёв будет номер $k$, у нечётных~--- $k + \nicefrac{1}{2}$. Ну и просто пишем последовательно формулы для I сначала, потом для II, и получаем переход
\[
	k \to k+ \frac{1}{2} \to k+1.
\]

Ясно, что при этом коэффициенты перехода перемножатся:
\[
	C = C_{\text{I}} C_{\text{II}} = \dfrac{1 - 2\sigma(1 - \cos lh)}{1 + 2\sigma(1 - \cos mh)} \cdot \dfrac{1 - 2\sigma(1 - \cos mh)}{1 + 2\sigma(1 - \cos lh)}.
\]
У этой штуки получается $|c| \leqslant 1$ всегда, наступает абсолютная устойчивость.

Такую схему называют \ti{схемой продольно-поперечной прогонки.}

\begin{rem}
	К сожалению, если то же самое сделать для трёхмерного уравнения теплопроводности (а там будет три схемы и разбиение слоя на три подслоя), абсолютной устойчивости не выйдет. Это можно проверить прямым вычислением.
\end{rem}

\begin{rem}
	Более подробно все выкладки можно прочитать в бумажном конспекте, там всё понятно. Ещё там рассказывается про \ti{схему расщепления}, которая всегда работает. Потом есть рассуждения про то, как избавиться от наших общих ограничений вроде периодичности граничных условий, кубической формы области и постоянных коэффициентов.
\end{rem}



\paragraph{Задача Дирихле для двумерного эллиптического уравнения, составление разностных уравений}
\label{par:pde::elldirprobl}

\begin{defn}\label{defn:pde::elldirprobl::leq}
	Рассмотрим уравнение в частных производных 2 порядка
	\[
		Lu = f \that \sum_{i,j}a_{ij} \frac{∂u}{∂x_i\, ∂x_j} + \sum_{i} a_j \pder{u}{x_i} + a u = f
	\]
	Пусть все функции заданы на области $Ω \subset \R^n$.

	Если квадратичная форма, соответствующая $L$ знакоопределена, то 
	\begin{itemize}
		\item $L$ называют эллиптическим оператором.
		\item $Lu = f$ называют эллиптическим уравнением.
	\end{itemize}
\end{defn}

\begin{exmp}
	Уравнение Пуассона: $L = Δ$
	\[
		Δu = f \that \sum_{i}\pder[2]{u}{x_i} = f \qquad a_{ii} = 1 > 0 
	\]
\end{exmp}

Задачу Коши для такого уравнения не поставить~"--- нету выделенной переменной.
Так что будем решать граничные задачи. Пусть $Γ = ∂Ω$
\begin{enumerate}[I]
	\item $\left.u\right|_{Γ} = φ$~"---  задача Дирихле
	\item $\left.\pder{u}{n}\right|_{Γ} = ψ$~"---  задача Неймана
\end{enumerate}

\begin{prop}\label{prop:pde::elldirprobl::max}
	Для эллиптических уравений работает принцип максимума:
	\[
		\max_{Ω \mathbin{\cup} ∂Ω} u = \max_{∂Ω} u 
	\]
\end{prop}
\begin{rem}
	Для разностных уравений он работает лишь если нету перекрёстных членов.
\end{rem}
Рассмотрим пока $n=2$.
Будем решать эти задачи методом сеток
\[
	\begin{aligned}
		Ω_h &= \left\{ (nh, ph) \in \ov-Ω \right\} \\
		Ω_h^0 &= \left\{ (nh, ph)\land ((n\pm1)\,h, ph) \land (nh, (p\pm1)\,h) \in \ov-Ω \right\}\\
		Γ_h &=  Ω_h \setminus Ω_h^0
	\end{aligned}
\]

Как вычислять производные в $Ω_h^0$:
\begin{align*} 
	&\pder[2]{u}{x} &&\longrightarrow & & \begin{matrix}
			& 0  & 0 & 0 \\
		p & \lfrac{1}{h^2}  & \lfrac{-2}{h^2} & \lfrac{1}{h^2} \\
			& 0  & 0 & 0 \\
			&    & n &   \\
	\end{matrix} &&& 
	&\pder[2]{u}{y} &&\longrightarrow& & \begin{matrix}
			& 0  & \lfrac{1}{h^2} & 0 \\
		p & 0  & \lfrac{-2}{h^2} & 0 \\
			& 0  & \lfrac{1}{h^2} & 0 \\
			&    & n &   \\
	\end{matrix} \\[1em]
			&\frac{∂^2u}{∂x\,∂y} &&\longrightarrow& & \begin{matrix}
			& -\lfrac{h^2}{4}  & 0 & \lfrac{h^2}{4} \\
				p & 0  & 0 & 0 \\
					& \lfrac{h^2}{4} & 0 & -\lfrac{h^2}{4}\\
					&    & n &   \\
	\end{matrix}
\end{align*}
Таким образом, видно что шаблон схемы состоит из 9 точек, а если перекрёстных членов нету, то 
из пяти. Будем считать что их всё-таки нету. Можно же привести к 
сумме квадратов.

Запишем разностное уравнение как в первой главе
\begin{equation} \label{eq:pde::elldirprobl::diffeq}
	A_{np} u_{n+1,p} + B_{np} u_{n-1, p} + C_{np} u_{n, p} + D_{np} u_{n, p+1} + E_{np} u_{n,p-1}  
	= f_{np}
\end{equation}

Нетрудно выразить и его коэффициенты
\[
	\begin{aligned}
		A_{np} &= \frac{a_{11}}{h^2} + \frac{a_1}{2h} & 
		B_{np} &= \frac{a_{11}}{h^2} - \frac{a_1}{2h} \\ 
		C_{np} &= -\frac{2a_{11}}{h^2} - \frac{2a_{22}}{h^2}+ a \\
		D_{np} &= \frac{a_{22}}{h^2} + \frac{a_2}{2h} &
		E_{np} &= \frac{a_{22}}{h^2} - \frac{a_2}{2h} 
	\end{aligned}
\]

Поскольку $a_{11}, a_{22} > 0$ (положительно определённый) $C_{np}$ скорее всего $< 0$
Для сходимости разностных схем мы обычно требовали диагонального преобладания,
а тут это выливается в условие на $a$
\[\def\do#1{{#1}_{np} + }
  \abs{C_{np}} > \abs{\docsvlist{A,B,D} E_{np}} \Leftarrow a < 0
\]

Осталось понять что делать с $Γ_h$.
\begin{enumerate}
	\item Простой снос на границу: выбираем $M\that u_{np} = φ(M)$. Точность тут $O(h)$.
		А нам бы лучше извернуться и сделать $O(h^2)$.
	\item Снос с интерполяцией: найдем поточнее точку пересечения с границей.\par
		Например, между ${n-1}$ и $n$
		\[
      p_1(x) = \frac{x-h(n-1)}{h} \, u_{n,p} + \frac{hn - x}{h}\, u_{n-1,p}
		\]
		Пусть граница пересекается в $n + d$, тогда у нас получается уравнение на $d$
		\[
			\left(1+\frac{d}{h}\right) \, u_{n,p} + \frac{d}{h}\, u_{n-1,p} = φ(M), \quad
      M = h(n+d),hp
		\]
		Вот тут уже $O(h^2)$, первый порядок мы убираем выбором $d$.
\end{enumerate}

% \begin{aux}
%   Можно похожим на снос с интерполяцией метод посчитать вторую производную на граю.
%   По сути меняем шаг в сетке. Но мы этого
%   делать не будем. Надеюсь.
%
%   А вот как решать задачу Неймана мы не разбирали. Вопрос называется задача Дирихле, так
%   пусть кроме неё ничего и не будет
% \end{aux}

\paragraph{Итерационный метод решения сеточной системы}
\label{par:pde::iterell}

Посмотрим на систему \eqref{eq:pde::elldirprobl::diffeq}.
Здесь ровно та же проблема, что и в неявной схеме в уравнении
теплопроводности: прогоночные коэффициенты стали матрицами.  По слоям решать
не выйдет: граница какой угодно формы.\note{на задачу коши наверное можно
посмотреть как на граничное условие на дне цилиндра.} 
А делать $O(N^4)$ как-то не хочется.

\begin{aux}
  Говоря об этом уравнении как о линейной системе, мы имеем в 
  виду что собрали одномерный вектор из $u_{ik}$ просто расположив
  строки друг за другом. Как двумерные массивы в фортране.
\end{aux}

Будем использовать итеративные методы
\[
  Au = f \iff u = Bu + g \qquad u_{n+1} = Bu_n + g 
\]
\begin{prop}[Теорема о сжимающем отображении]
  Если $\norm{B} < 1$ итерации выше сходятся.
\end{prop}

\begin{rem}
  Это достаточное условие. Необходимым и достаточным будет 
  $\max \abs{λ_B} < 1$. Вообще, $\norm{B} = \max \abs{λ_B}$ для симметричных матриц.
\end{rem}
\begin{prop}
    $\displaystyle
      \norm \infty{B} = \max_i \sum_j \abs{b_{ij}} 
    $
\end{prop}

Преобразуем нашу систему к пригодному для итераций виду
\[
  u_{np} = -\frac{1}{C_{np}} \left(  
	p_{np} u_{n+1,p} + B_{np} u_{n-1, p} + D_{np} u_{n, p+1} + E_{np} u_{n,p-1}  \right)
  - \frac{f_{np}}{C_{np}}
\]
Граничное условие тоже можно итеративно решать\note{тут кто-то перепутал знак}
\[
  u_{n,p} = -\frac{d}{h+d}\, u_{n-1,p} + \frac{h}{h+d}\,φ(M)
\]


Докажем, что метод сходится. 
\begin{tproof}
  C граничным условием всё понятно: $\frac{d}{h+d} < \frac{1}{2}$
  а вот с серединкой чуть хитрее
  \begin{enumerate}
    \item $a < 0$. Из диагонального преобладания сходу следует что 
      $\norm\infty{B} < 1$.
    \item $a = 0$. Тут  уже нужно возиться с более точным критерием.

      Пойдём от противного: пусть $\exists\, λ_B \that \abs{λ_B} =1$.
      Пусть $M = \max \abs{v_{np}}$, $Bv = λ_Bv$.
      Обозначим его $\abs{v_{n_0, p_{0}}}$
      Тогда 
      \[
        M = \abs{λ_B} \abs{v_{n_0, p_{0}}}\leqslant 
        \frac{\abs{A_{np}}}{\abs{C_{np}}} \abs{v_{n_0+1, p_0}} +
        \frac{\abs{B_{np}}}{\abs{C_{np}}} \abs{v_{n_0-1, p_0}} +
        \frac{\abs{D_{np}}}{\abs{C_{np}}} \abs{v_{n_0, p_0+1}} +
        \frac{\abs{E_{np}}}{\abs{C_{np}}} \abs{v_{n_0, p_0-1}} \leqslant M
      \]
      Тогда со всей неизбежностью мы получаем что и весь шаблон (крестик)
      равен $M$ по модулю. Таким способом мы неизбежно дойдём до границы
      \[
        \frac{d}{d+h} M = M, \quad \frac{d}{d+h}< \tfrac{1}{2}
      \]
      А того, что выше, не бывает.
  \end{enumerate}
\end{tproof}

\paragraph{Анализ сходимости простейшего итерационного метода для модельной задачи.}

\begin{aux}
  Предупреждение: эти два параграфа пишутся в последние пару часов.
  Качество их весьма сомнительно. Будьте осторожны.
\end{aux}


Рассмотрим уравнение Пуассона в $\R^2$
\[
  Δu = f \iff \pder[2]{u}{x} + \pder[2]{u}{x} = f(x,y)
\]
Поставим граничную задачу Дирихле:
\begin{itemize}
  \item $\ov-Ω = [0;1] × [0;1]$
  \item $\left. u\right|_{Γ} = 0$
\end{itemize}
Зададим квадратную сетку 
\[
	\begin{aligned}
    h &= \frac{1}{N}, & Ω_h &= \left\{ (ih, kh) \in \ov-Ω \right\} \\
	\end{aligned}
\]
Запишем, наконец, разностное уравнение
\[
  \begin{aligned}
    u_{0p} = u_{Np} = 0 &\qquad u_{n0} = u_{nN} = 0 \\
    \frac{u_{n+1,p} - 2u_{np} + u_{n-1, p}}{h^2} + \frac{u_{n, p+1} - 2u_{np} + u_{n,p - 1}}{h^2} &= f_{np}
  \end{aligned}
\]
Решать точно мы его, разумеется, не будем.
Соорудим шаг итераций
\[
  \begin{aligned}
    u_{mp} = \frac{1}4 \left(u_{n+1,p} + u_{n-1, p} + u_{n, p+1} + u_{n,p - 1}\right) 
    - \frac{h^2}{4}\,f_{ik}
  \end{aligned}
\]
В чуть более человеческой форме это выглядит так:
\[
  A_h u_h = f_h \to u_h = Bu_h + g, \quad B = \frac{h^2}{4} A_h + I  
\]
Поскольку $B$ симметричная, процесс сходится в геометрической прогрессии с показателем
\hbox{$\max \abs{λ_B} =: q$}. Так что озаботимся поисками $λ_B$.

Решим уравение на собственные значения $A_h u_h = λ u_h$ методом Фурье (разделения переменных), 
взяв решение сразу в такой форме
\[
  u_{np} = e^{iπm\frac{n}N}\,  e^{iπ\ell\frac{p}N}.
\]
Отсюда несложно получить, что\note{для краткости $λ_A = λ_{A_h}$}
\[
  λ_{ml} = \frac{2}{h^2} \left( \cos \frac{\pi m} N  - 1\right)
  + \frac{2}{h^2} \left( \cos \frac{\pi l}  nN  - 1\right) = λ_A
\]
Отсюда 
\[
  λ_B = \frac{1}{2} \left( \cos \frac {\pi m} N + \cos \frac {\pi l} N\right) 
\]
Оценим:
\[
  \begin{aligned}
    \max {λ_B} &= \cos \frac \pi N < 1 & \max {λ_B} &= \cos \pi \frac{N-1}N > -1 
  \end{aligned}
\]
Короче говоря, $\abs{q} < 1$  и метод славно сходится. Надо только понять как быстро.
\[
  q^n \leqslant ε  \so n \geqslant \frac{\log \frac 1ε}{\log \frac 1q}
\]
Можно говорить, что $\log \frac 1q$~--- это что-то вроде скорости сходимости. 
Оценим её через $h$
\[
  \cos \frac\pi N \approx 1 - \frac {π^2}2\,h^2 \so \log \frac 1q
  \sim \log \left(1+ \frac {π^2}2\,h^2\right) = O(h^2)
\]
Так что для достижения нужной точности потребуется $O(N^2)$ шагов.
Кажется, многовато. И так на шаге $O(N^2)$ операций.

\paragraph{Метод оптимальной верхней релаксации, описание}

Давайте попробуем какое-нибудь улучшить наш итеративный процесс в
сторону улучшения сходимости. Рассмотрим вот такую итеративную схему:
\[
  \begin{aligned}
    \ov~u_{np}^{n+1} &= -\frac{1}{C_{np}} \left(  
    A_{np} u_{n+1,p} + B_{np} u^{n+1}_{n-1, p} + D_{np} u_{n, p+1} + E_{np} u^{n+1}_{n,p-1}  \right)
    - \frac{f_{np}}{C_{np}} \\
      u_{np} &= u_{np} + ω(\ov~u_{ik - u_{ik}}), ω > 0\qquad
      \hbox{(интерполяция\note{а скорее экстраполяция})}
  \end{aligned}
\]
В зависимости от $ω$ схемы называются по-разному:
\begin{description}
  \item[$ω = 1$] Метод Зейделя
  \item[$ω < 1$] Нижняя релаксация
  \item[$ω > 1$] Верхняя релаксация
\end{description}

Какой смысл у такой схемы? Будем обсчитывать всё в сторону увеличения
индексов. На каждом шаге часть узлов мы посчитали вот только что, для
другой части у нас есть приближения с прошлого раза. 
Как-то вот так это выглядит~"--- 
\fbox{$
\begin{smallmatrix}
  & \bullet & \\
 \times &\ast  & \bullet\\
  & \times & \\
\end{smallmatrix}
$}.
Можно вообще хранить всего один массив.

Можно показать, что выбор $ω$ позволяет улучшить скорость сходимости метода.
Собственно, 
\[
  1 < ω_{\text{opt}} < 2,\quad ω_{\text{opt}} = \frac{2}{1 + \sqrt{1-λ_1^2}} \approx
  2 - c_1 h
\]
Здесь $λ_1$ наибольшее по модулю собственное число. Вообще, по идее, можно
подумать что мы решаем вариационную задачу и оптимизировать $ω$. Надеюсь не 
надо нам этого делать. Вообще, как я понял, $ω_{\text{opt}}$ подбирают эмпирически,
сначала на грубой сетке, потом уменьшают шаг и ещё оптимизируют.

Можно ещё увеличить точность, чередуя обходы сетки.
Там вроде $O\left(\sqrt{h}\right)$ получается.

\end{document}
