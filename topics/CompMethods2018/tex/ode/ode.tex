\documentclass{trlnotes}
\usepackage{trmath}
\addcompatiblelayout{commonplace}
\setlayout{commonplace}
\usepackage{trthm}
\usepackage{trsym} 
% % \RequirePackage{stmaryrd}

% \RequirePackage{etoolbox}


%\RequirePackage{cmap}
% \RequirePackage{hyperref}
\theoremstyle{definition}
\newtheorem{thm}{Теорема}
\newtheorem{de}{Определение}
\newtheorem{lm}{Лемма}
\newtheorem{exm}{Пример}
\newtheorem{pr}{Свойство}
\newtheorem{exc}{Упражнение}
\newtheorem{cor}{Следствие}
\newtheorem{st}{Утверждение}
\theoremstyle{remark}
\newtheorem{rem}{Замечание}
\newcommand*{\icom}{\ensuremath\text{\textit{,}}}
\newcommand*{\icol}{\ensuremath\text{\textit{:}}}
\newcommand*{\iscol}{\ensuremath\text{\textit{;}}}
\newcommand*{\com}{\ensuremath\text{\text{,}}}
\newcommand*{\col}{\ensuremath\text{\text{:}}}
\newcommand*{\scol}{\ensuremath\text{\text{;}}}
% \newcommand*{\so}{\ensuremath\Rightarrow}
\newcommand*{\bso}{\ensuremath\Leftarrow}
\newcommand*{\eqv}{\ensuremath\Leftrightarrow}
\newcommand*{\all}{\forall}
\newcommand*{\ex}{\exists \,}
% \newcommand*{\ov}{\overline}
\newcommand*{\un}{\underline}
\newcommand*{\ova}{\overrightarrow}
\newcommand*{\auth}[1]{\hfill \textit{#1}}
\newcommand*{\pd}{\partial}
% \newcommand*{\R}{\mathbb{R}}
% \newcommand*{\N}{\mathbb{N}}
\newcommand*{\DD}{\mathbb{D}}
\newcommand*{\K}{\mathbb{K}}
% \newcommand*{\C}{\mathbb{C}}
% \newcommand*{\C}{\mathbb{C}}
% \newcommand*{\Q}{\mathbb{Q}}
\newcommand{\An}{\wedge}
\newcommand{\Or}{\vee}
% \newcommand*{\Z}{\mathbb{Z}}
\newcommand*{\J}{\mathbb{J}}
\newcommand*{\bas}[2]{\overset{\vspace{-3pt}\tiny{\mb{#1}}}{#2}}
\newcommand*{\mc}{\mathcal}
\newcommand*{\mb}{\mathbf}
\newcommand*{\mf}{\mathfrak}
\newcommand*{\ti}{\textit}
\newcommand*{\la}{\langle}
\newcommand*{\ra}{\rangle}
\newcommand*{\tb}{\textbf}
\newcommand*{\mr}{\mathrm}
\newcommand*{\wt}{\widetilde}


\renewcommand{\arraystretch}{1}

\counterwithin{thm}{section}
\counterwithin{de}{section}
\counterwithin{lm}{section}
\counterwithin{st}{section}
\counterwithin{exm}{section}
\counterwithin{pr}{section}
\counterwithin{cor}{section}
\counterwithin{rem}{section}
\renewcommand{\thethm}{\arabic{section}.\arabic{thm}}
\renewcommand{\thede}{\arabic{section}.\arabic{de}}
\renewcommand{\thelm}{\arabic{section}.\arabic{lm}}
\renewcommand{\thest}{\arabic{section}.\arabic{st}}
\renewcommand{\theexm}{\arabic{section}.\arabic{exm}}
\renewcommand{\therem}{\arabic{section}.\arabic{rem}}
\renewcommand{\thepr}{\arabic{section}.\arabic{pr}}
\renewcommand{\thecor}{\arabic{section}.\arabic{cor}}



\counterwithin{subsection}{section}


% \DeclareMathOperator{\rk}{rk}
\DeclareMathOperator{\Dom}{Dom}
% \DeclareMathOperator{\id}{id}
\DeclareMathOperator{\Cl}{Cl}
% \DeclareMathOperator{\Res}{Res}
\DeclareMathOperator{\im}{Im}
% \DeclareMathOperator{\rot}{rot}
\DeclareMathOperator{\Div}{div}
% \DeclareMathOperator{\grad}{grad}
\DeclareMathOperator{\Id}{Id}
% \DeclareMathOperator{\Aut}{Aut}
% \DeclareMathOperator{\Stab}{Stab}
\DeclareMathOperator{\const}{const}

%\titleformat{\section}
%  {\sffamily\mdseries\upshape\LARGE}
%  {Билет \thesection:}{0.5em}{}



\usepackage{silence}
\WarningFilter{latex}{Reference}
\graphicspath{{../../img/}}

\newtagform{roman}[\renewcommand{\theequation}{\Roman{equation}}]()

\begin{document}
\paragraph{Краевая задача для ОДУ 2 порядка и сведение к задаче Коши}
\label{par:ode::bprobl}
\begin{defn}\label{defn:ode::bprobl}
  Рассмотрим ОДУ 2 порядка
  \[
    y'' + p(x) y' + q(x) y = f(x) \qquad y \in C^2([a;b])
  \]
  и 3 варианта условий на $y$
  \begin{enumerate}[I]
    \item $y(a) = A, \quad y(b) = B$ \label{it:ode::bprobl::cond:i}
    \item $y'(a) = A, \quad y'(b) = B$ \label{it:ode::bprobl::cond::ii}
    \item $y'(a) = α y(a) + A, \quad y'(b) = βy(b) + B$ \label{it:ode::bprobl::cond::iii}
    \end{enumerate}
  
  Если $y$~--- решение для которого выполнено какое-то из условий выше, то $y$~--- решение
  граничной задачи.
\end{defn}

\begin{defn}[Однородная краевая задача]\label{defn:ode::bprobl::hom}
  Положим $f \equiv 0$ в \ref{defn:ode::bprobl}.
\end{defn}
\begin{defn}[Однородные граничные условия]\label{defn:ode::bprobl::hombnd}
  Положим $A = B = 0$ в граничных условиях в \ref{defn:ode::bprobl}
\end{defn}

\begin{thrm}[об альтернативе]\label{thrm:ode::bprobl::alt}
  Рассмотрим однородную граничную задачу с однородными граничными условиями.
  Пусть $y_H$~--- решение однородной задачи. 

  Тогда
  \begin{enumerate}
    \item $y_0 \equiv 0$~--- единственное решение однородной задачи \so неоднородная краевая 
      задача имеет единственное решение
    \item $y_0 \equiv 0$~--- неединственное решение однородной задачи \so неоднородная краевая 
      задача имеет бесконечно много или не имеет решений вовсе
  \end{enumerate}
\end{thrm}


\begin{prf}
  Рассмотреть решение неоднородной краевой в виде $y(x) = y_0(x) + c_1 y_1(x) + c_2 y_2(x)$
  и подставить граничные условия, а дальше все следует из линейной алгебры.
\end{prf}

Разберёмся как численно найти $y_0, y_1, y_2$, потребовавшиеся в предыдущем доказательстве.
Будем считать что $p,q,f$ определены на $I \ni [a;b]$, так что $y$ можно продолжить 
на $(a-ε;b+ε)$. 
\begin{enumerate}
  \item $y(a) = 0$, $y'(a) = 0$. Поскольку $0$ явно решение однородной задачи, то что мы
    найдем будет как раз частным решением неоднородной задачи (Коши!).
  \item $y_H(a) = 1$, $y'_H(a) = 0$ и решаем мы тут однородную задачу (Коши!). Будем считать то что
    нашлось $y_1$
  \item $y_H(a) = 0$, $y'_H(a) = 1$. Скажем что это $y_2$. Здесь важно заметить про линейную
    независимость $y_1$ и $y_2$. Найдем определитель Вронского в точке $a$
    \[
      W = \begin{vmatrix}
        y_1(a) & y_2(a) \\
        y_1'(a) & y_2'(a) 
      \end{vmatrix} = 
      \begin{vmatrix}
        1 & 0 \\ 
        0 & 1 \\
      \end{vmatrix} = 1 \neq 0
    \]
    А тогда он нигде не ноль. А значит $y_1$ и $y_2$ линейно независимы.
\end{enumerate}

В рассуждении выше можно было бы взять другие начальные данные дабы
упростить себе жизнь. Ведь никто не запрещает запихать, например, кусок $y_2$ в $y_0$
(если мы уже знаем правильное $c_2$). Нам просто были нужны какие-то линейно независимые
решения однородной задачи.

Рассмотрим граничную задачу в форме \ref{it:ode::bprobl::cond::iii}
\begin{enumerate}
  \item $y(a) = 0$, $y'(a) = A$, нашли  $y_0$.
  \item $y_H(a) = 1$, $y'_H(a) = α$, нашли $y_1$.
  \item $y_H(a) = 0$, $y'_H(a) = 0$. Мы просто решили что $y_2 \equiv 0$. Эту ЗК мы даже
    не решаем, а сразу знаем ответ.
\end{enumerate}
При таком раскладе $y(x) = y_0(x) + c_1 y_1(x)$. 
Проверим левое граничное условие 
\[
  \begin{split}
    y(a) &= y_0(a) + c_1\, y_1(a) = 0 + c_1 \, 1 = c_1 \\
    y'(a) &= y_0'(a) + c_1\, y_1'(a) = A + c_1 \, α = A + α\ y(a)
  \end{split}
\]
Как видно, всё получилось.

В случае \ref{it:ode::bprobl::cond:i} можно сделать так:
\begin{enumerate}
  \item $y(a) = A$, $y'(a) = 0$, нашли $y_0$.
  \item $y_H(a) = 0$, $y_H'(a) = 1$, нашли $y_1$.
\end{enumerate}
Как видно, свободы в выборе $c_1$ хватает чтобы разобраться с правой границей.

\begin{exmp}
  $y'' - q^2 y =0$, $y(0) = 1$, $y(b) = 1$

  \underdev, а он важный вообще-то, из него необходимость метода прогонки следует.
\end{exmp}

\paragraph{Метод дифференциальной прогонки}
\end{document}
% vim:wrapmargin=3
