%------------------------------------------------------------
% Description : Complex calculus
% Author      : taxus-d <iliya.t@mail.ru>
% Created at  : Wed Jan  4 20:06:34 MSK 2017
%------------------------------------------------------------
\documentclass[12pt,timbord]{../../../notes}
\usepackage{silence}
\WarningFilter{latex}{Reference}
\graphicspath{{../../img/}}

\begin{document}

\paragraph{\underdev Интеграл от комплексной дифференциальной формы}
\label{par:tfcv::compint}
\note{здесь надо сильно больше определений}

\begin{defn}\label{defn:tfcv::compint::limit}
  Определим <<шаровую>> окрестность комплексного числа как $\{z\mid |z-a|<\varepsilon\}$,
  проколотую окрестность как $\{z\mid 0<|z-a|<\varepsilon\}$. Дальше
  можно уже рассмотреть базу таких окрестностей и ввести топологию как в $\R^2$.
  Аналогично вводятся пределы и непрерывности.
\end{defn}
\begin{defn}\label{defn:tfcv::compint::compintform}
  Пусть $G\subset\C$~--- область, $f\colon G\to C$, непрерывна, $f = f_1 + i f_2$, 
  $\omega(z,\del z) = f(z) \del
  z$~--- комплексная дифференциальная форма. \note{Тут определение по cути такое же как и раньше,
  дифференциал имеет символический смысл.} Пусть $\Gamma\subset G$~---  кривая, $\gamma$~--- её
  параметризация, $\gamma = \gamma_1 + i\gamma_2$

  Тогда 
  \[
    \int_\gamma := \int_a^b f(\gamma(t)) \dot\gamma(t) \, \del t 
    := \int_a^b (f_1(\gamma(t))\gamma_1(t) - f_2(\gamma(t))\gamma_2)\del t
    + \int_a^b (f_1(\gamma(t))\gamma_2(t) + f_2(\gamma(t))\gamma_1)\del t
  \]
\end{defn}

\subparagraph{Свойства:}
\begin{prop}\label{prop:tfcv::compint::intprop}
  см~\ref{par:lineint::defs}
\end{prop}
  
\begin{prop}\label{prop:tfcv::compint::intsum}
  Пусть 
  $\{t_i\}$~--- разбиение отрезка $[a;b]$, $z_i = \gamma(t_i)$, $\Delta z_i = z_{i+1} -
  z{i}$, $\tau_i \in [t_i, t_{i+1}]$, $\xi_i = \gamma(\tau_i)$. Пусть ещё
  \[
    \begin{split}
      \sigma = \sum_{i=0}^{n-1} f(\xi_i) \Delta z_i \\
      r = \max {|\Delta z_i|}
    \end{split}
  \]
  Тогда
  \[
    \displaystyle \int_\gamma f(z)\,\del z = \lim_{r\to 0} \sigma
  \]
\end{prop}
\begin{itlproof}
  Следует из вещественной теоремы Римана
\end{itlproof}

\begin{cor}\label{cor:tfcv::compint::length}
  Пусть $|f(z)| \leqslant M\;\: \forall z \in \Gamma$
  \[
    \left|\int_\gamma f(z) \,\del z\right| \leqslant M  \cdot \ell(\Gamma)
  \]
\end{cor}
\begin{itlproof}
  \[
    |\sigma| \leqslant \sum_{i} |f(\xi_i)| \cdot |\Delta z_i| \leqslant M \cdot \sum_i |\Delta
    z_i|
  \]
  А дальше просто  предельный переход в неравенстве.
\end{itlproof}


\begin{verbatim}
.........................
{censored by galactic vimperor}
.........................
\end{verbatim}

\setcounter{paragraph}{29}
\paragraph{Свойства дробно-линейного отображения}
\label{par:tfcv::fraclin}

\begin{defn}[Дробно-линейное отображение]\label{defn:tfcv::fraclin::def}
  $\displaystyle f(z) = \frac{a z + b}{c z + d}, \;\; ad \neq bc$ 
  В $\infty$ определим её как $\lfrac{a}{c}$, а в $-\lfrac{d}{c}$ как $\infty$.
\end{defn}

\begin{prop}\label{thrm:tfcv::fraclin::bij}
  Дробно-линейное отображение~--- гомеоморфизм $\exC$ в $\exC$.
\end{prop}
\begin{defn}\label{defn:tfcv::fraclin::infangle}
  Углом между двумя путями на бесконечности называется угол между образами утих путей при
  отображении $z \mapsto \frac{1}{z}$
\end{defn}
\begin{rem*}
  Геометрическая мотивировка связана с углами между путями через северный полюс сферы Римана.
\end{rem*}

\begin{prop}\label{thrm:tfcv::fraclin::conf}
  Дробно-линейное отображение конформно во всех точках $\exC$
\end{prop}

\begin{prop}\label{thrm:tfcv::fraclin::group}
  Дробно-линейные отображения образуют группу.
\end{prop}

\begin{prop}\label{prop:tfcv::fraclin::cicle}
  Дробно-линейные отображения переводят обобщённые окружности (прямые или окружности) в обобщённые
  окружности.
\end{prop}
\begin{itlproof}
  Дробно-линейное~--- композиция линейного и инверсии (с отражением относительно вещественной оси).
  С линейными всё ясно, а с инверсией надо доказывать. Окружность можно записать уравнением
  \[
    (z-a)(\bar z- \bar a) = R^2
  \]
  А прямую 
  \[
    (z-a)(\bar z - \bar a) = (z-b)(\bar z - \bar b) 
    \Leftrightarrow \overline{(a -  b)} z + (a-b) \bar z + |b|^2 - |a|^2 = 0 
  \]
  Посмотрим, прообразом чего она является
  \[
    \left(w^{-1} - a\right)\left(\bar{w}^{-1} - \bar a\right) 
    = \frac{(1 - aw)(1 - \bar a \bar w)}{|w|^2} = R^2 \Leftrightarrow 
    (|a|^2-1)\, |w|^2 - a \bar w - \bar a w + 1= 0 
  \]
  Дальше есть два случая:
  \begin{description}
    \item[$|a|=1$:] Это уравнение прямой с $|b|\neq |a|$. А такие прямые не проходят через $0$.
      Ну, точки на одной окружности равноудалены от её центра. А центр у неё в 0.
    \item[$|a|\neq 1$] Поделим на $|a|^2 - 1$.
      \[
        \left(w - \frac{a}{|a|^2 -1}\right)
        \overline{\left(w - \frac{a}{|a|^2 -1}\right)} = \frac{|a|^2}{|a|^2-1}-1=\frac{1}{|a|^2-1}
      \]
      а сие есть уравнение окружности.
  \end{description}
  Ну, оставшиеся случаи разбираются аналогично. Разве что прямая через начало координат 
  проще задаётся как 
  \[
    (e^{-ia} - e^{-ib} ) z + (e^{ia} - e^{ib}) \bar z = 0
  \]
  Ну и видно что не будет членов с $|w|^2$~--- выйдет прямая.
\end{itlproof}

\begin{prop}\label{prop:tfcv::fraclin::angarm}
  Дробно-линейное отображение сохраняет ангармоническое отношение:
  \[
    [z_1, z_2, z_3, z_4 ] = \frac{z_1 - z_2}{z_1 - z_3} / \frac{z_4 - z_2}{z_4 - z_4}
  \]
\end{prop}

\begin{prop}\label{prop:tfcv::fraclin::fixpoints}
  Дробно-линейное отображение однозначно задаётся 3 точками и их образами.
\end{prop}

\begin{verbatim}
.........................
{censored by galactic vimperor}
.........................
\end{verbatim}
\setcounter{paragraph}{41}
\paragraph{Классификация изолированных особых точек}
\label{par:tfcv::singclass}

\begin{defn}\label{defn:tfcv::singclass::sing}
  Особой точкой функции $f$ называется точка, где $f$ не голоморфна или не определена.
\end{defn}

\begin{defn}\label{defn:tfcv::singclass::isolsing}
  Изолированной особой точкой функции $f$ называется особая точка, в некоторой окрестности которой
  нет других особых точек.
\end{defn}

\setcounter{paragraph}{45}


\paragraph{Вычисление вычетов в полюсах}
\label{par:tfcv::resuduepole}

\begin{defn}\label{defn:tfcv::resuduepole::polord}
  Пусть $f$ имеет в $a$ полюс.
  Порядком полюса называется наименьшая отрицательная степень в разложении $f$ в ряд Лорана в
  кольце с центром в $a$.
\end{defn}

\begin{thrm}\label{thrm:tfcv::resuduepole::firt}
  Пусть $a$~--- полюс первого порядка функции $f$. Тогда
  \[
    \Res_a f =  \lim_{z\to a} f(z)
  \]
\end{thrm}
\begin{thrm}\label{thrm:tfcv::resuduepole::frac}
  Пусть $a$~--- ноль первого порядка для $\psi$, $\varphi(a) \neq 0$, $\varphi, \psi$ голоморфны в
  $U(a)$, $f = \frac{\varphi}{\psi}$. Тогда
  \[
    \Res_a f =  \frac{\varphi(a)}{\psi'(a)}
  \]
\end{thrm}

\begin{thrm}\label{thrm:tfcv::resuduepole::pirt}
  Пусть $a$~--- полюс $p$-го порядка функции $f$. Тогда
  \[
    \Res_a f =  \frac{1}{(p-1)!} \Biggl((z-a)^p f(z)\Biggr)^{(p-1)}_{z=a}
  \]
\end{thrm}

\paragraph{Вычисление интегралов с помощью вычетов}
\label{par:tfcv::intresidue}

\begin{enumerate}[I$\rangle$]
  \item Интеграл по периоду от периодической функции. \par
    Пусть $f\colon \R \to \C$. Тогда
    \[
      f = 2\pi i \, \sum_{a_k} \Res_{a_k} g,
    \]
    где $a_k$~--- вычеты функции $g(z)$, внутри единичной окружности. В функции $g$ $\sin/\cos$
    заменены на $\frac{1}{2} \left(z \pm z^{-1}\right)$ 
  \item Интеграл от рациональной функции на $\R$ \par
    Пусть $R(x) = \frac{P(x)}{Q(x)}$, $P, Q \in \R[x]$, $\deg P \leqslant \deg Q-2 $.  Тогда 
    \[
      \int_{-\infty}^{\infty} R(x) \,\del x= 2 \pi i \sum_{\Im a_k > 0} \Res_{a_k} R(z)
    \]
  \item $\displaystyle \int_{-\infty}^{\infty} f(z) \, e^{i \lambda z} \, \del z = I$ \par
  Пусть $f(z) \xrightarrow[z\to \infty]{}  0$, голоморфна всюду кроме $\{a_k\}$, нету особых точек
  на $\R$. Тогда
  \[
    I = 2 \pi i \sum_{\Im a_k > 0} \Res_{a_k} f(z) e^{i\lambda z}
  \]
\end{enumerate}

\begin{lem}[Жордана]\label{lem:tfcv::intresidue::jordan}
  Пусть $f$ голоморфна всюду кроме счётного числа особых точек, $f(z) \xrightarrow[z\to \infty]{}  0$.
  Тогда
  \[
    \int_{\Gamma_R} f(z) \, e^{i \lambda z} \, \del z \xrightarrow[R \to \infty]{} 0
  \]
\end{lem}

\setcounter{paragraph}{54}
\paragraph{Классические односвязные области. Теорема Римана}
\label{par:tfcv::riemanaut}

\begin{defn}\label{defn:tfcv::riemanaut::isom}
  Комплексным изоморфизмом областей $G$ и $H$ называется однолистное конформное отображение
  \note{Тут хватит и голоморфности с сюръективностью,
  ведь из однолистности производная нигде не обращается в 0}
  $f \colon G \to H$. Область $G$ и $H$ тогда называются и конформно эквивалентными (изоморфными).
\end{defn}
\begin{rem*}
  $f\colon G \to G$ при условиях выше~--- автоморфизм.
\end{rem*}

\begin{prop}\label{prop:tfcv::riemanaut::autgroup}
  Все автоморфизмы области $G$ с операцией композиции образуют группу $\Aut G$.
\end{prop}
\begin{itlproof}
  Пусть $f, g, h\in \Aut G$. Тогда $f \circ g \colon G \to G$, композиция биекций~--- биекция. Так
  что операция задана корректно.
  \begin{itemize}
    \item $\bigl(f \circ (g\circ h)\bigr)(x) = f(g(h(x))) = \bigl((f \circ g)\circ h\bigr)(x)$
    \item $\forall\, f\; \exists\,f^{-1}$, обратное~--- голоморфно и биекция, $ \Rightarrow $
      конформно и однолистно.
    \item $\id\colon G\to G$~--- конформно и однолистно. 
  \end{itemize}
\end{itlproof}

\subparagraph{Классические области}
\begin{enumerate}
  \item $\overline{\C}$
  \item ${\C}$
  \item $\D = \{z\mid |z| < 1\}$
\end{enumerate}

\begin{thrm}[Римана]\label{thrm:tfcv::riemanaut::rieman}
  Пусть область $G \subset \exC $. Тогда $G\cong$ одной из классических областей
  \begin{enumerate}
    \item $G = \exC \Rightarrow G \cong \exC$
    \item $G = \exC \setminus \{a\} \Rightarrow G \cong \C$
    \item $G = \exC \setminus U \Rightarrow G \cong \D$, $|U| > 1$
  \end{enumerate}
\end{thrm}

\paragraph{Лемма Шварца}
\label{par:tfcv::shwartz}


\paragraph{Лемма о подгруппе группы автоморфизмов}
\label{par:tfcv::subautgr}

\begin{defn}\label{defn:tfcv::subautgr::trans}
  Пусть $\Gamma < \Aut G$. Тогда говорят, что $\Gamma$~--- транзитивна, если
  \[\forall\, z_1, z_2 \in G\;\: \exists\, f \in \Gamma\colon f(z_1) = z_2\]
\end{defn}
\begin{rem*}
  Лучше конечно говорить, что действие группы автоморфизмов на $G$ транзитивно.
\end{rem*}

\begin{lem}\label{lem:tfcv::subautgr::subgrp}
  Пусть область $G\subset \exC$, $\Gamma$~--- транзитивна. Пусть к тому же $\exists\, z_0 \colon
  \Stab(z_0) < \Gamma$. Тогда $\Gamma=\Aut G$.
\end{lem}
\begin{itlproof}
  Выберем произвольный $f \in \Aut G$, пусть $z_1 = f(z_0)$. Из транзитивности $G$ 
  $\exists\, \gamma\in \Gamma\colon \gamma(z_1) = z_0$. Тогда $h = \gamma\circ f\in \Stab (z_0)$.
  Но из второго условия $\Stab (z_0) < \Gamma \Rightarrow h \in \Gamma$. Но тогда 
  \[
    \forall\, f\in \Aut G  \;\;f = \underbrace{\gamma^{-1}}_{\in \Gamma} \circ \underbrace{h}_{\in
    \Gamma} \in \Gamma
  \]
\end{itlproof}

\paragraph{Автоморфизмы классических областей}
\label{par:tfcv::classaut}
Здесь всё константы по умолчанию $\in \C$.
\begin{thrm}\label{thrm:tfcv::classaut::exC}
  $\displaystyle \Aut \exC  = \{f \mid f(z) = \frac{az+b}{cz+d}, ad-bc \neq 0\}$
\end{thrm}
\begin{ittproof}
  Пусть \[
    \Gamma = {f\mid f(z) = \frac{{a z + b}}{cz+d}}, \Gamma < \Aut \exC
  \]
  Композиция дробно-линейных~--- дробно-линейна, обратное~--- тоже дробно-линейно. Так что
  подгруппа. 

  Она транзитивна, для $\C$ хватит и линейного (сдвиг), а как отправить что-то в бесконечность, понятно.
  Давайте посмотрим, чему равен $\Stab \infty$. Нам нужно чтобы $\infty \mapsto \infty$. А значит
  $\C \mapsto \C$. Но из теоремы~\ref{thrm:tfcv::classaut::C} это линейные функции. А они явно
  входят в дробно-линейные. Так что $\Stab \infty < \Gamma$. А тогда по
  лемме~\ref{lem:tfcv::subautgr::subgrp} $\Gamma = \Aut \exC$
\end{ittproof}

\begin{thrm}\label{thrm:tfcv::classaut::C}
  $\displaystyle \Aut \C  = \{f \mid f(z) = {az+b}, a\neq 0\}$
\end{thrm}
\begin{ittproof}
  Пусть $A = U(\infty)$. Бесконечность~--- явно особая точка, надо подумать только какая.
  
  Пусть $\infty$~--- существенно особая точка. Но тогда по теореме Сохоцкого $f(A)$ всюду плотно в
  $\C$. А значит в $U(0) \subset \C\setminus U(\infty)$ есть точка из $f(A)$~--- проблемы с
  однолистностью (она же инъективность).

  Пусть $\infty$~--- устранимая особая точка. Но тогда в кольце $U(\infty)$
  \[
    f(z) = \frac{c_{-k}}{z^k} + \dotsb + c_0
  \]
  Но $f\in  \Aut G \Rightarrow f$ голоморфна в $0$. Беда

  Выхода нет~--- в $\infty$~--- полюс. Но тогда $f(z)$~--- какой-то полином, ведь для полюса нужно
  ограниченное число  членов в главной части ряда Лорана. Но любой полином степени $n
  $ имеет в $\C$ ровно $n$ корней. А у нас функция однолистная. Так что подходят полиномы лишь
  первой степени. Константу тоже нельзя, проблемы с однолистностью. \note{Все утверждения про
  полюс в бесконечности можно получить, рассмотрев $f(\lfrac 1 z)$ в $U(0)$}
\end{ittproof}

\begin{thrm}\label{thrm:tfcv::classaut::D}
  $\displaystyle \Aut \D  = \{f \mid f(z) = e^{i \theta}\frac{z-a}{1-\bar a z }, \theta \in \R,
  |a|<1 \}$
\end{thrm}
\begin{ittproof}
  Опять рассмотрим $\Gamma$ как в условии и покажем, что $\Gamma = \Aut \D$. Надо сначала показать
  хотя бы, что $ \Gamma < \Aut \D$. 
  \[
    \left| e^{i\theta}\, \frac{z-a}{1 - \bar a z} \right|  
  \]
  Проще всего домножить на сопряжённое
  \[
    \begin{split}
      \left| \frac{z-a}{1 - \bar a z} \right|^2
      = \frac{(z-a)\,(\bar z -  \bar a)}{(1 - \bar a z)\,(1 - a \bar z) }
      = \frac{|z|^2  -  a\bar z - z\bar a + |a|^2 }{1 - \bar a z - a \bar z + |a|^2|z|^2 } < 1
      \Leftrightarrow |z|^2 + |a|^2 < 1 + |a|^2 |z|^2 \Leftrightarrow (|a|^2 - 1)(|z|^2 - 1)> 0
    \end{split}
  \]
  Так что при $|z| < 1 \land |a| < 1$ это верно.
  
  Дальше легко найти обратное к $\gamma(z) = w$
  \[
  \gamma^{-1} (w) = \frac{w-e^{i\theta }}{w \bar a - e^{i\theta }} 
  = e^{i\theta_1} \frac{a_1 - z}{1- \bar a_1 z} \;\; (a_1 = e^{i\theta} a \in \D)
  \]

  С композицией тоже несложно разобраться
  \begin{align*}
    f_1(z) &= \frac{z- a_1}{1 -\bar a_1 z} \\
    f_2(z) &= \frac{z- a_2}{1 -\bar a_2 z} \\ 
    a &= \frac{a_1e^{-i \theta} + a_2}{1 + a_1 \bar a_2 e^{-i\theta}} 
      & \left| a \right| = |e^{-i\theta}f_1(-a_2e^{i\theta})|<1\\ 
    f_2(f_1(z)) &= e^{i\theta_2}\frac{e^{i\theta} z- e^{i\theta} a_2 -a_1 + a_1\bar a_2 z}%
    {1 + \bar a_1 a_2 e^{i\theta}  - \bar a_1e^{i\theta} z - \bar a_2 z}
    = \frac{z -a }{1  - \bar a z}
  \end{align*}
  
  Осталось показать оба условия из леммы~\ref{lem:tfcv::subautgr::subgrp}

  \begin{enumerate}
    \item Пусть $z_1, z_2\in \D$. Будем строить так: $z_1 \mapsto 0 \mapsto z_2$
      \begin{align*}
          f_1(z) &= \frac{z-z_1}{1-\bar z_1 z} & f_2^{-1} (z) &= \frac{z-z_2}{1-\bar z_2 z}
                 & f &= f_2 \circ f_ 1
      \end{align*}
    \item Посмотрим на $f \in \Stab 0$. По лемме Шварца $\forall\, z \in D\; |f(z)| \leqslant |z|$.
      Поскольку $\Stab 0$~--- группа, $\exists\, f^{-1}$ и 
      \[
        |z| = |f^{-1} (f(z)) | \leqslant |f(z)| \Rightarrow |f(z)| = |z|.
      \]
      А тогда по второму пункту леммы Шварца $f(z)= cz$, $|c|=1 \Rightarrow  c = e^{i\theta}$. 
      Следовательно, $\Stab 0 < \Gamma$. Тогда по уже упомянутой лемме $\Gamma = \Aut \D$
  \end{enumerate}
\end{ittproof}

\end{document}
% vim:wrapmargin=3
