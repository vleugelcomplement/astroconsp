%------------------------------------------------------------
% Description : Calculus in \R^n space
% Author      : taxus-d <iliya.t@mail.ru>
% Created at  : ??? 
%------------------------------------------------------------
\documentclass[12pt]{../../../notes}
\usepackage{silence}
\WarningFilter{latex}{Reference}
\graphicspath{{../img/}}

\begin{document}

\paragraph{Оценка приращения дифференциального отображения}
\label{par:diffspace::diffestim}

\begin{prop}\label{prop:diffspace::diffestim::lagrfail}
  Пусть $f\colon \R^n \to \R^m$, $m \geqslant 2$. Тогда формула Лагранжа
  \[
    f(b) - f(a) = f'(c)(b - a)
  \]
  не работает.
\end{prop}
\begin{exmp*}\label{exmp:diffspace::diffestim::lagrfail}
  Пусть 
  \[
    f(t) := (\cos t, \sin t), b - a = 2\pi
  \]
\end{exmp*}

\begin{thrm}[об оценке приращения отображения]\label{thrm:diffspace::diffestim::diffestim}
  Пусть $f\colon G \subset \R^n \to \R^m$, $G$~--- выпуклое, $f$~--- дифференцируема, 
  \[ 
    \forall\, x \in G \;\: \| f'(x) \| \leqslant M 
  \]
  Тогда $\forall\, a,b \in G \;\: \|f(b) - f(a)\| \leqslant M \| b-a \|$
\end{thrm}
\begin{ittproof}
  <<Окружим>>  исходную функцию:
  \[
    F = \psi \circ f \circ \varphi
  \]
  где 
  \begin{align*}
    \varphi&: \R^n \to \R^m & \varphi(t) &:= t(b-a) + a, & t&\in [0,1] \\
    \psi&: \R^m \to \R & \psi(y) &:= \langle y, \ell \rangle, & \ell &= f(b)-f(a) 
  \end{align*}
  Заметим, что $F$~--- обычная вещественнозначная функция. Так что для неё работает формула
  Лагранжа:
  \[
    \exists\, c \in [0,1] \colon F(1)-F(0) = F'(c)(1-0) = F'(c)
  \]
  Тогда из свойств нормы (по ходу дела обозначим $\varphi(c)$ за $x$):
  \[
    \|F'(c)\| = \|\psi'(f(x)) \cdot f'(x) \cdot \varphi'(c)\| \leqslant 
    \|\psi'(f(x)) \| \cdot \| f'(x) \| \cdot \| \varphi'(c)\|
  \]
  Здесь тонкость в обозначениях. Производные~--- вроде матрицы, поэтому их нормы~--- что-то
  странное на первый взгляд. На самом деле смысл немного иной. 
  \[
    \mathrm d L(x, h) = f'(x) \cdot h
  \]
  Таким образом, дифференциал~--- неплохое линейное отображение. А под <<нормой производной>>
  имеется в виду норма соответствующего линейного отображения.

  Теперь давайте что-нибудь скажем про эти нормы. 
  \begin{enumerate}
    \item $\varphi'(t) = (b-a) \Rightarrow \| \varphi'(c) \| = \| b - a\|$
    \item $\psi(y) = \langle y, l \rangle$, $\| \psi\| = \|\ell\|$      
  \end{enumerate}
  Так что 
  \[
    \| F'(c)\| \leqslant M \cdot \| \ell \| \cdot \|b-a\|
  \]
  С другой стороны:
  \[
    F(1) - F(0) = \psi(f(b)) - \psi(f(a)) = \langle f(b), \ell \rangle - \langle f(a), \ell \rangle
    = \langle \ell, \ell \rangle  = \|\ell \|^2
  \]
  В итоге, совмещая оба выражения, приходим к утверждению теоремы.
\end{ittproof}

\paragraph{Частные производные высших порядков}
\label{par:diffspace::highpartial}

\begin{defn}\label{defn:diffspace::highpartial}
  Пусть $f \colon G \subset \R^n \to \R$, существуют производные $k$-го порядка. Тогда
  \[
    \partial_{i_1, \dotsc, i_{k+1}}^{k+1} f(x) := \partial_{i_{k+1}}(\partial^k_{i_1, \dotsc, i_{k+1}} f) (x)
  \]
\end{defn}
\begin{rem}\label{rem:diffspace::highpartial::smooth}
  $C^p(G)$~--- класс функций, определённых в $G$ с непрерывной производной до $p$-го порядка
  включительно.
  Функции из $C^1$ ещё называются гладкими.
\end{rem}

\begin{thrm}[Зависимость производных $p$-го порядка от перестановки переменных]
  \label{thrm:diffspace::highpartial::permut}
  Пусть \mbox{$f \in C^p(G)$}, $x\in G$. При этом
  \[
    \begin{split}
      i &= \{ i_1, \dotsc, i_p \mid i_k \in \{1, \dotsc, n\}\} \\
      j &= \{ j_1, \dotsc, j_p \mid j_k \in \{1, \dotsc, n\}\} \\
      j &= \pi(i)
    \end{split}
  \]
  Тогда 
  $\partial_i^{\,p} f(x) = \partial_j^{\,p} f(x)$
\end{thrm}
\begin{rem}\label{rem:diffspace::highpartial::permut}
  Тут важно, что есть целая окрестность. Одной точки не хватит.
\end{rem}

\paragraph{<<Многомерный>> дифференциал высоких порядков}
\label{par:diffspace::highdiff}

\begin{defn}\label{defn:diffspace::highdiff}
  Пусть $f \colon G \subset \R^n \to \R$, $f\in C^p(G)$
  \[  
    \del^p f(x) := \sum_{1 \leqslant i_1 \leqslant \dotsb \leqslant i_p \leqslant n}
    \frac{\partial^p f}{\partial x_{i_p} \dotso \partial x_{i_p}} \, \del x_{i_1} \dotsm \del x_{i_p}
  \]
\end{defn}

\begin{stat}\label{stat:diffspace::highdiff::binom}
  Если частные производные можно переставлять, то
  \[
    \del^p f(x) = \sum_{\substack{\alpha_i \geqslant 0 \\ \sum \alpha_i = p}} 
    \frac{p!}{\alpha_1! \dotsm \alpha_n!} \, 
    \frac{\partial^p f}{\partial x_1^{\alpha_1}\dotso \partial x_n} \,\del x_{i_1} \dotsm \del x_{i_p}
  \]
\end{stat}

\paragraph{Формула Тейлора для функций многих переменных}
\label{par:diffspace::taylor}

\begin{thrm}\label{thrm:diffspace::taylor}
  Пусть $f \in C^p(G), G\in \R^n$, $a\in G$. Пусть также $h\in \R^n \colon a+h \in G$.
  Тогда 
  \[
    f(a+h) = \sum_{k=0}^p \frac{1}{k!} \, \del^k f(a,h) + R_p(h)
  \]
  Остаток $R_p(h)$ можно представить несколькими способами:
  \begin{enumerate}
    \item В форме Пеано: $R_p(h) = o(\|h\|^p)$
    \item В форме Лагранжа $R_p(h) = \dfrac{1}{(p+1)!}\, \del^{p+1} f(a + \theta h, h)$, 
      $\theta \in (0,1)$
  \end{enumerate}


\end{thrm}

\end{document}
