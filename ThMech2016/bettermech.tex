%------------------------------------------------------------
% Description : 
% Author      : Iliya Tikhonenko <iliya.t@mail.ru>
% Created at  : Fri Jun 16 18:12:20 MSK 2017
%------------------------------------------------------------
\documentclass[timbord]{longnotes}
\usepackage{tmath}
\usepackage{cussymb}
\graphicspath{{img/}}

\makeatletter
\let\old@v=\v
\def\v{\@ifstar\v@star\v@choice}
\def\v@choice#1{\ifmmode \v@vec{#1} \else \old@v{#1} \fi}
\def\v@star#1{\v@lin{#1}}
\def\v@vec#1{\mathbf{#1}}
\def\v@lin#1{\mathrm{#1}}
\makeatother

\begin{document}
\chapter{Кинематика точки}

\setcounter{paragraph}{1}
\paragraph{Косоугольные координаты}

Здесь можно немного добавить строгости, а то ничерта не понятно.
Пусть $V$~--- евклидово пространство (линейное со скалярным произведением).
Как нам определяли, $g_{ik} = \v{e_i} \cdot \v{e_k}$,
\[
   \v a \cdot \v b  = \sum_{ij}  a^i b^j g_{ij}
\]
Здесь $a^k$~--- коэффициенты разложения по $\v{e_k}$~--- называются контравариантными координатами.

Пусть $V^*$~--- сопряжённое к $V$, его базисом являются координатные функции
$\v*{f_k} \that \v*{f_k}(\v x) = x^k$. 
Поскольку задано скалярное произведение, задан канонический изоморфизм $V \to V^*$.
Нам, правда, потребуется $V^* \to V$.

Введём ещё одну систему \emph{векторов} в $V$ : $\v{e^k} = \v*{f_k^*}$, то есть 
$\v*{f_k}(\v x) = \v{e^k} \cdot \v x$. 
Она и называется взаимным
базисом, коэффициенты разложения по ней~--- ковариантные координаты.
Из линейности скалярного произведения, ровно такие же координаты будут у соответствующей
формы в $V^*$.
Линейную независимость легко получить из ЛНЗ $\v*{f_k}$, a
раз их $\dim V$, то полученные векторы являются базисом.

Так что можно сформулировать правило:
\begin{itemize}
  \item Контравариантные координаты~--- коэффициенты разложения по базису линейного пространства.
  \item Ковариантные координаты~--- коэффициенты разложения по базису пространства линейных форм.
\end{itemize}
Ещё можно определить $g^{ij} = \v{e^i \cdot e^j}$, и перенести это на 
соответствующие линейные формы. Обобщая дальше, можно вообще сказать, что $g_i^k = \delta_{ij}$.
Тогда $g$ будет задавать действие формы на вектор. Вроде физикам это зачем-то надо.

А после тирады выше уже развлекаться с индексами.

\begin{prop}
  $\v{e^k} \cdot \v{e_j} = \delta_{kj}$
\end{prop}
\begin{lproof}
  Следует из определения координатной функции, ведь $\v{e^k} \cdot \v x = \v*{f_k}(\v x)$
\end{lproof}

\begin{prop}
  $\v a\cdot \v b = \sum_i a^i b_i$
\end{prop}

\begin{prop}
  Пусть $\v r = \sum_k \xi^k \v {e_k}$ и $=\sum_k\xi_k \v{e^k}$.
  Тогда $\xi_k = \v r \cdot \v{e_k} = \sum_j \xi^j g_{jk}$
\end{prop}
\begin{lproof}
  Ну, $\v{r} \cdot \v{e_k} = \dsum_j \xi_j \, \v{e^j}\cdot \v{e_k} 
  = \sum_j \xi_j \,\delta_{jk} = \xi_k$. Вроде всё.
\end{lproof}

Аналогичная ситуация с $\xi^k$.
\begin{prop}
  $\xi^k = \v{r\cdot e^k} = \sum_j \xi_j g^{jk}$.
\end{prop}

\begin{prop}
\[
  \v{e^k} = \sum_j g^{jk} \v{e_j}, \quad \v{e_k} = \sum_j g_{jk} \v{e^j}
\]
\begin{lproof}
  Первое домножить на $\v{e^i}$, второе на $\v{e_i}$.
\end{lproof}
\end{prop}
\begin{prop}
  $\sum_{i} g^{i\ell}g_{ik} = \delta_{\ell k}$
\end{prop}
\begin{lproof}
%   Дельта-символ, как и дельта-функция проявляется в сумме (или интеграле).
  \[
    \sum_{i} g^{i\ell}g_{ik} = \sum_i g^{i\ell} \v{e_i \cdot e_k} = \v{e^\ell}\cdot \v{e_k} =
    \delta_{\ell k}
  \]  
\end{lproof}


Как видно, когда определения безкоординатные, жызнъ прекрасна!. 
\note{тут не опечатка, а отсылка к известной картинке \texttt{;)}}
\appendix
\chapter{Обозначения}
\begin{description}
  \item[$\v*{f}$]~--- линейная форма. \hfill$\langle$\verb+\mathrm f+$\rangle$
  \item[$\v{x}$]~--- вектор. \hfill$\langle$\verb+\mathbf x+$\rangle$
\end{description}
\end{document}
% vim:tw=100 cc=100
