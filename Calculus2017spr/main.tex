\documentclass[exam,timbord]{longnotes}
% \usepackage[backend=biber]{biblatex}
\usepackage{tmath}
\usepackage{cussymb}
\usepackage{docmute}
\graphicspath{{img/}}

% \addcontentsline{toc}{chapter}{Литература}
% \bibliographystyle{unsrt}
% \bibliography{stat}

\title{Меры и меры по борьбе с ними}
\date{\today}



\author{Лектор: А.~А.~Лодкин \\
Записал :\texttt{ta\lower 0.1em \hbox{x}us}}

\begin{document}
 
\maketitle
\epigraph{
  \sl
  А эти множества? \\
  Какой для них язык?.. Гор\'e душа летит,  \\
  Все необъятное в единый вздох теснится, \\
  И лишь молчание понятно говорит.
}{Студент на экзамене по теории меры}

\tableofcontents
\clearpage


\chapter{Теория меры и интегралы по мере}
\setcounter{paragraph}{0}
%------------------------------------------------------------
% Description : 
% Author      : Iliya Tikhonenko <iliya.t@mail.ru>
% Created at  : Tue Feb 21 00:00:16 MSK 2017
%------------------------------------------------------------
\documentclass[12pt, timbord]{../../../notes}
\usepackage{silence}
\WarningFilter{latex}{Reference}
\graphicspath{{../../img/}}

\begin{document}
\paragraph{Системы множеств}
\label{par:meas::setsys}

\begin{defn}\label{defn:meas::setsys::sub}
  Пусть здесь (и дальше) $X$~--- проивольное множество. Тогда $\pset P(X) \equiv 2^X$~--- множество всех 
  подмножеств $X$.
\end{defn}
\begin{exmp*}
  $X = \{1 \intrng n\} \Rightarrow \# \pset (X) = 2^n$ (это количество элементов, если что)
\end{exmp*}

\begin{defn}[Алгебра]\label{defn:meas::setsys::alg}
  Пусть $\alg \subset \pset (X)$. Тогда $\alg$~--- алгебра множеств, если
  \begin{enumerate}
    \item $\varnothing \in \alg$
    \item $X \in \alg$
    \item $A,B\in \alg \Rightarrow A \cap B, A \cup B, A \setminus B \in \alg$
  \end{enumerate}
\end{defn}
\begin{rem*}\underdev
  Заметим, что в алгебре пересечение (или объединение) \emph{конечного} числа её элементов лежит в алгебре. 
  Это можно доказать простой индукцией. А вот для бесконечных объединений пользоваться индукцией уже нельзя.
\end{rem*}
\end{document}



\chapter{Дифференциальная геометрия \underdev}
\setcounter{paragraph}{30}
%------------------------------------------------------------
% Description : Differential Geometry
% Author      : Iliya Tikhonenko <iliya.t@mail.ru>
% Created at  : Thu Jun  1 18:33:27 MSK 2017
%------------------------------------------------------------
\documentclass[12pt,timbord]{longnotes}
\usepackage{tmath} 
\usepackage{cussymb} 
\usepackage{silence}
\WarningFilter{latex}{Reference}
\graphicspath{{../../img/}}

\begin{document}
\paragraph{Регулярная кривая и её естественная параметризация}
\label{par:dg::curve}

\begin{defn}[Кривая, как отображение]\label{defn:dg::curve::map}
  Пусть задано гладкое отображение $t\in [a;b] \mapsto r(t) \in \R^3$, регулярное, то есть
  $rk r'(t) \equiv 1$. $t$~--- параметр, само отображение ещё можно
  называть параметризацией.

\end{defn}

\begin{defn}[Кривая, как класс отображений]\label{defn:dg::curve::class}
  Введём отношение эквивалентности отображений:
  \[
    r(t) \sim \rho(\tau) \Leftrightarrow 
    \exists\, \delta \colon [a;b] \leftrightarrow [\alpha, \beta ] \that \rho (\delta (t)) = r(t)
  \]
\end{defn}

А теперь будем их путать. \flame

\begin{defn}[Естественная параметризация]\label{defn:dg::curve::nat}
  Пусть $[a;b] = [t_0, t_1]$.
  Рассмотрим $\ov~s(t) = \dint_{t_0}^t | r'(t) | \, \del \tau $. Она, как видно, является 
  пройденным путём и неубывает  $ \Rightarrow $ годится на роль $\delta$.

  Так что можно рассматривать $s$ как параметр,  это собственно и есть
  естественная (натуральная) параметризация.
\end{defn}


\begin{prop}\label{prop:dg::curve::repar}
  Пусть есть две разных параметризации: $r(t)$ и $r(s)$ одной кривой. Тогда 
  \[
    \dot r \equiv \pder{r(s)}{s} = \Bigl(r'(t) \cdot (s'(t))^{-1}\Bigr)(t) = \frac{r'}{|r'|} 
  \]
  Как видно, натуральная почему-то обозначается точкой.
\end{prop}

\paragraph{Кривизна кривой}
\label{par:dg::curvature}

\begin{defn}[Касатальный вектор]\label{defn:dg::curvature::tang}
  $\tau := \dot r(s)$.
\end{defn}

\begin{defn}[Кривизна]\label{defn:dg::curvature}
  $k_1 = |\dot \tau |$
\end{defn}
\begin{defn}[Радиус кривизны]\label{defn:dg::curvature}
  $R = k_1 ^{-1}$
\end{defn}

\begin{prop}\label{prop:dg::curvature::orth}
  $\tau \perp \dot \tau$
\end{prop}

\begin{thrm}\label{thrm:dg::curvature::repar}
  $\displaystyle k_1 = \frac{| r' \times r''|}{|r'|^3} $
\end{thrm}

\paragraph{Кручение и нормаль}
\label{par:dg::norm}

\begin{defn}[Нормаль]\label{defn:dg::norm::norm}
  Пусть $k_1 \neq 0$. Тогда $\nu  := \frac{\dot \tau } {k_1}$.
\end{defn}
\begin{defn}[Бинормаль]\label{defn:dg::norm::binorm}
  $\beta = \tau \times \nu $.
\end{defn}

\begin{rem*}
  $(\tau,\nu,\beta)$~--- хороший кандидат для репера в какой-нибудь точке $P$.
\end{rem*}
\begin{defn}[Соприкасающаяся плоскость]\label{defn:dg::norm::tangpl}
  Пусть $k_1 > 0$, $P = r(s_0)$, $T$~--- плоскость, $T \ni P$, $N \perp T$~--- нормаль к ней.
  Допустим, $r(s+\Delta s) \cdot N  = h$, $h = o(\Delta s^2)$. Тогда $T$~--- соприкасающаяся
  плоскость.
\end{defn}
\begin{prop}\label{prop:dg::norm::tangpl}
  $ \tau , \nu \perp N  $;
  $(r-r_0, \dot r_0 , \ddot r_0)=0$~--- её уравнение
\end{prop}

\begin{defn}[Абсолютное кручение]\label{defn:dg::norm::abscrvn}
  $|k_2| := | \dot\beta|$
\end{defn}
\begin{thrm}\label{defn:dg::norm::abscrvn}
  $|k_2| = \left|\dfrac{(\dot r, \ddot r, \dddot r\,)}{k_1^2}\right|$
\end{thrm}

\begin{defn}[Кручение]\label{defn:dg::norm::crvn}
  $k_2 := \dfrac{-(\dot r, \ddot r, \dddot r\,)}{k_1^2}$
\end{defn}

\paragraph{Формулы Френе}
\label{par:dg::frene}

\begin{thrm}\label{thrm:dg::frene}
  \begin{equation}
    \label{eq:dg::frene}
    \begin{pmatrix}
      \dot\tau \\\dot\nu \\ \dot\beta
    \end{pmatrix}
    = 
    \begin{pmatrix}
      0 & k_1 & 0 \\
      -k_1 & 0 & -k_2 \\
      0 & k_2 & 0
    \end{pmatrix}\cdot
    \begin{pmatrix}
      \tau \\ \nu \\ \beta
    \end{pmatrix}
  \end{equation}
\end{thrm}

\begin{thrm}\label{thrm:dg::frene::reconst}
  Пусть $r(s)$~--- гладкая кривая с заданными $k_1$ и $k_2$, $k_1>0$. 
  Тогда система \eqref{eq:dg::frene} определит её с точностью до движения.
\end{thrm}

\paragraph{Регулярная поверхность. Касательная плоскость. Первая квадратичная форма}
\label{par:dg::tangplane}

\begin{defn}[Поверзность (двумерная)]\label{defn:dg::tangplane::manifold}
  Пусть задано гладкое отображение \[
    \varphi \colon (u,v) \in D \subset \R^2 \mapsto r=(x,y,z) \in \R^3
  \]
  Добавим условие регулярности $\rk \varphi' \equiv 2$ и условимся путать отображение и класс 
  оных.
\end{defn}

\begin{defn}\label{defn:dg::tangplane::tanv}
  \[
    \begin{split}
      r_u &:= (x_u', y_u', z_u') \\
      r_v &:= (x_v', y_v', z_v') \\
      n &:= \frac{r_u \times r_v}{|r_u \times r_v|} = \frac{N}{|N|} 
    \end{split}
  \]
  Отметим, что условие регулярности не дает векторному произведению обращаться в 0.
  
  Касательную плоскость можно было бы здесь определить через нормаль, но лучше пока ещё подумать.
  Может, абстракций добавить.
\end{defn}



\begin{defn}[Первая квадратичная форма]\label{defn:dg::tangplane::I}
  \[
    \begin{split}
      I :&= |\del r|^2  = r_u^2\, \del u^2 + 2r_u r_v\, \del u \,\del v + r_v^2\, \del v^2 \\
        &= E \, \del u^2 + 2F\, \del u \,\del v + G\, \del v^2 
    \end{split}
  \]
\end{defn}

\paragraph{Вычисление длин и площадей на поверхности}
\label{par:dg::area}

\begin{thrm}\label{thrm:dg::area::len}
  Пусть $M$~--- поверхность, $\gamma \colon t \to r \in M$.
  Тогда \[
    \ell(\gamma)= \dint_{t_0}^{t_1} \sqrt I.\; (\del s = I)
  \]
\end{thrm}

\begin{thrm}\label{thrm:dg::area::area}
  Пусть $M$~--- поверхность, $u,v\in D$, $I = E \, \del u^2 + 2F\, \del u \,\del v + G\, \del v^2$.
  Тогда
  \[
    S(M) = \iint_D \sqrt{EG - F^2}\, \del u\,\del v
  \]
\end{thrm}


\quest\plholdev{вкусный абстрактный кусок про меру на многообразии}

\begin{defn}\label{thrm:dg::area::manifold}
  Пусть $M$~--- подмногообразие $\R^n$.  Тогда 
  \[
      \lambda_k := \int _D \sqrt {\det g(t)} \, \del t, \quad
      g(t)_{ij} = \left(\pder{x}{t_i} \cdot \pder{x}{t_j}\right)\, (t)
  \]
\end{defn}

\begin{rem}
  Как видно, в $\R^2$, $g$ очень похож на матрицу 1ой квадратичной формы
\end{rem}

\begin{defn}\label{defn:dg::area::isom}
  Пусть $M_1$, $M_2$~--- пара поверхностей. Допустим, $\exists\, F \colon M_1 \to M_2$, 
  сохраняющее длины кривых. Тогда они называются  изометричными.
\end{defn}

\begin{thrm}\label{thrm:dg::area::Iisom}
  Пусть $M_1$, $M_2$~--- пара поверхностей. Допустим, что существуют их параметризации, 
  при которых $I_1= I_2$. Тогда они изометричны.
\end{thrm}

\paragraph{Вторая квадратичная форма}
\label{par:dg::II}

\begin{defn}\label{defn:dg::II}
  Снова рассмотрим поверхность с  какой-то параметризацией. Тогда 
  ${\rm II} := - \del r \, \del n = L \, \del u^2 + 2N\, \del u \,\del v + M\, \del v^2$.
\end{defn}

\begin{prop}\label{prop:dg::II}
  ${\rm II} = n \cdot \del^2 r$
\end{prop}

\begin{prop}[Типы точек на поверхности]\label{prop:dg::II::ptypes}
  Здесь названия связаны с типом соприкасающегося параболоида. Его можно добыть, рассматривая
  $\Delta r \cdot n$.
  \begin{description}
    \item[${\rm II} > 0$:] Эллиптический
    \item[${\rm II} < 0$:] Он же
    \item[${\rm II} \lessgtr 0$:] Гиперболический
    \item[${\rm II} \geqslant 0 \lor {\rm II} \leqslant 0$:] Параболический (вроде цилиндра)
    \item[${\rm II} = 0$:] Точка уплощения
  \end{description}
\end{prop}


\paragraph{Нормальная кривизна в данном направлении. Главные кривизны}
\label{par:meas::curfctr}

\begin{defn}\label{defn:meas::curfctr::normsec}
  Нормальное сечение поверхности~--- сечение плоскостью, 
  содержащей нормаль к поверхности (в точке).
\end{defn}

\begin{lem}\label{lem:meas::curfctr::normsec}
  Нормальное сечение~--- кривая.
\end{lem}

Сначала рассмотрим несколько более общий случай

\begin{thrm}[Менье]\label{thrm:meas::curfctr::menie}
  Пусть $\gamma$~--- кривая $ \subset M$, $ \gamma \ni P$.
  Тогда $k_0 = k_1 \cos (\underbrace{\nu \,\text{\^;}\, n}_\theta) = \frac{\rm II}{\rm I} $.
\end{thrm}

\begin{rem}\label{rem:meas::curfctr::menie::ref}
  Ещё можно сформулировать так: для всякой кривой на повехности, проходящей через точку 
  в заданном направлении $k_0 = \rm const$
\end{rem}

а теперь сузим обратно.
\begin{defn}\label{defn:meas::curfctr::normcrvf}
  Нормальная кривизна~--- кривизна нормального сечения.
\end{defn}

Для нормального сечения $\cos\theta = \pm 1$.

Если немного переписать и ввести параметр $t= \del v /\del u $
\[
  k_1(t) = |k_0(t)| = \left|\frac{L + 2Nt + Mt^2}{E + 2Ft +Gt^2} \right|
\]
Этот параметр $t$ и задаёт <<направление>> нормального сечения. 
Так что $k_0(t)$ и есть та самая <<кривизна в данном направлении>>.

Теперь найдем экстремумы $\frac{\rm II}{\rm I} (t)$. 
\begin{thrm}\label{thrm:meas::curfctr::minmax}
  $\exists\, k_{\min}, k_{\max}$, $k_{\min} \cdot k_{\max} = \frac{LM - N^2}{EG - F^2}$.
\end{thrm}

\begin{defn}\label{defn:meas::curfctr::main}
  $k_{\min}, k_{\max}$~--- главные кривызны.
\end{defn}


\paragraph{Гауссова кривизна поверхности. Теорема Гаусса}
\label{par:meas::gauss}

\begin{defn}[Гауссова кривизна]\label{defn:meas::gauss::crvf}
  $K = k_{\min} \cdot k_{\max}$.
\end{defn}

\begin{defn}[Гауссово отображение]\label{defn:meas::gauss::map}
  Пусть $M$~--- поверхность, $n$~--- нормаль к ней в точке $P$, $S$~--- единичная сфера.
  Тогда $G: n \mapsto C \in S$ ($C$~--- точка на сфере).
\end{defn}

\begin{thrm}\label{thrm:meas::gauss::lim}
  Пусть $U$~--- окрестность $P \subset M$, $M$~--- поверхность, $\mathcal N$~--- поле нормалей
  на $U$. Допустим, что $V = G(\mathcal N)$, она вроде как окрестность $G(n_P)$. 

  Тогда \[
    |K| = \lim_{U \to P} \frac{\iint_V |n_u \times n_v|}{\iint_U|r_u \times r_v|} 
  \]
\end{thrm}

\paragraph{Геодезическая кривизна. Теорема Гаусса-Бонне.}
\label{par:meas::bonnet}

\begin{defn}[Геодезическая кривизна]\label{defn:meas::bonnet::geodcurv}
  Пусть $M$~--- поверхность, $T$~--- касательная к ней в точке $P$. Допустим,
  $\gamma \subset M$ проходит через $P$. Рассмотрим проекцию $\gamma$ на $T$.
  Тогда $\varkappa := k_\gamma $~--- и есть геодезическая кривизна.
\end{defn}

\begin{defn}\label{defn:meas::bonnet::geodcurv}
  Если для кривой $\varkappa(s) \equiv 0$, то она называется геодезической.
\end{defn}

\begin{thrm}[Гаусса-Бонне]\label{thrm:meas::bonnet}
  Пусть $M$~--- гладкая поверхность, $P_1, \dotsc, P_n$~--- вершины криволинейного многоугольника,
  $P_i, P_{i+1} = \gamma $, $\alpha_i$~--- углы при вершинах. Тогда \[
    \sum_i \alpha_i + \sum_i \int _{\gamma_i} \kappa \, \del s  = 2 \pi - \iint_P K \, \del s
  \]
\end{thrm}




\end{document}
% vim: tw=100 cc=100



\chapter{Анализ Фурье \underdev}
\setcounter{paragraph}{49}
%------------------------------------------------------------
% Description : Fourier series
% Author      : Iliya Tikhonenko <iliya.t@mail.ru>
% Created at  : Fri Jun  2 12:47:32 MSK 2017
%------------------------------------------------------------
\documentclass[12pt,draft,timbord]{longnotes}
\usepackage{tmath}
\usepackage{cussymb}
\usepackage{silence}
\WarningFilter{latex}{Reference}
\graphicspath{{../../img/}}

\begin{document}

\paragraph{Гильбертово пространство. \texorpdfstring{$\summb_2$}{L2}}
\label{par:fourier::hilb}

\begin{defn}\label{defn:fourier::hilb}
  Пусть $H$~--- линейное пространство над полем $\C$. Введём на нём (эрмитово) скалярное
  произведение, связанную с ним норму и метрику. Допустим, оно полно по введённой метрике.
  Тогда $H$~--- гильбертово пространство.
\end{defn}

\begin{rem}
  Если полноты нет, то пространство называется предгильбертовым.
\end{rem}

\begin{stat}\label{stat:fourier::hilb::contscm}
  Скалярное произведение~--- непрерывно.
\end{stat}
\begin{exmp}\label{exmp:fourier::hilb::L2}
  Пусть $(X,\mu)$~--- пространство  с мерой. Рассмотрим пространство $\ov~{L}$
  \[
    \ov~{L}:= \left\{f ~\middle|~ f\colon X \to \C, \text{ измерима}, \int_X |f|^2 \, \del \mu < \infty\right\}
  \]
  Скалярное произведение зададим так:
  \[
    \langle f,g \rangle = \int_X f \cdot \ov-{g} \, \del \mu 
  \]
  Введем теперь отношение эквивалентности $f\sim g := f = g \alev$. 
  Тогда $\summb_2 = \bigslant{\ov~{L}}{\sim}$.
\end{exmp}

\begin{thrm}\label{thrm:fourier::hilb::comp}
  $\summb_2$ полно по мере, введённой выше.
\end{thrm}

\paragraph{Ортогональные системы. Ряд Фурье в гильбертовом пространстве.}
\label{par:fourier::orthseries}

\begin{defn}\label{defn:fourier::orthseries::delta}
  $\displaystyle \delta_{ij}
  \begin{cases}
    1, & i =j \\
    0, & i\neq j
  \end{cases}
  $
\end{defn}


\begin{defn}\label{defn:fourier::orthseries}
  Пусть $H$~--- гильбертово. Рассмотрим $f_1, \dotsc, f_n \in H$. 
  Допустим, $\langle f_i,f_j \rangle = \delta_{ij}$. Тогда $(f_i)$~--- ортогональная
  система.
\end{defn}

\begin{thrm}[Пифагора \sour]\label{thrm:fourier::orthseries::pif}
  Пусть $(f_i)$~--- ортогональная система. Допустим, $f = \sum_k f_k$. Тогда
  \[
    \|f\| ^2 = \sum_k \|f_k\|^2
  \]
\end{thrm}

\begin{defn}\label{defn:fourier::orthseries::fs}
  Пусть $(e_i)$~--- ортогональная система, $f \in H$. Тогда
  \[
    \begin{split}
      c_n &= \left\langle f, \frac{e_n}{\|e_n\|} \right\rangle \text{~--- коэффициенты Фурье $f$} \\
      f &=\sum_k c_k e_k  \text{~--- ряд Фурье $f$}
    \end{split}
  \]
  
\end{defn}

\begin{thrm}[Неравенсто Бессля]\label{thrm:fourier::orthseries::ineq}
  Пусть $f\in H$,  $(e_i)$~--- ортогональная система. Тогда \[
    \sum_n |c_n|^2 \|e_n\|^2 \leqslant \|f\|^2
  \]
\end{thrm}

\begin{defn}\label{defn:fourier::orthseries::compl}
  Пусть $(e_i)$~--- ортогональная система. Допустим 
  \[
    \forall\, f \in \summb_2 \holds f \sim \sum_n c_n e_n 
  \]
  Тогда $(e_i)$~--- полная система.
\end{defn}
\begin{stat}\label{stat:fourier::orthseries::uniq}
  Разложение в ряд Фурье по полной ортогональной системе~--- единственно.
\end{stat}

\paragraph{Тригонометрические системы}
\label{par:fourier::trigseries}

\begin{defn}\label{defn:fourier::trigseries::space}
  $\summb_2^{2\pi} = \summb_2\bigl((0;2\pi),\mu\bigr) \, \cap $ $\{2\pi$-периодичные функции$\}$.
\end{defn}

\begin{prop}\label{prop:fourier::trigseries::orthsin}
  $1, \cos x, \sin x, \cos 2x, \dotsc $~--- ортогональная система
\end{prop}
\begin{prop}\label{prop:fourier::trigseries::orthexp}
  $1, e^x, e^{2x}, \dotsc $~--- ортогональная система
\end{prop}

\begin{thrm}\label{thrm:fourier::trigseries::comp}
  Тригонометрические системы выше~--- полны.
\end{thrm}
\begin{tproof}
  \quest Вообще, тут большой кусок теории.
\end{tproof}

\begin{defn}\label{defn:fourier::trigseries::vp}
  Будем понимать 
  \[
    \sum_{-\infty}^\infty a_n := \Vp \sum_{-\infty}^\infty a_n 
    = \lim_{N\to +\infty}  \sum_{-N}^N a_n
  \]
\end{defn}

\begin{prop}\label{prop:fourier::trigseries::sin}
  Коэффициенты разложения по синусам и косинусам:
   \begin{align*}
     a_n &= \frac{1}{\pi} \int_0^{2\pi} f(x) \, \cos nx \, \del x \; (n \geqslant 1)\\ 
     b_n &= \frac{1}{\pi} \int_0^{2\pi} f(x) \, \sin nx \, \del x \; (n \geqslant 1)\\ 
     a_0 &= \frac{1}{2\pi} \int_0^{2\pi} f(x) \, \del x \\ 
     \ov~{a_0} &= \frac{1}{\pi} \int_0^{2\pi} f(x) \, \del x  = 2 a_0\\ 
     f(x) \sim a_0 + \sum_{k=1}^\infty a_n \, \cos nx + \sum_{k=1}^\infty b_n \, \sin nx  
   \end{align*}
\end{prop}
\begin{prop}\label{prop:fourier::trigseries::exp}
  Коэффициенты разложения по экспонентам:
   \begin{align*}
     c_n &= \frac{1}{2\pi} \int_0^{2\pi} f(x) \, e^{-inx} \, \del x \\ 
     f(x) &\sim \sum_{k=-\times}^\infty c_n  e^{inx}
   \end{align*}
\end{prop}

\paragraph{Ядро Дирихле. Лемма Римана-Лебега}
\label{par:fourier::dirker}

\begin{defn}[Ядро Дирихле]\label{defn:fourier::dirker::dir}
  $\mathcal D_n(x) := \dsum_{-n}^n e^{ikx}$
\end{defn}
\begin{lem}[Свойства ядра Дирихле]\label{lem:fourier::dirker::dir}
  ${}$
  \begin{enumerate}
    \item $\mathcal D_n(-x) = \mathcal D(x)$
    \item $\mathcal D_n(x) = \dfrac{\sin(n+\lfrac{1}{2})x}{\sin \frac x 2}$
    \item всякие следствия отсюда
  \end{enumerate}
\end{lem}
\begin{defn}[Ядро Фейера]\label{defn:fourier::dirker::fey}
  $\mathcal F_n(x) := \frac{1}{n} \dsum_{k=0}^{n-1} \mathcal D_k(x)$
\end{defn}
\begin{lem}[Свойства ядра Фейера]\label{lem:fourier::dirker::fey}
  ${}$
  \begin{enumerate}
    \item $\mathcal F_n(-x) = \mathcal F(x)$
    \item $\mathcal F_n(x)= \dfrac1 n \cdot\dfrac{\sin^2\left(\frac{nx}{2}\right)}{\sin^2\frac x 2}$
    \item всякие следствия отсюда
  \end{enumerate}
\end{lem}

\begin{lem}[Римана-Лебега]\label{lem:fourier::dirker::rimleb}
  Пусть $f\in \summb(\R)$. Тогда
  \begin{align*}
    \intR f(x) \sin nx \, x &\xto{n\to \infty} 0 \\
    \intR f(x) \cos nx \, x &\xto{n\to \infty} 0\\
    \intR f(x) e^{-inx} \, x &\xto{n\to \infty} 0\\
  \end{align*}
\end{lem}


\paragraph{Теорема Дини о поточечной сходимости}
\label{par:fourier::dini}

\begin{thrm}[Дини]\label{thrm:fourier::dini}
  Пусть $f\in \summb_2^{2\pi}$, $x\in\R$. Допустим, $f$ удовлетворяет условию Дини:
  \[
    \exists\, L \in \C, \delta > 0 \that u \in \summb\bigl((0;\delta)\bigr), 
    u(t) = \frac{f(x+t) + f(x-t) - 2L}{t}
  \]
  Тогда 
  \[
    S_n(x) = \sum_{k=-n}^{n} c_n e^{ikx} \xto{n \to \infty} L
  \]
\end{thrm}

\begin{prop}\label{prop:fourier::dini::dinisimp}
  Частные случаи условия Дини:
  \begin{enumerate}
    \item $\exists$ конечные $f(x\pm0)$, $f'(x\pm 0)$.
      При этом $L = \frac{1}{2} (f(x+0) + f(x-0))$.
    \item $f$ непрерывна в $x$, $\exists$ конечные $f'(x\pm 0)$.
      При этом $L = f(x)$.
    \item $f \diffin x$.
      При этом $L = f(x)$.
  \end{enumerate}
\end{prop}

\paragraph{Свойства коэффициентов Фурье}
\label{par:fourier::coef}

{\denot $\displaystyle\ov^{f}(n):= c_n = \frac{1}{2\pi}\int_{-\pi}^\pi f(x)\,e^{-inx}\,\del x$}

\begin{prop}\label{prop:fourier::coef::tozero}
  $f\in \summb_2^{2\pi} \Rightarrow \ov^{f} (n) \xto{n\to \infty } 0$ 
\end{prop}
\begin{prop}\label{prop:fourier::coef::deriv}
  Пусть $\exists\, f' \in \summb_2^{2\pi}$. Тогда
  \begin{itemize}
    \item $\ov^{f'} (n) = in \ov^{f}(n)$ 
    \item $\ov^{f}(n) = o\left(\frac{1}{n}\right)$, $n\to \infty$ 
  \end{itemize}
\end{prop}
\begin{prop}\label{prop:fourier::coef::nder}
  Пусть $\exists\, f^{(p)} \in \summb_2^{2\pi}$. Тогда
  \begin{itemize}
    \item $\ov^{f^{(p)}} (n) = (in)^p \cdot  \ov^{f'}(n)$ 
    \item $\ov^{f}(n) = o\left(\frac{1}{n^p}\right)$, $n\to \infty$  
  \end{itemize}
\end{prop}

\begin{prop}\label{prop:fourier::coef::equiv}
  Пусть $\displaystyle c_n = O \left(\frac{1}{n^{p+2}}\right)$. 
  Тогда $\exists\, \varphi \in C^p_{2\pi}\that \varphi \sim f$.
\end{prop}

\paragraph{Сходимость рядов Фурье..}
\label{par:fourier::conv}

\begin{enumerate}[$1^\circ$]
  \item $f\in \summb_1^{2\pi} \Rightarrow 
    \forall\, \Delta \subset [-\pi,\pi] \holds 
    \dint_\Delta f(x) \, \del x = \sum_{-\infty}^\infty \ov^{f} (n) \int_\Delta e^{inx}\, \del x$.
  \item $f\in \summb_1^{2\pi} \Rightarrow c_n$ определены.
  \item $f\in \summb_2^{2\pi} \Rightarrow \|S_n - f\| \to 0$.
  \item $f\in C^{(p)} \Rightarrow c_n$ быстро убывают.
  \item $c_n$ быстро убывают $\Rightarrow f\in C^{(p)}$ .
  \item теорема Дини \ref{thrm:fourier::dini}
  \item теорема Фейера \ref{thrm:fourier::conv::fey}
\end{enumerate}

\begin{thrm}[Фейера]\label{thrm:fourier::conv::fey}
  Пусть $f\in C^{2\pi}$. Тогда $ \sigma_n \unito^\R f$, 
  где $\sigma_n = \frac{1}{n} \dsum_{k=0}^{n-1} S_k$. (сходимость по Чезаро).
\end{thrm}

\paragraph{Преобразование Фурье}
\label{par:fourier::transform}

\begin{defn}\label{defn:fourier::transform}
  Пусть $f \in \summb_1 (\R)$. Тогда\[
    \ov^{f} (s) := \frac{1}{2\pi}  \intR f(x) e^{-isx} \, \del x
  \]
\end{defn}

\begin{enumerate}
  \item $|\ov^{f} (s) | \leqslant \frac{1}{2\pi} \|f\|_1$.
  \item $\ov^{f}(s) \in C^0 $.
  \item $\Bigl(g(x) = x^n f(x) \in \summb_1 \Bigr)\Rightarrow \ov^{f}(s) \in C^{(n)} $.
  \item $\ov^f(s) \xto{s \to \infty} 0 $.
  \item $\Bigl( f\in C^{(p)},f^{(p)}\in \summb_1\Bigr)
    \Rightarrow\ov^f(s) = o\left(\frac{1}{|s|^p}\right)$ .
  \item $f\in \summb_1, a\in \R, g(x) = f(x-a)$ $ \Rightarrow $ $\ov^g(s) = e^{-isa}\,\ov^f(s)$
  \item $f, g\in \summb_1$. Тогда \[
      \ov^{f\ast g} (s) = 2 \pi \,(\ov^f(s) \cdot \ov^g(s))
    \]
  \item Интегральная формула Фурье \ref{thrm:fourier::transform::recov}
\end{enumerate}

\begin{thrm}[формула восстановления Дини]\label{thrm:fourier::transform::recov}
  Пусть $f\in \summb_1(\R)$, $x\in \R$, $L\in \C$ \note{Тут по идее все можно в $\C$}.
  Допустим $f$ удовлетворяет условию Дини в точке $x$ с константой $L$. Тогда
  \[
    \check{\vphantom{\sum}\hat{f}} (x) = L
  \]
  Для непрерывных функций \[
    f(x) = \Vp \intR \ov^f(s) \, e^{isx} \, \del x
  \]
\end{thrm}

\paragraph{Решение уравнения теплопроводности}
\label{par:fourier::heat}

Само уравнение теплопроводности выглядит так:
\[
  \pder{u}{t} = a^2 \cdot \frac{\partial^2 u}{\partial x^2} 
\]
Но к нему ещё есть пара начальных условий:
\begin{align*}
  u(x,0) = f(x) \\
  f \in \summb \qquad f \in C^2_x
\end{align*}

\underdev: \plholdev{решить что-ли..}

В итоге получится что-то вроде\[
  u(x,t) = \frac{1}{2a\sqrt{\pi t}}  \cdot \intR \, \exp \left(-\frac{(x-y)^2}{4a^2 t}\right)\cdot f(y) \, \del y 
\]

\end{document}
%vim: tw=100 cc=100


\clearpage

\appendix
\chapter{Обозначения}
%------------------------------------------------------------
<<<<<<< HEAD
% Description : 
=======
% Description : Стандартные обозначения 
>>>>>>> 3833706880c79d3dba418a8cf75be15927795956
% Author      : taxus-d <iliya.t@mail.ru>
% Created at  : Sat Jan 14 17:39:43 MSK 2017
%------------------------------------------------------------
\documentclass[12pt]{../../notes}
\usepackage{silence}
\WarningFilter{latex}{Reference}
\graphicspath{{../../img/}}


\begin{document}

\noindent\rule{\textwidth}{0.01em}

\begin{description}
  \item[\underdev]~--- ещё правится. Впрочем, относится почти ко всему.
  \item[$\square\cdots\blacksquare$]~--- начало и конец доказательства теоремы
  \item[$\blacktriangledown\cdots\blacktriangle$]~--- начало и конец доказательства более мелкого 
    утверждения
  \item[\sour]~--- кривоватая формулировка
  \item[\flame]~--- набирающему зело не нравится билет
  \item[\plholdev{что-то}]~--- тут будет \texttt{что-то}, но попозже
  \item[$a\intrng b$]~--- для $a,b \in \Z$ это просто $[a;b]\cap \Z$
  \item[$\equiv$]~--- штуки эквивалентны. Часто используется в этом смысле в
    определениях, когда вводится два разных обозначения одного и того же
    объекта.
\end{description}
\end{document}





\end{document}

