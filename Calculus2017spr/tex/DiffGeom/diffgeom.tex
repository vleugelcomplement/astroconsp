%------------------------------------------------------------
% Description : Differential Geometry
% Author      : Iliya Tikhonenko <iliya.t@mail.ru>
% Created at  : Thu Jun  1 18:33:27 MSK 2017
%------------------------------------------------------------
\documentclass[12pt,timbord]{longnotes}
\usepackage{tmath} 
\usepackage{cussymb} 
\usepackage{silence}
\WarningFilter{latex}{Reference}
\graphicspath{{../../img/}}

\begin{document}
\paragraph{Регулярная кривая и её естественная параметризация}
\label{par:dg::curve}

\begin{defn}[Кривая, как отображение]\label{defn:dg::curve::map}
  Пусть задано гладкое отображение $t\in [a;b] \mapsto r(t) \in \R^3$, регулярное, то есть
  $rk r'(t) \equiv 1$. $t$~--- параметр, само отображение ещё можно
  называть параметризацией.

\end{defn}

\begin{defn}[Кривая, как класс отображений]\label{defn:dg::curve::class}
  Введём отношение эквивалентности отображений:
  \[
    r(t) \sim \rho(\tau) \Leftrightarrow 
    \exists\, \delta \colon [a;b] \leftrightarrow [\alpha, \beta ] \that \rho (\delta (t)) = r(t)
  \]
\end{defn}

А теперь будем их путать. \flame

\begin{defn}[Естественная параметризация]\label{defn:dg::curve::nat}
  Пусть $[a;b] = [t_0, t_1]$.
  Рассмотрим $\ov~s(t) = \dint_{t_0}^t | r'(t) | \, \del \tau $. Она, как видно, является 
  пройденным путём и неубывает  $ \Rightarrow $ годится на роль $\delta$.

  Так что можно рассматривать $s$ как параметр,  это собственно и есть
  естественная (натуральная) параметризация.
\end{defn}


\begin{prop}\label{prop:dg::curve::repar}
  Пусть есть две разных параметризации: $r(t)$ и $r(s)$ одной кривой. Тогда 
  \[
    \dot r \equiv \pder{r(s)}{s} = \Bigl(r'(t) \cdot (s'(t))^{-1}\Bigr)(t) = \frac{r'}{|r'|} 
  \]
  Как видно, натуральная почему-то обозначается точкой.
\end{prop}

\paragraph{Кривизна кривой}
\label{par:dg::curvature}

\begin{defn}[Касатальный вектор]\label{defn:dg::curvature::tang}
  $\tau := \dot r(s)$.
\end{defn}

\begin{defn}[Кривизна]\label{defn:dg::curvature}
  $k_1 = |\dot \tau |$
\end{defn}
\begin{defn}[Радиус кривизны]\label{defn:dg::curvature}
  $R = k_1 ^{-1}$
\end{defn}

\begin{prop}\label{prop:dg::curvature::orth}
  $\tau \perp \dot \tau$
\end{prop}

\begin{thrm}\label{thrm:dg::curvature::repar}
  $\displaystyle k_1 = \frac{| r' \times r''|}{|r'|^3} $
\end{thrm}

\paragraph{Кручение и нормаль}
\label{par:dg::norm}

\begin{defn}[Нормаль]\label{defn:dg::norm::norm}
  Пусть $k_1 \neq 0$. Тогда $\nu  := \frac{\dot \tau } {k_1}$.
\end{defn}
\begin{defn}[Бинормаль]\label{defn:dg::norm::binorm}
  $\beta = \tau \times \nu $.
\end{defn}

\begin{rem*}
  $(\tau,\nu,\beta)$~--- хороший кандидат для репера в какой-нибудь точке $P$.
\end{rem*}
\begin{defn}[Соприкасающаяся плоскость]\label{defn:dg::norm::tangpl}
  Пусть $k_1 > 0$, $P = r(s_0)$, $T$~--- плоскость, $T \ni P$, $N \perp T$~--- нормаль к ней.
  Допустим, $r(s+\Delta s) \cdot N  = h$, $h = o(\Delta s^2)$. Тогда $T$~--- соприкасающаяся
  плоскость.
\end{defn}
\begin{prop}\label{prop:dg::norm::tangpl}
  $ \tau , \nu \perp N  $;
  $(r-r_0, \dot r_0 , \ddot r_0)=0$~--- её уравнение
\end{prop}

\begin{defn}[Абсолютное кручение]\label{defn:dg::norm::abscrvn}
  $|k_2| := | \dot\beta|$
\end{defn}
\begin{thrm}\label{defn:dg::norm::abscrvn}
  $|k_2| = \left|\dfrac{(\dot r, \ddot r, \dddot r\,)}{k_1^2}\right|$
\end{thrm}

\begin{defn}[Кручение]\label{defn:dg::norm::crvn}
  $k_2 := \dfrac{-(\dot r, \ddot r, \dddot r\,)}{k_1^2}$
\end{defn}

\paragraph{Формулы Френе}
\label{par:dg::frene}

\begin{thrm}\label{thrm:dg::frene}
  \begin{equation}
    \label{eq:dg::frene}
    \begin{pmatrix}
      \dot\tau \\\dot\nu \\ \dot\beta
    \end{pmatrix}
    = 
    \begin{pmatrix}
      0 & k_1 & 0 \\
      -k_1 & 0 & -k_2 \\
      0 & k_2 & 0
    \end{pmatrix}\cdot
    \begin{pmatrix}
      \tau \\ \nu \\ \beta
    \end{pmatrix}
  \end{equation}
\end{thrm}

\begin{thrm}\label{thrm:dg::frene::reconst}
  Пусть $r(s)$~--- гладкая кривая с заданными $k_1$ и $k_2$, $k_1>0$. 
  Тогда система \eqref{eq:dg::frene} определит её с точностью до движения.
\end{thrm}

\paragraph{Регулярная поверхность. Касательная плоскость. Первая квадратичная форма}
\label{par:dg::tangplane}

\begin{defn}[Поверзность (двумерная)]\label{defn:dg::tangplane::manifold}
  Пусть задано гладкое отображение \[
    \varphi \colon (u,v) \in D \subset \R^2 \mapsto r=(x,y,z) \in \R^3
  \]
  Добавим условие регулярности $\rk \varphi' \equiv 2$ и условимся путать отображение и класс 
  оных.
\end{defn}

\begin{defn}\label{defn:dg::tangplane::tanv}
  \[
    \begin{split}
      r_u &:= (x_u', y_u', z_u') \\
      r_v &:= (x_v', y_v', z_v') \\
      n &:= \frac{r_u \times r_v}{|r_u \times r_v|} = \frac{N}{|N|} 
    \end{split}
  \]
  Отметим, что условие регулярности не дает векторному произведению обращаться в 0.
  
  Касательную плоскость можно было бы здесь определить через нормаль, но лучше пока ещё подумать.
  Может, абстракций добавить.
\end{defn}



\begin{defn}[Первая квадратичная форма]\label{defn:dg::tangplane::I}
  \[
    \begin{split}
      I :&= |\del r|^2  = r_u^2\, \del u^2 + 2r_u r_v\, \del u \,\del v + r_v^2\, \del v^2 \\
        &= E \, \del u^2 + 2F\, \del u \,\del v + G\, \del v^2 
    \end{split}
  \]
\end{defn}

\paragraph{Вычисление длин и площадей на поверхности}
\label{par:dg::area}

\begin{thrm}\label{thrm:dg::area::len}
  Пусть $M$~--- поверхность, $\gamma \colon t \to r \in M$.
  Тогда \[
    \ell(\gamma)= \dint_{t_0}^{t_1} \sqrt I.\; (\del s = I)
  \]
\end{thrm}

\begin{thrm}\label{thrm:dg::area::area}
  Пусть $M$~--- поверхность, $u,v\in D$, $I = E \, \del u^2 + 2F\, \del u \,\del v + G\, \del v^2$.
  Тогда
  \[
    S(M) = \iint_D \sqrt{EG - F^2}\, \del u\,\del v
  \]
\end{thrm}


\quest\plholdev{вкусный абстрактный кусок про меру на многообразии}

\begin{defn}\label{thrm:dg::area::manifold}
  Пусть $M$~--- подмногообразие $\R^n$.  Тогда 
  \[
      \lambda_k := \int _D \sqrt {\det g(t)} \, \del t, \quad
      g(t)_{ij} = \left(\pder{x}{t_i} \cdot \pder{x}{t_j}\right)\, (t)
  \]
\end{defn}

\begin{rem}
  Как видно, в $\R^2$, $g$ очень похож на матрицу 1ой квадратичной формы
\end{rem}

\begin{defn}\label{defn:dg::area::isom}
  Пусть $M_1$, $M_2$~--- пара поверхностей. Допустим, $\exists\, F \colon M_1 \to M_2$, 
  сохраняющее длины кривых. Тогда они называются  изометричными.
\end{defn}

\begin{thrm}\label{thrm:dg::area::Iisom}
  Пусть $M_1$, $M_2$~--- пара поверхностей. Допустим, что существуют их параметризации, 
  при которых $I_1= I_2$. Тогда они изометричны.
\end{thrm}

\paragraph{Вторая квадратичная форма}
\label{par:dg::II}

\begin{defn}\label{defn:dg::II}
  Снова рассмотрим поверхность с  какой-то параметризацией. Тогда 
  ${\rm II} := - \del r \, \del n = L \, \del u^2 + 2N\, \del u \,\del v + M\, \del v^2$.
\end{defn}

\begin{prop}\label{prop:dg::II}
  ${\rm II} = n \cdot \del^2 r$
\end{prop}

\begin{prop}[Типы точек на поверхности]\label{prop:dg::II::ptypes}
  Здесь названия связаны с типом соприкасающегося параболоида. Его можно добыть, рассматривая
  $\Delta r \cdot n$.
  \begin{description}
    \item[${\rm II} > 0$:] Эллиптический
    \item[${\rm II} < 0$:] Он же
    \item[${\rm II} \lessgtr 0$:] Гиперболический
    \item[${\rm II} \geqslant 0 \lor {\rm II} \leqslant 0$:] Параболический (вроде цилиндра)
    \item[${\rm II} = 0$:] Точка уплощения
  \end{description}
\end{prop}


\paragraph{Нормальная кривизна в данном направлении. Главные кривизны}
\label{par:meas::curfctr}

\begin{defn}\label{defn:meas::curfctr::normsec}
  Нормальное сечение поверхности~--- сечение плоскостью, 
  содержащей нормаль к поверхности (в точке).
\end{defn}

\begin{lem}\label{lem:meas::curfctr::normsec}
  Нормальное сечение~--- кривая.
\end{lem}

Сначала рассмотрим несколько более общий случай

\begin{thrm}[Менье]\label{thrm:meas::curfctr::menie}
  Пусть $\gamma$~--- кривая $ \subset M$, $ \gamma \ni P$.
  Тогда $k_0 = k_1 \cos (\underbrace{\nu \,\text{\^;}\, n}_\theta) = \frac{\rm II}{\rm I} $.
\end{thrm}

\begin{rem}\label{rem:meas::curfctr::menie::ref}
  Ещё можно сформулировать так: для всякой кривой на повехности, проходящей через точку 
  в заданном направлении $k_0 = \rm const$
\end{rem}

а теперь сузим обратно.
\begin{defn}\label{defn:meas::curfctr::normcrvf}
  Нормальная кривизна~--- кривизна нормального сечения.
\end{defn}

Для нормального сечения $\cos\theta = \pm 1$.

Если немного переписать и ввести параметр $t= \del v /\del u $
\[
  k_1(t) = |k_0(t)| = \left|\frac{L + 2Nt + Mt^2}{E + 2Ft +Gt^2} \right|
\]
Этот параметр $t$ и задаёт <<направление>> нормального сечения. 
Так что $k_0(t)$ и есть та самая <<кривизна в данном направлении>>.

Теперь найдем экстремумы $\frac{\rm II}{\rm I} (t)$. 
\begin{thrm}\label{thrm:meas::curfctr::minmax}
  $\exists\, k_{\min}, k_{\max}$, $k_{\min} \cdot k_{\max} = \frac{LM - N^2}{EG - F^2}$.
\end{thrm}

\begin{defn}\label{defn:meas::curfctr::main}
  $k_{\min}, k_{\max}$~--- главные кривызны.
\end{defn}


\paragraph{Гауссова кривизна поверхности. Теорема Гаусса}
\label{par:meas::gauss}

\begin{defn}[Гауссова кривизна]\label{defn:meas::gauss::crvf}
  $K = k_{\min} \cdot k_{\max}$.
\end{defn}

\begin{defn}[Гауссово отображение]\label{defn:meas::gauss::map}
  Пусть $M$~--- поверхность, $n$~--- нормаль к ней в точке $P$, $S$~--- единичная сфера.
  Тогда $G: n \mapsto C \in S$ ($C$~--- точка на сфере).
\end{defn}

\begin{thrm}\label{thrm:meas::gauss::lim}
  Пусть $U$~--- окрестность $P \subset M$, $M$~--- поверхность, $\mathcal N$~--- поле нормалей
  на $U$. Допустим, что $V = G(\mathcal N)$, она вроде как окрестность $G(n_P)$. 

  Тогда \[
    |K| = \lim_{U \to P} \frac{\iint_V |n_u \times n_v|}{\iint_U|r_u \times r_v|} 
  \]
\end{thrm}

\paragraph{Геодезическая кривизна. Теорема Гаусса-Бонне.}
\label{par:meas::bonnet}

\begin{defn}[Геодезическая кривизна]\label{defn:meas::bonnet::geodcurv}
  Пусть $M$~--- поверхность, $T$~--- касательная к ней в точке $P$. Допустим,
  $\gamma \subset M$ проходит через $P$. Рассмотрим проекцию $\gamma$ на $T$.
  Тогда $\varkappa := k_\gamma $~--- и есть геодезическая кривизна.
\end{defn}

\begin{defn}\label{defn:meas::bonnet::geodcurv}
  Если для кривой $\varkappa(s) \equiv 0$, то она называется геодезической.
\end{defn}

\begin{thrm}[Гаусса-Бонне]\label{thrm:meas::bonnet}
  Пусть $M$~--- гладкая поверхность, $P_1, \dotsc, P_n$~--- вершины криволинейного многоугольника,
  $P_i, P_{i+1} = \gamma $, $\alpha_i$~--- углы при вершинах. Тогда \[
    \sum_i \alpha_i + \sum_i \int _{\gamma_i} \kappa \, \del s  = 2 \pi - \iint_P K \, \del s
  \]
\end{thrm}




\end{document}
% vim: tw=100 cc=100

