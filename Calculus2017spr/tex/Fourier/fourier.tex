%------------------------------------------------------------
% Description : Fourier series
% Author      : Iliya Tikhonenko <iliya.t@mail.ru>
% Created at  : Fri Jun  2 12:47:32 MSK 2017
%------------------------------------------------------------
\documentclass[12pt,draft,timbord]{longnotes}
\usepackage{tmath}
\usepackage{cussymb}
\usepackage{silence}
\WarningFilter{latex}{Reference}
\graphicspath{{../../img/}}

\begin{document}

\paragraph{Гильбертово пространство. \texorpdfstring{$\summb_2$}{L2}}
\label{par:fourier::hilb}

\begin{defn}\label{defn:fourier::hilb}
  Пусть $H$~--- линейное пространство над полем $\C$. Введём на нём (эрмитово) скалярное
  произведение, связанную с ним норму и метрику. Допустим, оно полно по введённой метрике.
  Тогда $H$~--- гильбертово пространство.
\end{defn}

\begin{rem}
  Если полноты нет, то пространство называется предгильбертовым.
\end{rem}

\begin{stat}\label{stat:fourier::hilb::contscm}
  Скалярное произведение~--- непрерывно.
\end{stat}
\begin{exmp}\label{exmp:fourier::hilb::L2}
  Пусть $(X,\mu)$~--- пространство  с мерой. Рассмотрим пространство $\ov~{L}$
  \[
    \ov~{L}:= \left\{f ~\middle|~ f\colon X \to \C, \text{ измерима}, \int_X |f|^2 \, \del \mu < \infty\right\}
  \]
  Скалярное произведение зададим так:
  \[
    \langle f,g \rangle = \int_X f \cdot \ov-{g} \, \del \mu 
  \]
  Введем теперь отношение эквивалентности $f\sim g := f = g \alev$. 
  Тогда $\summb_2 = \bigslant{\ov~{L}}{\sim}$.
\end{exmp}

\begin{thrm}\label{thrm:fourier::hilb::comp}
  $\summb_2$ полно по мере, введённой выше.
\end{thrm}

\paragraph{Ортогональные системы. Ряд Фурье в гильбертовом пространстве.}
\label{par:fourier::orthseries}

\begin{defn}\label{defn:fourier::orthseries::delta}
  $\displaystyle \delta_{ij}
  \begin{cases}
    1, & i =j \\
    0, & i\neq j
  \end{cases}
  $
\end{defn}


\begin{defn}\label{defn:fourier::orthseries}
  Пусть $H$~--- гильбертово. Рассмотрим $f_1, \dotsc, f_n \in H$. 
  Допустим, $\langle f_i,f_j \rangle = \delta_{ij}$. Тогда $(f_i)$~--- ортогональная
  система.
\end{defn}

\begin{thrm}[Пифагора \sour]\label{thrm:fourier::orthseries::pif}
  Пусть $(f_i)$~--- ортогональная система. Допустим, $f = \sum_k f_k$. Тогда
  \[
    \|f\| ^2 = \sum_k \|f_k\|^2
  \]
\end{thrm}

\begin{defn}\label{defn:fourier::orthseries::fs}
  Пусть $(e_i)$~--- ортогональная система, $f \in H$. Тогда
  \[
    \begin{split}
      c_n &= \left\langle f, \frac{e_n}{\|e_n\|} \right\rangle \text{~--- коэффициенты Фурье $f$} \\
      f &=\sum_k c_k e_k  \text{~--- ряд Фурье $f$}
    \end{split}
  \]
  
\end{defn}

\begin{thrm}[Неравенсто Бессля]\label{thrm:fourier::orthseries::ineq}
  Пусть $f\in H$,  $(e_i)$~--- ортогональная система. Тогда \[
    \sum_n |c_n|^2 \|e_n\|^2 \leqslant \|f\|^2
  \]
\end{thrm}

\begin{defn}\label{defn:fourier::orthseries::compl}
  Пусть $(e_i)$~--- ортогональная система. Допустим 
  \[
    \forall\, f \in \summb_2 \holds f \sim \sum_n c_n e_n 
  \]
  Тогда $(e_i)$~--- полная система.
\end{defn}
\begin{stat}\label{stat:fourier::orthseries::uniq}
  Разложение в ряд Фурье по полной ортогональной системе~--- единственно.
\end{stat}

\paragraph{Тригонометрические системы}
\label{par:fourier::trigseries}

\begin{defn}\label{defn:fourier::trigseries::space}
  $\summb_2^{2\pi} = \summb_2\bigl((0;2\pi),\mu\bigr) \, \cap $ $\{2\pi$-периодичные функции$\}$.
\end{defn}

\begin{prop}\label{prop:fourier::trigseries::orthsin}
  $1, \cos x, \sin x, \cos 2x, \dotsc $~--- ортогональная система
\end{prop}
\begin{prop}\label{prop:fourier::trigseries::orthexp}
  $1, e^x, e^{2x}, \dotsc $~--- ортогональная система
\end{prop}

\begin{thrm}\label{thrm:fourier::trigseries::comp}
  Тригонометрические системы выше~--- полны.
\end{thrm}
\begin{tproof}
  \quest Вообще, тут большой кусок теории.
\end{tproof}

\begin{defn}\label{defn:fourier::trigseries::vp}
  Будем понимать 
  \[
    \sum_{-\infty}^\infty a_n := \Vp \sum_{-\infty}^\infty a_n 
    = \lim_{N\to +\infty}  \sum_{-N}^N a_n
  \]
\end{defn}

\begin{prop}\label{prop:fourier::trigseries::sin}
  Коэффициенты разложения по синусам и косинусам:
   \begin{align*}
     a_n &= \frac{1}{\pi} \int_0^{2\pi} f(x) \, \cos nx \, \del x \; (n \geqslant 1)\\ 
     b_n &= \frac{1}{\pi} \int_0^{2\pi} f(x) \, \sin nx \, \del x \; (n \geqslant 1)\\ 
     a_0 &= \frac{1}{2\pi} \int_0^{2\pi} f(x) \, \del x \\ 
     \ov~{a_0} &= \frac{1}{\pi} \int_0^{2\pi} f(x) \, \del x  = 2 a_0\\ 
     f(x) \sim a_0 + \sum_{k=1}^\infty a_n \, \cos nx + \sum_{k=1}^\infty b_n \, \sin nx  
   \end{align*}
\end{prop}
\begin{prop}\label{prop:fourier::trigseries::exp}
  Коэффициенты разложения по экспонентам:
   \begin{align*}
     c_n &= \frac{1}{2\pi} \int_0^{2\pi} f(x) \, e^{-inx} \, \del x \\ 
     f(x) &\sim \sum_{k=-\times}^\infty c_n  e^{inx}
   \end{align*}
\end{prop}

\paragraph{Ядро Дирихле. Лемма Римана-Лебега}
\label{par:fourier::dirker}

\begin{defn}[Ядро Дирихле]\label{defn:fourier::dirker::dir}
  $\mathcal D_n(x) := \dsum_{-n}^n e^{ikx}$
\end{defn}
\begin{lem}[Свойства ядра Дирихле]\label{lem:fourier::dirker::dir}
  ${}$
  \begin{enumerate}
    \item $\mathcal D_n(-x) = \mathcal D(x)$
    \item $\mathcal D_n(x) = \dfrac{\sin(n+\lfrac{1}{2})x}{\sin \frac x 2}$
    \item всякие следствия отсюда
  \end{enumerate}
\end{lem}
\begin{defn}[Ядро Фейера]\label{defn:fourier::dirker::fey}
  $\mathcal F_n(x) := \frac{1}{n} \dsum_{k=0}^{n-1} \mathcal D_k(x)$
\end{defn}
\begin{lem}[Свойства ядра Фейера]\label{lem:fourier::dirker::fey}
  ${}$
  \begin{enumerate}
    \item $\mathcal F_n(-x) = \mathcal F(x)$
    \item $\mathcal F_n(x)= \dfrac1 n \cdot\dfrac{\sin^2\left(\frac{nx}{2}\right)}{\sin^2\frac x 2}$
    \item всякие следствия отсюда
  \end{enumerate}
\end{lem}

\begin{lem}[Римана-Лебега]\label{lem:fourier::dirker::rimleb}
  Пусть $f\in \summb(\R)$. Тогда
  \begin{align*}
    \intR f(x) \sin nx \, x &\xto{n\to \infty} 0 \\
    \intR f(x) \cos nx \, x &\xto{n\to \infty} 0\\
    \intR f(x) e^{-inx} \, x &\xto{n\to \infty} 0\\
  \end{align*}
\end{lem}


\paragraph{Теорема Дини о поточечной сходимости}
\label{par:fourier::dini}

\begin{thrm}[Дини]\label{thrm:fourier::dini}
  Пусть $f\in \summb_2^{2\pi}$, $x\in\R$. Допустим, $f$ удовлетворяет условию Дини:
  \[
    \exists\, L \in \C, \delta > 0 \that u \in \summb\bigl((0;\delta)\bigr), 
    u(t) = \frac{f(x+t) + f(x-t) - 2L}{t}
  \]
  Тогда 
  \[
    S_n(x) = \sum_{k=-n}^{n} c_n e^{ikx} \xto{n \to \infty} L
  \]
\end{thrm}

\begin{prop}\label{prop:fourier::dini::dinisimp}
  Частные случаи условия Дини:
  \begin{enumerate}
    \item $\exists$ конечные $f(x\pm0)$, $f'(x\pm 0)$.
      При этом $L = \frac{1}{2} (f(x+0) + f(x-0))$.
    \item $f$ непрерывна в $x$, $\exists$ конечные $f'(x\pm 0)$.
      При этом $L = f(x)$.
    \item $f \diffin x$.
      При этом $L = f(x)$.
  \end{enumerate}
\end{prop}

\paragraph{Свойства коэффициентов Фурье}
\label{par:fourier::coef}

{\denot $\displaystyle\ov^{f}(n):= c_n = \frac{1}{2\pi}\int_{-\pi}^\pi f(x)\,e^{-inx}\,\del x$}

\begin{prop}\label{prop:fourier::coef::tozero}
  $f\in \summb_2^{2\pi} \Rightarrow \ov^{f} (n) \xto{n\to \infty } 0$ 
\end{prop}
\begin{prop}\label{prop:fourier::coef::deriv}
  Пусть $\exists\, f' \in \summb_2^{2\pi}$. Тогда
  \begin{itemize}
    \item $\ov^{f'} (n) = in \ov^{f}(n)$ 
    \item $\ov^{f}(n) = o\left(\frac{1}{n}\right)$, $n\to \infty$ 
  \end{itemize}
\end{prop}
\begin{prop}\label{prop:fourier::coef::nder}
  Пусть $\exists\, f^{(p)} \in \summb_2^{2\pi}$. Тогда
  \begin{itemize}
    \item $\ov^{f^{(p)}} (n) = (in)^p \cdot  \ov^{f'}(n)$ 
    \item $\ov^{f}(n) = o\left(\frac{1}{n^p}\right)$, $n\to \infty$  
  \end{itemize}
\end{prop}

\begin{prop}\label{prop:fourier::coef::equiv}
  Пусть $\displaystyle c_n = O \left(\frac{1}{n^{p+2}}\right)$. 
  Тогда $\exists\, \varphi \in C^p_{2\pi}\that \varphi \sim f$.
\end{prop}

\paragraph{Сходимость рядов Фурье..}
\label{par:fourier::conv}

\begin{enumerate}[$1^\circ$]
  \item $f\in \summb_1^{2\pi} \Rightarrow 
    \forall\, \Delta \subset [-\pi,\pi] \holds 
    \dint_\Delta f(x) \, \del x = \sum_{-\infty}^\infty \ov^{f} (n) \int_\Delta e^{inx}\, \del x$.
  \item $f\in \summb_1^{2\pi} \Rightarrow c_n$ определены.
  \item $f\in \summb_2^{2\pi} \Rightarrow \|S_n - f\| \to 0$.
  \item $f\in C^{(p)} \Rightarrow c_n$ быстро убывают.
  \item $c_n$ быстро убывают $\Rightarrow f\in C^{(p)}$ .
  \item теорема Дини \ref{thrm:fourier::dini}
  \item теорема Фейера \ref{thrm:fourier::conv::fey}
\end{enumerate}

\begin{thrm}[Фейера]\label{thrm:fourier::conv::fey}
  Пусть $f\in C^{2\pi}$. Тогда $ \sigma_n \unito^\R f$, 
  где $\sigma_n = \frac{1}{n} \dsum_{k=0}^{n-1} S_k$. (сходимость по Чезаро).
\end{thrm}

\paragraph{Преобразование Фурье}
\label{par:fourier::transform}

\begin{defn}\label{defn:fourier::transform}
  Пусть $f \in \summb_1 (\R)$. Тогда\[
    \ov^{f} (s) := \frac{1}{2\pi}  \intR f(x) e^{-isx} \, \del x
  \]
\end{defn}

\begin{enumerate}
  \item $|\ov^{f} (s) | \leqslant \frac{1}{2\pi} \|f\|_1$.
  \item $\ov^{f}(s) \in C^0 $.
  \item $\Bigl(g(x) = x^n f(x) \in \summb_1 \Bigr)\Rightarrow \ov^{f}(s) \in C^{(n)} $.
  \item $\ov^f(s) \xto{s \to \infty} 0 $.
  \item $\Bigl( f\in C^{(p)},f^{(p)}\in \summb_1\Bigr)
    \Rightarrow\ov^f(s) = o\left(\frac{1}{|s|^p}\right)$ .
  \item $f\in \summb_1, a\in \R, g(x) = f(x-a)$ $ \Rightarrow $ $\ov^g(s) = e^{-isa}\,\ov^f(s)$
  \item $f, g\in \summb_1$. Тогда \[
      \ov^{f\ast g} (s) = 2 \pi \,(\ov^f(s) \cdot \ov^g(s))
    \]
  \item Интегральная формула Фурье \ref{thrm:fourier::transform::recov}
\end{enumerate}

\begin{thrm}[формула восстановления Дини]\label{thrm:fourier::transform::recov}
  Пусть $f\in \summb_1(\R)$, $x\in \R$, $L\in \C$ \note{Тут по идее все можно в $\C$}.
  Допустим $f$ удовлетворяет условию Дини в точке $x$ с константой $L$. Тогда
  \[
    \check{\vphantom{\sum}\hat{f}} (x) = L
  \]
  Для непрерывных функций \[
    f(x) = \Vp \intR \ov^f(s) \, e^{isx} \, \del x
  \]
\end{thrm}

\paragraph{Решение уравнения теплопроводности}
\label{par:fourier::heat}

Само уравнение теплопроводности выглядит так:
\[
  \pder{u}{t} = a^2 \cdot \frac{\partial^2 u}{\partial x^2} 
\]
Но к нему ещё есть пара начальных условий:
\begin{align*}
  u(x,0) = f(x) \\
  f \in \summb \qquad f \in C^2_x
\end{align*}

\underdev: \plholdev{решить что-ли..}

В итоге получится что-то вроде\[
  u(x,t) = \frac{1}{2a\sqrt{\pi t}}  \cdot \intR \, \exp \left(-\frac{(x-y)^2}{4a^2 t}\right)\cdot f(y) \, \del y 
\]

\end{document}
%vim: tw=100 cc=100
